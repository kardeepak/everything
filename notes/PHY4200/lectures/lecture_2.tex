
%----------------------------------------------------------------------------------------
%	Lecture 2
%----------------------------------------------------------------------------------------

\chapter{The Domain of Quantum Mechanics}

\bigbreak
\section{Application of Quantum Mechanics}

In Classical Mechanics, everything is certain, predictable and determined.
Whereas in Quantum Mechanics, we have uncertainty. But it is completely predictable.
But instead of predicting the state of the universe, you predict probabilities.

The boundary between Classical and Quantum Mechanics is the boundary between things that are large and small./
This is not a precise way to state things.

Mathematically, Quantum Mechanics apply in the following case : 
\begin{itemize}
	\item When angular momemntum $L \sim \hbar$
	\item When uncertainties $\Delta p \Delta x \sim \hbar$
	\item When uncertainties $\Delta E \Delta t \sim \hbar$.
	\item When action $S \sim \hbar$
\end{itemize}


\begin{exmp}
	Electron in hydrogen atom.
	
	Energy : $10eV \sim \frac{p^2}{2m} \Rightarrow \Delta p \sim 1.7 \times 10^{-24} kg \frac{m}{s}$

	Size of atom : $0.1nm \sim 10^{-10} m = \Delta x$
	
	$$ \Delta p \Delta x \sim 1.7 \times 10^{-34} \sim \hbar $$
\end{exmp}


\begin{exmp}
	Speck of dust in light breeze.

	Here, mass $\sim 10^{-6} kg$ and velocity $\sim 1 m/s$
	and the size of speck of dust $\sim 10^{-5} m$. 
	Then the momemntum will be mass times velocity and the uncertainty in momemntum let's say is $1\%$.
	The uncertainty in position will be similar to the size of the speck.
	So $\Delta x \sim 10^{-5}$.

	$$ p \sim 10^{-6} kg m / s  \quad \Delta p 10^{-8} kg m / s $$ 

	$$ \Delta p \Delta x \sim 10^{-14} \sim 10^{20} \hbar $$

	Here, we can see that the uncertainties are too large so this is in the realm of Classical Physics.

\end{exmp}

\subsubsection*{Example Systems when QM is important}

\begin{itemize}
	\item Single Particles (atoms, molecules, electrons, photons)
	\item Semiconductors (crystals)
	\item Lasers
	\item Low temperatures ( < 100K )
\end{itemize}

We're adding things to this list all the time.

\pagebreak

\begin{prblm}
	If the timescale of interactions of two helium atoms is 10ns, 
	what is the energy scale at which quantum effects become important?
	\begin{proof}[Solution]
		\begin{align*}
			\Delta E \Delta t & \sim \hbar \\
			\Delta E \times 10 \times 10^{-9} s & \sim 6.626 \times 10^{-34} J s \\
			\Delta E & \sim 6.626 \times 10^{-26} J 
		\end{align*}
	\end{proof}
\end{prblm}

\begin{prblm}
	If the energy scale at the temperature $T$ is given by $\Delta E \sim k_B T$ 
	where $k_B = 1.38 \times 10^{-23} J / K$, what temperature must the helium be cooled to
	for quantum effects to become important.
	\begin{proof}[Solution]
		We know $\Delta E$ from the last problem, so 
		\begin{align*}
			\Delta E & \sim 6.626 \times 10^{-26} J \\
			1.38 \times 10^{-23} J/K \times T & \sim 6.626 \times 10^{-26} J \\ 
			T & \sim 4.801 \times 10^{-3} K
		\end{align*}
	\end{proof}
\end{prblm}