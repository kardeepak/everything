
%----------------------------------------------------------------------------------------
%	Lecture 8
%----------------------------------------------------------------------------------------

\chapter{Curve Sketching}

\bigbreak

{\bf Goal: } To draw the graph of $f$ using behaviour of $f'$ and $f''$. 
We want to graph qualitatively but not to scale. 


\section{Rules for Curve Sketching}

\begin{enumerate}
    \item Plot the discontinuities of $f$ especially the infinite ones.
    \item Find the critical points. These are points where $f'(x) = 0$.
    \item Plot the critical points and critical values. Decide the sign of $f'(x)$ in between critical points.
    \item Find and plot the zeros of $f$. Only do this if its relatively easy.
    \item Determine the behaviour at end points (or at $x = \pm \infty$)
\end{enumerate}


\section{Second Derivative Information}

When $f'' > 0, f'$ is increasing. When $f'' < 0, f'$ is decreasing.


\begin{figure}[ht!]
	\centering
	\includegraphics[scale=0.65]{./images/lecture_8_figure_1.png}
	\caption{$f$ is convex (concave up). The slope increases from negative to positive as $x$ increases.}
\end{figure}


\begin{figure}[ht!]
	\centering
	\includegraphics[scale=0.65]{./images/lecture_8_figure_2.png}
	\caption{$f$ is concave down. The slope decreases from positive to negative as $x$ increases.}
\end{figure}


Therefore, the sign of the second derivative tells us about the concavity/convexity of the graph.
Thus the second derivative is good for two purposes.

\begin{enumerate}
    \item Deciding whether a critical point is minimum or maximum. This is known as the \underline{second derivative} test.
    \begin{center}
    \begin{tabular}{ |c|c|c| }
        \hline
        $f'(x)$ & $f''(x)$ & Critical point is a: \\ 
        \hline
        $0$ & negative & maximum \\  
        \hline
        $0$ & positive & minimum \\ 
        \hline
    \end{tabular}
    \end{center}
    \item Concave/Convex ``decoration.''
\end{enumerate}

The points where $f'' = 0$ are called \textit{inflection points}. 
Usually these are the points where graph changes from concave up to concave down, or vice-versa.

\begin{figure}[ht!]
	\centering
	\includegraphics[scale=0.65]{./images/lecture_8_figure_3.png}
	\caption{Inflection point: $y = 3x - x^3, y'' = -6x = 0$, at $x = 0$.}
\end{figure}