
%----------------------------------------------------------------------------------------
%	Lecture 13
%----------------------------------------------------------------------------------------

\chapter{First Fundamental Theorem of Calculus}
\bigbreak

\section{Fundamental Theorem of Calculus (FTC 1)}

\begin{mdframed}
\begin{center}
    If $f(x)$ is continuous and $F'(x) = f(x)$, then \\
    \ilds{ \int^b_a f(x) = F(b) - F(a) }
\end{center}
\end{mdframed}

{\bf Notation: } \ilds{ F(x)\Big|_a^b = F(a)\Big|_{x=a}^{x=b} = F(b) - F(a) }
\bigbreak
{\bf Example 1.} \ilds{F(x) = \frac{x^3}{3}, F'(x) = x^2; \int_a^b x^2 dx = \frac{x^3}{3} \Big|_a^b = \frac{b^3}{3} - \frac{a^3}{3} } 
\bigbreak

Integrals have an additive propery : 
$$
    \int_a^b f(x) dx = \int_b^c f(x) dx = \int_a^c f(x) dx
$$

\underline{New Definition : }
$$
    \int_b^a f(x) dx = - \int_a^b f(x) dx
$$


\subsection{Estimation : }
If $f(x) <= g(x)$, then \ilds{ \int_a^b f(x) dx <= \int_a^b g(x) dx } (only if $a < b$)

\subsection{Change of Variable : }
If $f(x) = g(u(x))$, then we write, $du = u'(x)dx$ and for indefinite integrals
$$ \int g(u) du = \int g(u(x)) u'(x) dx = \int f(x)u'(x)dx $$

For definite integrals, ( where $u_1 = u(x_1)$ and $u_2 = u(x_2)$ ) 
$$ \int_{x_1}^{x_2} f(x)u'(x)dx \int_{u_1}^{u_2} g(u) du $$
