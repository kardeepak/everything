
%----------------------------------------------------------------------------------------
%	Lecture 14
%----------------------------------------------------------------------------------------

\chapter{Second Fundamental Theorem of Calculus}
\bigbreak

\section{Second Fundamental Theorem of Calculus (FTC 2)}

\begin{mdframed}
\begin{center}
	If \ilds{ F(x) = \int_a^x f(t) dt} and $f$ is continuous, then $F'(x) = f(x)$.
\end{center}
\end{mdframed}

{\bf Geometric Proof of FTC 2: } Use the area interpretation: $F(x)$ is the area under the curve between $a$ and $x$.

\begin{align*}
	\Delta F & = F(x + \Delta x) - F(x) \\
	\Delta F & \approx (base)(height) \approx (\Delta x)f(x) \\
	\frac{\Delta F}{\Delta x} & \approx = f(x) \\
	\text{Hence, } \quad \lim_{\Delta x \to 0} \frac{\Delta F}{\Delta x} & = f(x)
\end{align*}

But by definition of derivative, $$\lim_{\Delta x \to 0} \frac{\Delta F}{\Delta x} = F'(x)$$
Therefore, $$F'(x) = f(x)$$

Another way, to prove FTC 2 is as follows : 
\begin{align*}
	\frac{\Delta F}{\Delta x} & = \frac{1}{\Delta x} \left[ \int_a^{x+\Delta x} f(t) dt - \int_a^x f(t) dt \right] \\
		& = \frac{1}{\Delta x} \left[ \int_x^{x+\Delta x} f(t) dt \right] \\
	F'(x) & = \lim_{\Delta x \to 0} \frac{\Delta F}{\Delta x} \\
	F'(x) & = \lim_{\Delta x \to 0} \frac{1}{\Delta x} \left[ \int_x^{x+\Delta x} f(t) dt \right] \\
	F'(x) & = \lim_{\Delta x \to 0} \frac{1}{\Delta x} f(x) \Delta x
\end{align*}

We can approximate the area to $f(x) \Delta x$ as $\Delta x \to 0$. Thus, $$ F'(x) = f(x) $$
