
%----------------------------------------------------------------------------------------
%	Lecture 10
%----------------------------------------------------------------------------------------

\chapter{Differentials And Antiderivatives}

\bigbreak

\section{Differentials}

New notation:
$$ \boxed{ dy = f'(x)dx } \quad ( y = f(x) ) $$

Both $dy$ and $f'(x)dx$ are called \textit{differentials}. 
You can think of \ilds{ \diff{y}{x} = f'(x) } as a quotient of differentials.
One way this is used is for linear approximations.
$$\frac{\Delta y}{\Delta x} \approx \diff{y}{x} $$

\section{Anti-derivatives}
\ilds{ F(x) = \int f(x) dx } means that $F$ is the antiderivative of $f$.
Other ways of saying this are :
$$ F'(x) = f(x) \quad \text{or,} \quad dF = f(x)dx $$

\subsection*{Examples}
\begin{enumerate}
	\item \ilds{ \int \sin x dx = - \cos x + c } where $c$ is any constant.
	\item \ilds{ \int x^n dx = \frac{x^{n+1}}{n+1} + c } where $n \neq -1$.
	\item \ilds{ \int \frac{dx}{x} = \ln(|x|) + c } (This takes care of $n = -1$ in 2.)
	\item \ilds{ \int \sec^2 x dx = \tan x + c }
	\item \ilds{ \int \frac{dx}{\sqrt{1-x^2}} = \sin^{-1} x + c }
	\item \ilds{ \int \frac{dx}{1+x^2} = \tan^{-1} x + c }
\end{enumerate}


\subsection*{Uniquness of Anti-Derivative upto a constant}
If $F'(x) = f(x)$ and $G'(x) = f(x)$, then $G(x) = F(x) + c$ for some constant $c$.

Proof: $$(G-F)' = f - f = 0$$
Recall that we have corollary in Mean Value Theorem that if the derivative of a function is zero then the function is a constant.
Hence, $G(x) - F(x) = c$ (for some constant $c$). That is, $G(x) = F(x) +c$.
