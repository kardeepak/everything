
%----------------------------------------------------------------------------------------
%	Lecture 4
%----------------------------------------------------------------------------------------

\chapter{Chain Rule}  

\bigbreak
\section{Chain Rule}

We've got a general procedure for differentiating expressions with addition, subtraction, multiplication and division.
What about \underline{composition}?

\subsection{Differentiating composition : $y = f(x) = \sin x , x = g(t) = t^2.$}

So, $y = f(g(t)) = \sin t^2$. To find \ilds{ \diff{y}{t} }, write
$$ \frac{\Delta y}{\Delta t} = \frac{\Delta y}{\Delta x} \cdot \frac{\Delta x}{\Delta t} $$

As $\Delta t \to 0, \Delta x \to 0$ too, because of continuity. So we get : 
$$ \diff{y}{t} = \diff{y}{x} \diff{x}{t} \leftarrow \textbf{The Chain Rule!} $$

In this example, \ilds{ \diff{x}{t} = 2t } and \ilds{ \diff{y}{x} = \cos x }.
\begin{align*}
	\text{So, $\quad$} \diff{}{t} \sin(t^2) 
		& = (\diff{y}{x})(\diff{x}{t}) \\
		& = (\cos x)(2t) \\
		& = (2t)(\cos(x^2)) \\
\end{align*}

\subsection{Another Notation for Chain Rule}

$$ \diff{}{t}f(g(t)) = f'(g(t))g'(t) $$
$$ \text{Note : } f \circ g \neq g \circ f $$

\subsection{Extending Power Rule : \ilds{ \diff{}{x} x^n = nx^{n-1} } for $n = 0, -1, -2, -3\ldots$}

There are two ways to proceed. \ilds{ x^{-n} = \left( \frac{1}{x} \right)^n } or \ilds{ x^{-n} = \frac{1}{x^n} }

\begin{enumerate}
	\item \ilds{ \diff{}{x} (x^{-n}) 
		= \diff{}{x} \left( \frac{1}{x} \right)^n 
		= n \left( \frac{1}{x} \right)^{n-1} \left( \frac{-1}{x^2} \right) 
		= -nx^{-(n-1)}x^{-2}
		= -nx^{-n-1}
	}
	\item \ilds{ \diff{}{x} (x^{-n}) 
		= \diff{}{x} \left( \frac{1}{x^n} \right)
		= nx^{n-1} \left( \frac{-1}{x^{2n}} \right)
		= -nx^{-n-1}
	}
	\item \ilds{ \diff{}{x} x^0 = \diff{}{x} 1 = 0 = 0x^{0-1} }
\end{enumerate}



Thus, $$ \diff{}{x} x^n = nx^{n-1} \text{ for } n = \ldots -2, -1, 0, 1, 2, \ldots $$