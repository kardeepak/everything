
%----------------------------------------------------------------------------------------
%	Lecture 5
%----------------------------------------------------------------------------------------

\chapter{Implicit Differentiation}

\bigbreak
\section{Implicit Differentiation}

$$ \diff{}{x} x^a = ax^{a-1} $$

We proved the above for $a = 0, 1, 2\ldots$ explicitly.
From this, we also got the formula for $a = -1, -2\ldots$
Let us try to extend this formula to cover rational numbers, as well:
$ a = \frac{m}{n}; y = x^{\frac{m}{n}}$ where $m$ and $n$ are integers.

We want to compute $\diff{y}{x}$. 
We can say $y^n = x^m$ so $ny^{n-1}\diff{y}{x} = mx^{m-1}$.
Solve for $\diff{y}{x}$ :
$$ \diff{y}{x} = \frac{m}{n} \frac{x^{m-1}}{y^{n-1}} $$
Substituting $y = x^{\frac{m}{n}}$,
\begin{equation*}
\begin{split}
	\diff{y}{x} & = \frac{m}{n} \left( \frac{x^{m-1}}{y^{n-1}} \right) \\
		& = \frac{m}{n} \left( \frac{x^{m-1}}{(x^{m/n})^{n-1}} \right) \\
		& = \frac{m}{n} \left( \frac{x^{m-1}}{x^{m-m/n}} \right) \\
		& = \frac{m}{n} \left( x^{m - 1 - m + m/n} \right) \\
\text{So, } \quad \diff{y}{x} & = \frac{m}{n} x^{\frac{m}{n} - 1} \\
\end{split}
\end{equation*}

This is the same answer we were hoping to get. 
Thus, we have extended the Power Rule to rational numbers.

\subsubsection*{Example 1. $y^3+x^2y+2 = 0$. Find $\diff{y}{x}$}

Here, we can differentiate on both sides to get 
$$ 3y^2\diff{y}{x} + 2xy + x^2\diff{y}{x} = 0 $$

Rearranging, $$ \diff{y}{x} = \frac{-2xy}{x^2+3y^2} $$
