
%----------------------------------------------------------------------------------------
%	Chapter 1 - Problems
%----------------------------------------------------------------------------------------

\chapter*{Problems}

\bigbreak

\begin{prblm}
	If $r$ is rational ($r \neq 0$) and $x$ is irrational, prove that $r + x$ and $rx$ are irrational.
\end{prblm}

\begin{proof}[Solution]
	Assume $r + x = p$ is rational. Since $r, x \in R$ so there exists a $-r$.
	Now $(-r) + p = (-r) + r + x = x$ so now $x = p - r$. Since $p$ and $r$ are rational hence $x$ is also rational.
	This is a contradiction. Hence, $r + x$ is irrational.

	Assume $rx = q$ is rational. Since $r \neq 0$ then there exists a $(1/r)$.
	Now $(1/r)q = (1/r)rx = x$ so now $x = q/r$. Since $q$ and $r$ are rational hence $x$ is also rational.
	This is a contradiction. Hence, $rx$ is irrational.
\end{proof}

\begin{prblm}
	Prove that there is no rational number whose square is $12$.
\end{prblm}

\begin{proof}[Solution]
	Assume $12 = \left( \frac{p}{q} \right)^2$ where $p$ and $q$ are integers with no common factor.
	Now $p^2 = 12q^2$ which shows that $p^2$ is a multiple of $3$.
	This implies that $p$ is a multiple of $3$.

	Let $p = 3r$ so now $9r^2 = 12q^2$.
	Dividing by $3$, we get, $3r^2 = 4q^2$. Now we get $4q^2$ is a multiple of $3$.
	Since $4$ is not a multiple of $3$, hence, $q^2$ must be multiple of $3$.

	Hence, $q$ is a multiple of $3$. This is a contradiction since $p$ and $q$ have no common multiple.
	Thus, there are no such $p$ and $q$.

\end{proof}

\begin{prblm}
	The axioms for multiplication imply the following statements :
	\begin{enumerate}[a)]
		\item If $x \neq 0$ and $xy = xz$ then $y = z$
		\item If $x \neq 0$ and $xy = x$ then $y = 1$
		\item If $x \neq 0$ and $xy = 1$ then $y = (1/x)$
		\item If $x \neq 0$ then $\frac{1}{(1/x)}= x$
	\end{enumerate}
	Prove the above.
\end{prblm}

\begin{proof}[Solution]
	We'll use only the multiplication axioms.
	\begin{enumerate}[a)]
		\item
			By axiom of multiplicative identity, $y = 1y$.

			By axiom of multiplicative inverse, $y = (1/x) x y$ since $x \neq 0$.

			Given $xy = xz$ so $y = (1/x) x z$.

			By axiom of multiplicative inverse, $y = 1z$. By M4, $y = z$. Hence, proved.

		\item Take $z = 1$ in (a) to prove this.
		\item Take $z = (1/x)$ in (a) to prove this.
		\item Now, $(1/x) \frac{1}{(1/x)} = 1 = (1/x) x $ if $x \neq 0$, using (a),
		we get $x = \frac{1}{(1/x)}$. Hence, proved.
	\end{enumerate}
\end{proof}

\begin{prblm}
	Let $E$ be a nonempty subset of an ordered set;
	suppose $\alpha$ is a lower bound of $E$ and $\beta$ is an upper bound of $E$.
	Prove that $\alpha \leq \beta$.
\end{prblm}

\begin{proof}[Solution]
	If $E$ is nonempty and there exists $x \in E$ such that $\alpha \leq x$ and $x \leq \beta$.
	Now there are four cases :
	\begin{enumerate}
		\item $\alpha < x$ and $x < \beta$ implies $\alpha < \beta$ by axiom of ordered sets.
		\item $\alpha = x$ and $x < \beta$ implies $\alpha < \beta$.
		\item $\alpha < x$ and $x = \beta$ implies $\alpha < \beta$.
		\item $\alpha = x$ and $x = \beta$ implies $\alpha = \beta$.
	\end{enumerate}
	Hence, proved that $\alpha \leq \beta$.
\end{proof}


\begin{prblm}
	Let $A$ be a nonempty set of real numbers which is bounded below.
	Let $-A$ be set of all numbers $-x$, where $x \in A$. Prove that
	$$ \text{ inf } A = - \text{sup}(-A) $$
\end{prblm}

\begin{proof}[Solution]
	Since $A$ is nonempty subset of $R$ and bounded below.
	By greatest-lower-bound property of $R$, we have $\alpha = \text{ inf } A$.

	Now for each $x \in A$, $x \geq \alpha$. Let $p = (-x) + (-\alpha)$.
	So by ordered field axiom, we get $p + x \geq p + \alpha \implies (-\alpha) \geq (-x)$.
	Hence, for each $(-x) \in -A$, $(-\alpha) \geq (-x)$. Thus, $(-\alpha)$ is an upper bound of $-A$.
	Let's take $\beta = (-\alpha)$.

	Now if $\gamma < \beta$ then $-\gamma > \alpha$, hence $-\gamma$ is not an lower bound of $A$.
	So there exists $x \in A$ such that $x > (-\gamma)$ which implies $(-x) < \gamma$.
	So for any $\gamma < \beta$ there exists $(-x) \in -A$ such that $(-x) < \gamma$.
	Hence, $\gamma$ is not an upper of $-A$ if $\gamma > \beta$.

	Thus, by definition of least-upper-bound, $\beta = (-\alpha)$ is the least-upper-bound of $A$.

	Thus, $\alpha = \text{ inf }A$ and $\beta = \text{sup}(-A)$ and $\beta = - \alpha \implies \alpha = - \beta$.

	Hence, proved that $\text{ inf } A = - \text{sup}(-A)$.

\end{proof}

\begin{prblm}
	Fix $b > 1$.
	\begin{enumerate}[(a)]
		\item If $m, n, p, q$ are integers, $n > 0$ and $q > 0$, and $r = m / n = p / q$, prove that
		$$ (b^m)^{(1/n)} = (b^p)^{(1/q)} $$
		Hence, it makes sense to define $b^r = (b^m)^{(1/n)}$

		\begin{proof}
			Let $k = mq = np$. Now, $((b^m)^{(1/n)})^{k} = b^{mp}$
			and $((b^p)^{(1/q)})^{k} = b^{mp}$.
			Now, by Theorem \ref{pwrthm}, $(b^m)^{(1/n)} = (b^p)^{(1/q)}$
		\end{proof}


		\item Prove that $b^{r+s} = b^r b^s$ if $r$ and $s$ are rational numbers.

		\begin{proof}
			Let $r = \frac{a}{b}$ and $s = \frac{c}{d}$ where $b > 0$ and $d > 0$ so $r + s = \frac{ad + bc}{bd}$.
			$$ b^{r+s} = (b^{ad + bc})^{(1 / bd)}
						= (b^{ad} b^{bc})^{(1/bd)}
						= (b^{ad})^{(1/bd)} (b^{bc})^{(1/bd)}
						= b^{\frac{a}{d}} b^{\frac{c}{d}}
						= b^r b^s $$
		\end{proof}

		\item If $x$ is real then define $B(x)$ to be the set of all numbers $b^t$, where $t$ is rational and $t \leq x$.
		Prove that $$ b^r = \text{ sup } B(r) $$ when $r$ is rational.
		Hence, it makes sense to define $$ b^x = \text{ sup } B(x) $$ for every real $x$.

		\begin{proof}
			Here, we first redefine, $B(x) = { b^t | t \in Q, t < r }$.

			If $r < t$ and $b > 1$ then $b^r < b^t$. Hence, $b^r$ is an upper bound of $B(r)$.

			If $x <= 0 < b^r$ then $b > 1 > x$ so $x$ is not an upper bound.
			If $0 < x < b^r$ then $b^{1/n} < \frac{b^r}{x}$ then $x < b^{r - (1/n)} \in B(r)$. So $x$ is not an upper bound of $x$.

			Thus, $b^r = \text{ sup } B(r)$. Hence, it makes sense to define $b^x = \text{ sup } B(x)$ for every real $x$
		\end{proof}

		\item Prove that $b^{x+y} = b^{x} b^{y}$ for all real $x$ and $y$.

		\begin{proof}
			Let $p$ be a rational number such that $p < x + y$. Let $r$ be a rational such that $p - y < r < x$ and $s = p - r$.
			Now $r < x$ and $s = p - r < y$. So $b^p = b^r b^s$ as $p, r, s \in Q$.

			So any $b^p \in B(x + y)$ is equivalent to $b^r b^s$ where $b^r \in B(x)$ and $b^s \in B(y)$.
			So $b^r < b^x$ and $b^s < b^y$ so $b^p < b^x b^y$.
			So $M = b^x b^y$ is an upper bound of $B(x + y)$.

			Suppose $0 < c < b^x b^y$ then $\frac{c}{b^x} < b^y$.
			Let $m = (1/2)(\frac{c}{b^x} + b^y)$. Then $\frac{c}{b^x} < m < b^y$,
			and there exist $u \in B(x)$ and $v \in B(y)$ such that $c / m < u$ and $m < v$.
			Now $c = (c/m)m < uv \in B(x + y)$ so $c$ is not an upper bound of $B(x + y)$.

			Thus, $\text{ sup } B(x+y) = b^x b^y$.
		\end{proof}
	\end{enumerate}
\end{prblm}

\begin{prblm}
	Fix $b > 1, y > 0$, and prove that there is unique real number $x$ such that $b^x = y$,
	by completing the following outline.
	\begin{enumerate}[(a)]
		\item For any positive integer $n$, $b^n - 1 \geq n(b-1)$.
		\begin{proof}
			We know that $b^n - a^n = (b - a)( \sum_{k=0}^{n-1} b^{n-k-1} a^k)$.
			If $0 < a < b$ then $a^k < b^k$ for any positive integer $k$. So $$b^{n-k-1}a^k \geq a^{n-k-1}a^k = a^{n-1}$$ where equality is for $k = 0$.
			So, $$ b^n - a^n \geq (b-a) \left( \sum_{k=0}^{n-1} a^{n-1} \right) = (b-a)(n a^{n-1})$$
			Put $a = 1$ to get $$ b^n - 1 \geq n(b-1) $$

		\end{proof}

		\item Hence, $b - 1 \geq n(b^{1/n} - 1)$.

		Replace $b$ with $b^{1/n}$ to prove this.

		\item If $t > 1$ and $n > \frac{(b-1)}{(t-1)}$ then $b^{(1/n)} < t$

		\begin{proof}
			\begin{align*}
				n > \frac{b-1}{t-1} & \geq \frac{n(b^{(1/n)} - 1)}{t-1} \\
				\implies n(t - 1) & \geq n(b^{(1/n)} - 1) \\
				\implies t & \geq b^{(1/n)}
			\end{align*}
		\end{proof}

		\item If $w$ is such that $b^w < y$ then $b^{w + (1/n)} < y$ for sufficiently large $n$.
		\begin{proof}
			Putting $t = y b^{-w}$, we get, $t > 1$ as $b^w < y$.
			Taking $n > \frac{b-1}{t-1}$ we get $b^{(1/n)} < t$ so $b^{w + (1/n)} < y$.

			Hence, proved.
		\end{proof}


		\item If $w$ is such that $b^w > y$ then $b^{w - (1/n)} > y$ for sufficiently large $n$.
		\begin{proof}
			Putting $t = (1/y) b^{w}$, we get, $t > 1$ as $b^w > y$.
			Taking $n > \frac{b-1}{t-1}$ we get $b^{(1/n)} < t$ so $b^{w - (1/n)} > y$.

			Hence, proved.
		\end{proof}

		\item Let $A$ be the set of all $w$ such that $b^w < y$, and show that $x = \sup A$ satisfies $b^x = y$.
		\begin{proof}
			We will show that $b^x < y$ and $b^x > y$ both lead to contradiction.

			Assume $b^x < y$ so there must exists a $n$ such that $b^{x + (1/n)} < y$
			so $x < x + (1/n) \in A$ which contradicts the fact that $x$ is an upper bound of $A$.

			Assume $b^x > y$ so there must exist a $n$ such that $b^{x - (1/n)} > y$
			so $b^w < b^{x - (1/n)}$ implies $w < x - (1/n)$. Thus, $x - (1/n)$ is also an upper bound of $A$
			which contradicts the fact that $x$ is the least upper bound of $A$.

			Hence, $b^x = y$ proved.
		\end{proof}

		\item Let there be two real numbers $x_1 < x_2$ such that $b^{x_1} = b^{x_1} = y$.
		That means, that the set $A = {w | b^w < y}$ has two least upper bounds which is impossible.
		Thus, a real number $x$ is a unique number such that $b^x = y$ for $b > 1$ and $y > 0$.
	\end{enumerate}
\end{prblm}


\begin{prblm}
	Prove that no order can be defined in the complex field that turns it into an ordered field. \\
	{\it Hint: -1 is a square.}
	\begin{proof}[Solution]
		By \ref{ordfprop} (d) we know that a perfect square must be greater than $0$.
		So $1 * 1 = 1 > 0$ and $i * i = -1 > 0$.
		But $1 > 0$ implies that $-1 < 0$ by \ref{ordfprop} (a).
		This is a contradiction. Thus, the complex field cannot be an ordered field.
	\end{proof}
\end{prblm}

\begin{prblm}
	Suppose $z = a + bi$, $w = c + di$. Define $z < w$ if $a < c$, and also if $a = c$ but $b < d$.
	Prove that this turns the set of all complex numbers into an ordered set.
	(This type of order relation is called a dictionary order, or lexicographic order, for obvious reasons.)
	Does this ordered set have the least-upper-bound property?

	\begin{proof}[Solution]
		The proof of \ref{ords} (a) is trivial. The real numbers are an ordered set so taking all the cases of
		order between $a, c$ and $b, d$ will give us 9 cases which all show that one of the relations is true.

		To prove \ref{ords} (b), let's take $z = a + bi$, $w = c + di$ and $x = e + fi$.
		We are given $z < w$ and $w < x$. Now we have four cases,
		\begin{enumerate}
			\item $a < c$ and $c < e$ implies $a < c$ so $z < x$.
			\item $a < c$ and $c = e$ but $d < f$ implies $a < e$ so $z < x$.
			\item $a = c$ but $b < d$ and $c < e$ implies $a < e$ so $z < x$.
			\item $a = c$ but $b < d$ and $c = e$ but $d < f$ implies $a = e$ but $b < d$ so $z < x$.
		\end{enumerate}
		Thus, we have show that if $z < w$ and $w < x$ then $z < x$.
		Hence, proved.
	\end{proof}
\end{prblm}

\begin{prblm}
	Suppese $z = a + bi$ and $w = u + vi$, and
	$$	a = \left( \frac{|w| + u}{2} \right)^{1/2} , \quad
		b = \left( \frac{|w| - u}{2} \right)^{1/2} $$
	Prove that $z^2 = w$ if $v \geq 0$ and that $(\overline{z})^2 = w$ if $v < 0$.
	Conclude that every complex number (with one exception!) has two complex square roots.

	\begin{proof}[Solution]
		\begin{align*}
			z^2 & = (a + bi)(a + bi) = a^2 + abi + abi + b^2 i^2 \\
				& = a^2 - b^2 + 2abi \\
				& = \frac{|w|+u}{2} - \frac{|w|-u}{2}
					+ 2 \left( \frac{|w|+u}{2} \right)^{1/2} \left( \frac{|w|-u}{2} \right)^{1/2} \\
				& = u + (|w|^2 - u^2)^{1/2} i \\
				& = u + |v|i
		\end{align*}
		Similarly, $(\overline{z})^2 = u - |v|i$.

		So if $v \geq 0$ then $z^2 = w$ and if $v < 0$ then $(\overline{z})^2 = w$.

		Thus, we've shown that for every complex number $w$ there are two square roots $z = \pm ( a + bi )$
		where $a$ and $b$ are defined above. Except when $w = 0$.
	\end{proof}
\end{prblm}

\begin{prblm}
	If $z$ is a complex number, prove that there exists an $r \geq 0$ and a complex number $w$ with $|w| = 1$
	such that $z = rw$. Are $w$ and $r$ always uniquely determined by $z$?
	\begin{proof}[Solution]
		Take $z = a + bi$, $r = \sqrt{a^2 + b^2}$ and $w = z / r$. We already have $r \geq 0$
		Now, $w = \frac{a}{\sqrt{a^2+b^2}} + \frac{bi}{\sqrt{a^2+b^2}}$.
		So $|w| = \frac{a^2}{a^2+b^2} + \frac{b^2}{a^2+b^2} = 1$. Hence, proved.

		We can see that for each $z$, there is one and only one value of $r$ and $w$.
		Thus, $r$ and $w$ are uniquely determined by $z$.

	\end{proof}
\end{prblm}


\begin{prblm}
	If $z_1, z_2, \dots, z_n$ are complex numbers, prove that
	$$ | z_1 + z_2 + \dots + z_n | \leq |z_1| + |z_2| + \dots + |z_n| $$
	\begin{proof}[Solution]
		This can be done by repeated application of the triangle inequality, that is,
		$|x + y| \leq |x| + |y|$.
	\end{proof}
\end{prblm}

\begin{prblm}
	If $x, y$ are complex then, prove that $\big| |x| - |y| \big| \leq |x - y|$
	\begin{proof}[Solution]
		Here, $x'$ is the complex conjugate of $x$.
		\begin{align*}
		|x-y|^2
			& = (x-y)(x'-y') \\
			& = xx' - xy' - x'y + yy' \\
			& = |x|^2 - 2Re(xy') + |y|^2 \\
			& <= |x|^2 -2|xy'| + |y|^2 \\
			& = |x|^2 - 2|x||y| + |y|^2 \\
			& = (|x| - |y|)^2
		\end{align*}
		Thus, by \ref{pwrthm}, we have $|x-y| \leq \big| |x|-|y| \big|$
	\end{proof}
\end{prblm}


\begin{prblm}
	Given $|z| = 1$. Find $|1+z|^2 + |1-z|^2$
	\begin{proof}[Solution]
		\begin{align*}
		|1+z|^2 + |1-z|^2
			& = (1+z)(1+z') + (1-z)(1-z') \\
			& = 1 + zz' + z + z' + 1 + zz' - z - z' \\
			& = 2 + 2|z|^2 = 4
		\end{align*}
	\end{proof}
\end{prblm}


\begin{prblm}
	Under what conditions does equality bold in the Schwarz inequality?
	\begin{proof}[Solution]
		The theorem shows us that equality holds if $B = 0$
		or $Ba_j - Cb_j = 0$ for all $j$. That is, $a_j$ are proportional of $b_j$
	\end{proof}
\end{prblm}

\begin{prblm}
	Suppose $k \geq 3$, $x, y \in \mathbb{R}^k$, $|x-y| = d > 0$, and $r > 0$.
	\begin{proof}
		Take the inequality $|x-y| \leq |x-z| + |y-z|$.
		If there exists some $z$ such that $|z-x| = |z-y| = r$ then $d \leq 2r$ must be true.

		Thus, if $d < 2r$ there are no such $z$.

		If $d = 2r$ then let's
	\end{proof}
\end{prblm}

