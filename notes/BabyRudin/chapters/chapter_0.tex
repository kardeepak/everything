
%----------------------------------------------------------------------------------------
%	Chapter 1
%----------------------------------------------------------------------------------------

\chapter{The Real and Complex Number System}

\bigbreak
\section{Ordered Sets}

\begin{defn}
    \label{ords}
    Let $S$ be a set. An {\it order} on S is a relation, 
    with the following two properties : 
    \begin{enumerate}
        \item If $x \in S$ and $y \in S$ then one and only one of the statements
            $$ x < y ; \quad x = y ; \quad y < x $$
            is true.
        \item If $x, y, z \in S$, if $x < y$ and $y < z$ then $x < z$.
    \end{enumerate}
    It is often convenient to write $y > x$ instead of $x < y$.
    The notation $x \leq y$ means either $x < y$ or  $x = y$. 
    In other words, $x \leq y$ is negation of $x > y$.
\end{defn}

\begin{defn}
    An {\it ordered set} is a set $S$ in which an order is defined.
\end{defn}
For example, the set of rational numbers $Q$ is a ordered set if $r < s$ 
is defined to mean that $s-r$ is a positive rational number.

\begin{defn}
    Suppose $S$ is an ordered set, and $E \subset S$. 
    If there exists a $\beta \in S$ such that $x \leq \beta$ for every $x \in E$, 
    we say that $E$ is {\it bounded above},
    and $\beta$ is an upper bound of $E$
\end{defn}
Lower bounds are defined in the same way (with $\geq$ in place of $\leq$).

\begin{defn}
    \label{supremum}
    Suppose $S$ is an ordered set, $E \subset S$, and $E$ is bounded above. 
    Suppose there extsts an $\alpha \in S$ with the following properties: 
    \begin{enumerate}
        \item $\alpha$ is an upper bound of E. 
        \item If $\gamma \in S$ and $\gamma < \alpha$ then $\gamma$ is not an upper bound of $E$.
    \end{enumerate}
    Then $\alpha$ is called the least upper bound of $E$ 
    [that there is at most one such $\alpha$ is clear from (2).] 
    or the supremum of $E$, and we write $$ \alpha = \text{ sup } E $$
    
    The greatest lower bound, or infimum, of a set $E$ which is bounded below is defined in the same manner: 
    The statement $$ \alpha = \text{ inf } E $$ means that $\alpha$ is a lower bound of $E$ 
    and that no $\beta \in S$ with $\beta > \alpha$ is not a lower bound of $E$.
\end{defn}

\begin{rem}
    If $\alpha \in S$ is the supremum of $E$ then $\alpha$ may or may not belong to $E$.
\end{rem}

\begin{defn}
    An ordered set $S$ is said to have the {\it least-upper-bound property} if the following is true:
    
    If $E \subset S$, $E$ is not empty and $E$ is bounded above, then $\text{ sup } E$ exists.
\end{defn}

\pagebreak

\begin{thm}
    Suppose $S$ is an ordered set with the least-upper-bound property, 
    $B \subset S$, $B$ is not empty and $B$ is bounded below.
    Let $L$ be the set of all lower bounds of $B$ then $$ \alpha = \text{ sup } L $$
    exists in $S$, and $\alpha = \text{ inf } B$.

    In particular, $\text{ inf } B$ exists in $S$.
\end{thm}

\begin{proof}
    Since $B$ is bounded below, $L$ is not empty. 
    \bigbreak 
    For all $x \in L$, $x$ is lower bound of $B$ so $x \leq y$ for all $y \in B$.
    So for all $y \in B$, $y \geq x$ for all $x \in L$.
    Thus, {\it each $y \in B$ is an upper bound of $L$}. 
    Thus, $L$ is bounded above.
    \bigbreak
    Since $L$ is bound above then by least-upper-bound property $\alpha = \text{ sup } L$ exists in $S$. 
    \bigbreak
    Assume that there exists an $x \in B$ such that $x < \alpha$.
    By Definition \ref{supremum}, $x$ is not an upper bound of $L$.
    This is a contradiction since each $x \in B$ is an upper bound of $L$ as shown earlier.
    \bigbreak
    Thus, for all $x \in B$, $x \geq \alpha$ which means that $\alpha$ is a lower bound of $B$. 
    Thus, $\alpha \in L$.
    \bigbreak
    Since $y \leq \alpha$ for all $y \in L$. 
    This implies that if $\gamma > \alpha$ then $\gamma \notin L$.
    Thus, if $\gamma > \alpha$ then $\gamma$ is not a lower bound of $B$. 
    \bigbreak
    Now, we proved that $\alpha$ exists, it is the lower bound of $B$ and if $\gamma > \alpha$ 
    then $\gamma$ is not a lower bound of $B$.
     
    So, $\alpha$ is the {\it greatest-lower-bound} of $B$, that is, $\alpha = \text{ inf } B$.
\end{proof}


\section{Fields}

\begin{defn}
    A {\it field} is a set $F$ with two operations called {\it addition} and {\it multiplication},
    which satisfy the following so-called ``field axioms'' (A), (M) and (D) : 
    \subsubsection{(A) Axiom of Additions}
    \label{add}
    \begin{enumerate}[(\text{A}1)]
        \item If $x \in F$ and $y \in F$ then $x + y \in F$.
        \item Addition is commutative : $x + y = y + x$ for all $x, y \in F$
        \item Addition is associative $x + (y + z) = (x + y) + z$ for all $x, y, z \in F$
        \item $F$ contains an identity element $0$ such that $0 + x = x$ for all $x \in F$
        \item For every $x \in F$, there exists a $(-x) \in F$ such that $x + (-x) = 0$
    \end{enumerate}

    \subsubsection{(M) Axiom of Multiplications}
    \label{mul}
    \begin{enumerate}[(M1)]
        \item If $x \in F$ and $y \in F$ then $xy \in F$ 
        \item Multiplication is commutative : $xy = yx$ for all $x, y \in F$
        \item Multiplication is associative : $x(yz) = (xy)z$ for all $x, y, z \in F$
        \item $F$ contains an identity element $1 \neq 0$ such that $1x = x$ for every $x \in F$
        \item For every $x \in F$ such that $x \neq 0$, there exists an element $(1/x) \in F$ 
        such that $x \cdot (1/x) = 1$
    \end{enumerate}

    \subsubsection{(D) Axiom of Distribution}
    \label{dis}
    $$ x (y + z) = xy + xz $$ holds for every $x, y,, z \in F$
\end{defn}

\begin{rem}
    \begin{enumerate}
        \item One usually writes $ x - y, \frac{x}{y}, x + y + z, xyz, x^2, x^3, 2x, 3x, \dots $ instead of
        $ x + (-y), x \cdot \left( \frac{1}{y} \right), (x+y)+z, (xy)z, xx, xxx, x+x, x+x+x, \dots $.
        \item The field axiums clearly hold in $Q$ if addition and multiplication have their customary meaning.
        Thus, $Q$ is a field.

    \end{enumerate}
    
\end{rem}

\begin{prop}
    \label{addprop}
    The axioms for addition imply the following statements : 
    \begin{enumerate}[a)]
        \item If $x + y = x + z$ then $y = z$
        \item If $x + y = x$ then $y = 0$
        \item If $x + y = 0$ then $y = (-x)$
        \item $-(-x) = x$
    \end{enumerate}
\end{prop}

\begin{proof}
    We'll use only the addition axioms of \ref{add}.
    \begin{enumerate}[a)]
        \item 
            By A4, $y = 0 + y$.         
            By A5, $y = (-x) + x + y$. 

            Given $x + y = x + z$ so $y = (-x) + x + z$. 

            By A5, $y = 0 + z$. By A4, $y = z$. Hence, proved. 
        
        \item Take $z = 0$ in (a) to prove this.
        \item Take $z = (-x)$ in (a) to prove this.
        \item Since, $(-x) + x = 0$, using (c), we get $x = -(-x)$. Hence, proved.
    \end{enumerate}
\end{proof}


\begin{prop}
    \label{mulprop}
    The axioms for multiplication imply the following statements : 
    \begin{enumerate}[a)]
        \item If $x \neq 0$ and $xy = xz$  then $y = z$
        \item If $x \neq 0$ and $xy = x$ then $y = 1$
        \item If $x \neq 0$ and $xy = 1$ then $y = (1/x)$
        \item If $x \neq 0$ then $\frac{1}{(1/x)}= x$
    \end{enumerate}
\end{prop}


\begin{proof}
    We'll use only the multiplication axioms of \ref{mul}.
    \begin{enumerate}[a)]
        \item 
            By M4, $y = 1y$.         
            By M5, $y = (1/x) x y$ since $x \neq 0$. 

            Given $xy = xz$ so $y = (1/x) x z$.

            By M5, $y = 1z$. By M4, $y = z$. Hence, proved. 
        
        \item Take $z = 1$ in (a) to prove this.
        \item Take $z = (1/x)$ in (a) to prove this.
        \item Now, $(1/x) \frac{1}{(1/x)} = 1$ if $x \neq 0$, using (c), 
        we get $x = \frac{1}{(1/x)}$. Hence, proved.
    \end{enumerate}
\end{proof}


\begin{prop}
    \label{disprop}
    The field axioms imply the following statements, for any $x, y, z \in F$.
    \begin{enumerate}[a)]
        \item $0x = 0$
        \item If $x \neq 0$ and $y \neq 0$ then $xy \neq 0$
        \item $(-x)y = -(xy) = x(-y)$
        \item $(-x)(-y) = xy$
    \end{enumerate}
\end{prop}

\begin{proof}
    \begin{enumerate}[a)]
        \item By \ref{dis} (D), $0x + 0x = (0 + 0)x$.
        
        By A4, $0x + 0x = 0x$. By \ref{addprop} (b), we get $0x = 0$. Hence, proved.

        \item Assume that there exist $x, y$ such that $x \neq 0$ and $y \neq 0$ but $xy = 0$
        Sincr $x \neq  0$ and $y \neq 0$, then by \ref{mul} (M5), 
        we know that $\left( \frac{1}{x} \right) x = 1$ 
        and $\left( \frac{1}{y} \right) y = 1$.

        Since $x \neq 0$ and $y \neq 0$. Then using (a), we get :
            $$ 1 = \left( \frac{1}{y} \right) y = \left( \frac{1}{y} \right) 1 y
                = \left( \frac{1}{y} \right) \left( \frac{1}{x} \right) x y 
                = \left( \frac{1}{y} \right) \left( \frac{1}{x} \right) 0 
                = 0
            $$

        This is a contradiction as axiom \ref{mul} (M4) says that $1 \neq 0$.
        Hence, if $x \neq 0$ and $y \neq 0$ then $xy \neq 0$.

        \item By \ref{dis} (D), we have $xy + (-x)y = (x+(-x))y$.
        By A5, we have $xy + (-x)y = 0y$.

        By (a), we have $xy + (-x)y = 0$.
        By \ref{addprop} (c), we have $(-x)y = -(xy)$. Hence, proved.

        Now, $x(-y) = (-y)x = -(yx) = -(xy)$ by first half of (c) 
        and because multiplication is commutative \ref{mul} (M2).

        \item $(-x)(-y) = -(x(-y)) = -(-(xy)) = xy$ by (c) and \ref{addprop} (d).
    \end{enumerate}
\end{proof}


\begin{defn}
    \label{ordf}
    An {\it ordered field} is a field $F$ which is also an ordered set, such that
    \begin{enumerate}[(i)]
        \item If $x, y, z \in F$ and $y < z$ then $x + y < x + z$
        \item If $x, y \in F$, $x > 0$ and $y > 0$ then $xy > 0$
    \end{enumerate}
    If $x > 0$, we call it {\it positive}; if $x < 0$, $x$ is {\it negative}.
\end{defn}

\begin{prop}
    The following statements are true in every ordered field.
    \begin{enumerate}[a)]
        \item If $x > 0$ then $ -x < 0$ and {\it vice versa}.
        \item If $x > 0$ and $y < z$ then $xy < yz$.
        \item If $x < 0$ and $y < z$ then $xy > xz$.
        \item If $x \neq 0$ then $x^2 > 0$. In particular, $1 > 0$
        \item If $0 < x < y$ then $0 < \frac{1}{y} < \frac{1}{x} $
    \end{enumerate}
\end{prop}

\begin{proof}
    \begin{enumerate}[a)]
        \item Since $0 < x$. By \ref{ordf} (i), $(-x) + 0 < (-x) + x = 0$. Thus, $(-x) < 0$.

        Now if $x < 0$ then $0 = (-x) + x < (-x) + 0$. Thus, $(-x) > 0$.
        Hence, proved.

        \item If $y < z$ then by \ref{ordf} (i) $(-y) + y < (-y) + z \Rightarrow z - y > 0$.
        
        Now by \ref{ordf} (ii), $x(z - y) > 0$ because $x > 0$ and $z - y > 0$.
        
        By distributive law, $xz - xy > 0$ implies $xz - xy + xy > 0 + xy$ by \ref{ordf} (i).
        
        Thus, $xz  > xy$. Hence, proved.

        \item If $x < 0$ then $(-x) > 0$ by (a). 
        Since $y < z$ so by (b), $(-x)y < (-x)z$. 

        By \ref{disprop} (c), we have $-(xy) < -(xz)$.
        Now by \ref{ordf} (i), $-(xy) + xy + xz < -(xz) + xy + xz$.
        
        Thus, $xz < xy \Rightarrow xy > xz$. Hence, proved.

        \item If $x \neq 0$ then $x > 0$ and $x < 0$ are the only two possibilites.
        We'll prove each case.
        
        If $x > 0$ then by \ref{ordf} (ii), $x^2 > 0$.

        If $x < 0$, by (a), $(-x) > 0$ so $(-x)(-x) > 0$. 
        By \ref{disprop} (d), $(-x)(-x) = xx = x^2$ so $x^2 > 0$.
        
        Hence, proved. Also $1^2 = 1 \cdot 1 = 1 > 0$.
        
        \item Assume $\frac{1}{x} < 0$. Given $x > 0$, by (b), 
         we have $x \cdot \left( \frac{1}{x} \right) < 0 \left( \frac{1}{x} \right)$.
    
        By \ref{disprop} (a) and \ref{mul} (M5), we have $1 < 0$. 
        This is a contradiction as we proved in (d) that $1 > 0$.

        Thus, $\frac{1}{x} > 0$. Similarly, since $y > 0$ then $\frac{1}{y} > 0$.

        Now by \ref{ordf} (ii), we have $( \frac{1}{x} ) ( \frac{1}{y} ) 
            = ( \frac{1}{y} )( \frac{1}{x} ) > 0$.
        Since $x < y$ so by (b), we get $(\frac{1}{y}) (\frac{1}{x}) x < (\frac{1}{x}) (\frac{1}{y}) y $.
        Now by \ref{mul} M5, we get $( \frac{1}{y} ) < ( \frac{1}{x} )$.
    \end{enumerate}
\end{proof}

\begin{exmp}
    Let's take sets $A$ and $B$ such that $A = \{ r | r^2 < 2, r \in Q \}$
    and $B = \{ r | r^2 > 2, r \in Q \}$.

    We know that $Q$ is a ordered field. Now we will prove that $A$ has no upper-bound in $Q$.
    We know that for all $p \in A$ and  $q \in B$, $p < q$. Thus, $B$ is the set of upper-bounds of $A$.
    And $A$ is the set of lower bounds of $B$.

    Let's take an element $p \in Q$ and $$ q = p - \frac{p^2 - 2}{p + 2} = \frac{2p + 2}{p + 2} $$
    Now $$ q^2 - 2 = \frac{4p^2 + 8p + 4}{p^2 + 4p + 4} - 2 = \frac{2p^2 - 4}{(p+2)^2} = \frac{2(p^2 - 2)}{(p+2)^2} $$

    If $p \in A$ then $p^2 < 2$ which implies $q > p$ and $q^2 - 2 < 0$ so $q \in A$.
    And if $p \in B$ then $p^2 > 2$ which implies $q > p$ and $q^2 - 2 > 0$ so $q \in B$.

    Now let's assume $\alpha \in B$ is the least-upper-bound of $A$.
    But as we've seen we can find $\beta < \alpha$ such that $\beta \in B$.
    This is a contradiction. Thus, there is no least-upper-bound of $A$.
    
    Similarly, there is no greatest-lower-bound of $B$

\end{exmp}

With this example, we have shown that $Q$ doesn't have a least-upper-bound property.

\section{The Real Field}

\begin{thm}
    There exists an ordered field $R$ which has the least-upper-bound property.
    
    Moreover, $R$ contains $Q$ as a subfield.
\end{thm}

\subsection{Proof}
\begin{enumerate}[{\bf Step 1.}]
\item
The members of the field $R$ will be certain subsets of $Q$ called {\it cuts}.
A cut is by definition, any $\alpha \subset Q$ with the following three properties.

\begin{enumerate}[(I)]
    \item $\alpha$ is not empty, and $\alpha \neq Q$
    \item If $p \in \alpha$, $q \in Q$ and $q < p$ then $q \in \alpha$
    \item If $p \in \alpha$, then $p < r$ for some $r \in \alpha$
\end{enumerate}

The letters $p, q, r, \dots$ always denote rational numbers 
while $\alpha, \beta, \gamma, \dots$ will denote cuts.

Note that (III) simply says that there is no largest member.
(II) implies the following facts which will be used freely : 

If $p \in \alpha$ and $q \notin \alpha$, then $q > p$. 
[ Assume $q < p$ then by (II) implies that $q \in \alpha$. Contradiction. ]

If $r \notin \alpha$ and $r < s$ then $s \notin \alpha$.
[ Assume $s \in \alpha$ then (II) implies that $r \in alpha$. Contradiction. ]

\item 
Define $\alpha < \beta$ to mean: $\alpha$ is a {\it proper subset} of $\beta$.

Let us check that it meets the requirements of \ref{ords}.

It is clear that if $\alpha < \beta$ and $\beta < \gamma$ then $\alpha < \gamma$.
It is also clear that atmost one of the three conditions will hold : 
$$ \alpha < \beta; \quad \alpha = \beta; \quad \beta < \alpha $$

Now we need to prove that atleast of these conditions hold.

Let's assume that first two of the above statements don't hold true for some $\alpha$ and $\beta$.
Then $\alpha$ is not a subset of $\beta$.
So there exists $p \in \alpha$ such that $p \notin \beta$.

Now, for all $q \in \beta$, we have $p > q$. 
So by (II), we have $q \in \alpha$. Thus, $\beta < \alpha$.

We can assume the last two to be false and prove the first one exactly the same way.

Now if first and last statements are false then we need to prove that they are equal.
Let's assume they are not equal.

Then there exists $p \in \alpha$ and $q \in \beta$ such that $q \notin \alpha$ and $p \notin \beta$.
This means that $p \neq q$.

Since $p, q \in Q$, $p \neq q$ and $Q$ is an ordered set. 
So either $p < q$ or $q < p$.

If $p < q$ then $p \in \beta$ by (II). If $q < p$ then $q \in \alpha$ by (II).
This is a contradiction. Hence, our assumption that $\alpha$ and $\beta$ are not equal must be wrong.

Thus, $R$ is an ordered set.

\item The ordered set has least upper bound property. 

To prove this, let $A$ be a non-empty subset of $R$ and assume that $\beta \in R$ is an upper bound of $A$.
Define $\gamma$ to be the union of all elements of $A$.
Thus, $p \in \gamma$ if and only if $p \in \alpha$ for some $\alpha \in A$.
We shall prove that $\gamma \in R$ and $\gamma = \text{ sup } A$.

Since $A$ is not empty so there exists an $\alpha \in A$.
Since $\alpha \in R$ so $\alpha$ is not empty. 
Thus, $\gamma$ is not empty as it is union of all such $\alpha$.

Since for each $\alpha \in A$, by definition of upper bound $\alpha \leq \beta \Rightarrow \alpha \subseteq \beta$ so $\gamma \subseteq \beta$. 
Therefore, $\gamma \neq Q$.

If $p \in \gamma$, $q \in Q$ and $q < p$ then there exists some $\alpha \in A$ such that $p \in \alpha$.
Therefore, by (II), $q \in \alpha$ which implies that $q \in \gamma$ since $\alpha \subseteq \gamma$.

If $p \in \gamma$ then $p \in \alpha$ for some $\alpha \in A$. So $p < r$ for some $r \in \alpha$.
Since $r \in \alpha$ so $r \in \gamma$ as $\alpha \subseteq \gamma$.
Thus, if $p \in \gamma$ then $p < r$ for some $r \in \gamma$.

Hence, proved that $\gamma \in R$.

For each $\alpha \in A$, $\alpha \subseteq \gamma \Rightarrow \alpha \leq \gamma$. Thus, $\gamma$ is an upper bound of $A$.

Suppose $\delta < \gamma$. Then there is an $s \in \gamma$ such that $s \notin \delta$.
But $s \in \alpha$ for some $\alpha \in A$. So $\alpha$ is not a subset of $\delta$.
Thus, $\delta$ is not an upper bound of $A$.

Hence, proved that $\gamma = \text{ sup } A$.

\item If $\alpha \in R$ and $\beta \in R$ we define $\alpha + \beta$ to be the set of all sums $r + s$,
where $r \in \alpha$ and $s \in \beta$.

We defind $0^*$ to be the set of all negative rational number.
It is clear that $0^*$ is a cut. 

We verify the axioms for addition at \ref{add} hold in $R$ with $0^*$ playing the role of $0$.

\begin{enumerate}[(\text{A}1)]
    \item If $\alpha \in R$ and $\beta \in R$ then $\alpha + \beta \in R$.
    
    We need to show that $\alpha + \beta$ is a cut. 
    Since $\alpha$ is not empty and $\beta$ is not empty so $\alpha + \beta$ is not empty as well.
    
    If $p \notin \alpha$ and $q \notin \beta$ then $p > r$ for each $r \in \alpha$ and $q > s$ for each $s \in \alpha$.
    Thus, $p + q > r + s$ for each $r + s \in \alpha + \beta$. Thus, $p + q \notin \alpha + \beta$.
    Thus, $\alpha + \beta \neq Q$. Thus, it satisfies (I).

    If $p = r + s \in \alpha + \beta$ with $r \in \alpha$ and $s \in \beta$, and $q < r + s$
    then let $r' = q - s < r$ so $q - s \in \alpha$ thus $q = (q - s) + s \in \alpha + \beta$.
    Thus, it satisfies (II).

    If $p = r + s \in \alpha + \beta$ with $r \in \alpha$ and $s \in \beta$ then there exists
    $r' in \alpha$ such that $r' > r$ so $q = r' + s \in \alpha + \beta$ is such that $q > p$. 
    Thus, it satisfies (III).

    Hence, $\alpha + \beta \in R$.

    \item Since $\alpha + \beta = \{ r + s | r \in \alpha , s \in \beta \} = \{ s + r | s \in \beta, r \in \alpha \} = \beta + \alpha$. 
    Because addition is commutative in rational numbers.
    
    Thus, addition is commutative.

    \item Same as above as $Q$ follows the associative law.
    
    \item If $r \in \alpha$ and $s \in 0^*$ then $s < 0$ which means $r + s < r$.
    By (II), $r + s \in \alpha$. Thus, $\alpha + 0^* \subseteq \alpha$.

    If $p \in \alpha$ then there is $r > p$ for some $r \in \alpha$.
    Thus, $p - r < 0$ so $p - r \in 0^*$. Now, $(p - r) + r = p \in \alpha + 0^*$.
    Thus, $\alpha \subseteq \alpha + 0^*$. So $\alpha = \alpha + 0^*$.

    \item Fix $\alpha \in R$. Let $\beta$ be a set of all $p$ with the following property : 
    
    {\it There exists $r > 0$ such that $- p - r \notin \alpha$.}

    In other words, some rational number less than $- p$ fails to be in $\alpha$.
    
    {\it We need show that $\beta \in R$ and $\alpha + \beta = 0^*$ }

    If $s \notin \alpha$ and $p = - s - 1$ then $- p - 1 = s \notin \alpha$ so $p \in \beta$. Thus, $\beta$ is not empty.
    If $q \in \alpha$ then $-q \notin \beta$ ( because $-(-q) - r = q - r < q$ so $q-r \in \alpha$ ). Thus, $\beta \neq Q$. 
    Hence, $\beta$ satisfies (I).

    If $p \in \beta$ and $q < p$ and $ - p - r \notin \alpha$ then $ -p < -q $ so $ - p - r < - q - r $. 
    Since $ - p - r \notin \alpha$, by (II), $ - q - r \notin \alpha $. Thus, $q \in\beta$. Hence, $\beta$ satisfies (II).

    If $p \in \beta$ and $ - p - r \notin \alpha$ and $q = p + r/2$ then $ - q - r/2 = - p - r \notin \alpha$ and $r > 0$ so $q > p$.
    Hence, $\beta$ satisfies (III).

    Thus, $\beta \in R$.

    Now, $\alpha + \beta$ is the set of all sums $p + q$ where $p \in \alpha$ and $q \in \beta$. 
    We need to show that $p + q$ is a negative rational number for all values of $p$ and $q$.
    Since $q \in \beta$ we have $ - q - r \notin \alpha$ for some $r > 0$. 
    Thus, $ - q - r > p  \Rightarrow - r > p + q $.
    Since $r > 0$ so $- r < 0$. Thus, $p + q < 0$. So $p + q \in 0^*$ for all values of $p$ and $q$.
    Thus, $\alpha + \beta \subseteq 0^*$

    Now pick $v \in 0^*$ and $w = -v/2$. Then $w > 0$, and there is an integer $n$ such that $nw \in \alpha$
    but $(n+1)w \notin \alpha$. (This is based on the archimedian proprty of rational numbers.)
    Put $p = -(n+2)w$. Now $p \in \beta$ because $- p - w \notin \alpha$ and $v = nw + p \in \alpha + \beta$.
    Thus, $0^* \subseteq \alpha + \beta$. 

    Hence, $\alpha + \beta = 0^*$. This $\beta$ will be deonted by $-\alpha$.

\end{enumerate}

\item Having proved the addition axioms defined in \ref{add} (A),
it follows the Proposition \ref{addprop} is valid in $R$.

Now we can prove one of the requirements in definition \ref{ordf}.

\end{enumerate}