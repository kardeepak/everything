
%----------------------------------------------------------------------------------------
%	Chapter 1
%----------------------------------------------------------------------------------------

\chapter{The Real and Complex Number System}

\bigbreak
\section{Ordered Sets}

\begin{defn}
	\label{ords}
	Let $S$ be a set. An {\it order} on S is a relation,
	with the following two properties :
	\begin{enumerate}
		\item If $x \in S$ and $y \in S$ then one and only one of the statements
			$$ x < y ; \quad x = y ; \quad y < x $$
			is true.
		\item If $x, y, z \in S$, if $x < y$ and $y < z$ then $x < z$.
	\end{enumerate}
	It is often convenient to write $y > x$ instead of $x < y$.
	The notation $x \leq y$ means either $x < y$ or $x = y$.
	In other words, $x \leq y$ is negation of $x > y$.
\end{defn}

\begin{defn}
	An {\it ordered set} is a set $S$ in which an order is defined.
\end{defn}
For example, the set of rational numbers $Q$ is a ordered set if $r < s$
is defined to mean that $s-r$ is a positive rational number.

\begin{defn}
	Suppose $S$ is an ordered set, and $E \subset S$.
	If there exists a $\beta \in S$ such that $x \leq \beta$ for every $x \in E$,
	we say that $E$ is {\it bounded above},
	and $\beta$ is an upper bound of $E$
\end{defn}
Lower bounds are defined in the same way (with $\geq$ in place of $\leq$).

\begin{defn}
	\label{supremum}
	Suppose $S$ is an ordered set, $E \subset S$, and $E$ is bounded above.
	Suppose there extsts an $\alpha \in S$ with the following properties:
	\begin{enumerate}
		\item $\alpha$ is an upper bound of E.
		\item If $\gamma \in S$ and $\gamma < \alpha$ then $\gamma$ is not an upper bound of $E$.
	\end{enumerate}
	Then $\alpha$ is called the least upper bound of $E$
	[that there is at most one such $\alpha$ is clear from (2).]
	or the supremum of $E$, and we write $$ \alpha = \text{ sup } E $$

	The greatest lower bound, or infimum, of a set $E$ which is bounded below is defined in the same manner:
	The statement $$ \alpha = \text{ inf } E $$ means that $\alpha$ is a lower bound of $E$
	and that no $\beta \in S$ with $\beta > \alpha$ is not a lower bound of $E$.
\end{defn}

\begin{rem}
	If $\alpha \in S$ is the supremum of $E$ then $\alpha$ may or may not belong to $E$.
\end{rem}

\begin{defn}
	An ordered set $S$ is said to have the {\it least-upper-bound property} if the following is true:

	If $E \subset S$, $E$ is not empty and $E$ is bounded above, then $\text{ sup } E$ exists.
\end{defn}

\pagebreak

\begin{thm}
	Suppose $S$ is an ordered set with the least-upper-bound property,
	$B \subset S$, $B$ is not empty and $B$ is bounded below.
	Let $L$ be the set of all lower bounds of $B$ then $$ \alpha = \text{ sup } L $$
	exists in $S$, and $\alpha = \text{ inf } B$.

	In particular, $\text{ inf } B$ exists in $S$.
\end{thm}

\begin{proof}
	Since $B$ is bounded below, $L$ is not empty.
	\bigbreak
	For all $x \in L$, $x$ is lower bound of $B$ so $x \leq y$ for all $y \in B$.
	So for all $y \in B$, $y \geq x$ for all $x \in L$.
	Thus, {\it each $y \in B$ is an upper bound of $L$}.
	Thus, $L$ is bounded above.
	\bigbreak
	Since $L$ is bound above then by least-upper-bound property $\alpha = \text{ sup } L$ exists in $S$.
	\bigbreak
	Assume that there exists an $x \in B$ such that $x < \alpha$.
	By Definition \ref{supremum}, $x$ is not an upper bound of $L$.
	This is a contradiction since each $x \in B$ is an upper bound of $L$ as shown earlier.
	\bigbreak
	Thus, for all $x \in B$, $x \geq \alpha$ which means that $\alpha$ is a lower bound of $B$.
	Thus, $\alpha \in L$.
	\bigbreak
	Since $y \leq \alpha$ for all $y \in L$.
	This implies that if $\gamma > \alpha$ then $\gamma \notin L$.
	Thus, if $\gamma > \alpha$ then $\gamma$ is not a lower bound of $B$.
	\bigbreak
	Now, we proved that $\alpha$ exists, it is the lower bound of $B$ and if $\gamma > \alpha$
	then $\gamma$ is not a lower bound of $B$.

	So, $\alpha$ is the {\it greatest-lower-bound} of $B$, that is, $\alpha = \text{ inf } B$.
\end{proof}


\section{Fields}

\begin{defn}
	A {\it field} is a set $F$ with two operations called {\it addition} and {\it multiplication},
	which satisfy the following so-called ``field axioms'' (A), (M) and (D) :
	\subsubsection{(A) Axiom of Additions}
	\label{add}
	\begin{enumerate}[(\text{A}1)]
		\item If $x \in F$ and $y \in F$ then $x + y \in F$.
		\item Addition is commutative : $x + y = y + x$ for all $x, y \in F$
		\item Addition is associative $x + (y + z) = (x + y) + z$ for all $x, y, z \in F$
		\item $F$ contains an identity element $0$ such that $0 + x = x$ for all $x \in F$
		\item For every $x \in F$, there exists a $(-x) \in F$ such that $x + (-x) = 0$
	\end{enumerate}

	\subsubsection{(M) Axiom of Multiplications}
	\label{mul}
	\begin{enumerate}[(M1)]
		\item If $x \in F$ and $y \in F$ then $xy \in F$
		\item Multiplication is commutative : $xy = yx$ for all $x, y \in F$
		\item Multiplication is associative : $x(yz) = (xy)z$ for all $x, y, z \in F$
		\item $F$ contains an identity element $1 \neq 0$ such that $1x = x$ for every $x \in F$
		\item For every $x \in F$ such that $x \neq 0$, there exists an element $(1/x) \in F$
		such that $x \cdot (1/x) = 1$
	\end{enumerate}

	\subsubsection{(D) Axiom of Distribution}
	\label{dis}
	$$ x (y + z) = xy + xz $$ holds for every $x, y,, z \in F$
\end{defn}

\begin{rem}
	\begin{enumerate}
		\item One usually writes $ x - y, \frac{x}{y}, x + y + z, xyz, x^2, x^3, 2x, 3x, \dots $ instead of
		$ x + (-y), x \cdot \left( \frac{1}{y} \right), (x+y)+z, (xy)z, xx, xxx, x+x, x+x+x, \dots $.
		\item The field axioms clearly hold in $Q$ if addition and multiplication have their customary meaning.
		Thus, $Q$ is a field.

	\end{enumerate}

\end{rem}

\begin{prop}
	\label{addprop}
	The axioms for addition imply the following statements :
	\begin{enumerate}[a)]
		\item If $x + y = x + z$ then $y = z$
		\item If $x + y = x$ then $y = 0$
		\item If $x + y = 0$ then $y = (-x)$
		\item $-(-x) = x$
	\end{enumerate}
\end{prop}

\begin{proof}
	We'll use only the addition axioms of \ref{add}.
	\begin{enumerate}[a)]
		\item
			By A4, $y = 0 + y$.
			By A5, $y = (-x) + x + y$.

			Given $x + y = x + z$ so $y = (-x) + x + z$.

			By A5, $y = 0 + z$. By A4, $y = z$. Hence, proved.

		\item Take $z = 0$ in (a) to prove this.
		\item Take $z = (-x)$ in (a) to prove this.
		\item Since, $(-x) + x = 0$, using (c), we get $x = -(-x)$. Hence, proved.
	\end{enumerate}
\end{proof}


\begin{prop}
	\label{mulprop}
	The axioms for multiplication imply the following statements :
	\begin{enumerate}[a)]
		\item If $x \neq 0$ and $xy = xz$ then $y = z$
		\item If $x \neq 0$ and $xy = x$ then $y = 1$
		\item If $x \neq 0$ and $xy = 1$ then $y = (1/x)$
		\item If $x \neq 0$ then $\frac{1}{(1/x)}= x$
	\end{enumerate}
\end{prop}


\begin{proof}
	We'll use only the multiplication axioms of \ref{mul}.
	\begin{enumerate}[a)]
		\item
			By M4, $y = 1y$.
			By M5, $y = (1/x) x y$ since $x \neq 0$.

			Given $xy = xz$ so $y = (1/x) x z$.

			By M5, $y = 1z$. By M4, $y = z$. Hence, proved.

		\item Take $z = 1$ in (a) to prove this.
		\item Take $z = (1/x)$ in (a) to prove this.
		\item Now, $(1/x) \frac{1}{(1/x)} = 1$ if $x \neq 0$, using (c),
		we get $x = \frac{1}{(1/x)}$. Hence, proved.
	\end{enumerate}
\end{proof}


\begin{prop}
	\label{disprop}
	The field axioms imply the following statements, for any $x, y, z \in F$.
	\begin{enumerate}[a)]
		\item $0x = 0$
		\item If $x \neq 0$ and $y \neq 0$ then $xy \neq 0$
		\item $(-x)y = -(xy) = x(-y)$
		\item $(-x)(-y) = xy$
	\end{enumerate}
\end{prop}

\begin{proof}
	\begin{enumerate}[a)]
		\item By \ref{dis} (D), $0x + 0x = (0 + 0)x$.

		By A4, $0x + 0x = 0x$. By \ref{addprop} (b), we get $0x = 0$. Hence, proved.

		\item Assume that there exist $x, y$ such that $x \neq 0$ and $y \neq 0$ but $xy = 0$
		Sincr $x \neq 0$ and $y \neq 0$, then by \ref{mul} (M5),
		we know that $\left( \frac{1}{x} \right) x = 1$
		and $\left( \frac{1}{y} \right) y = 1$.

		Since $x \neq 0$ and $y \neq 0$. Then using (a), we get :
			$$ 1 = \left( \frac{1}{y} \right) y = \left( \frac{1}{y} \right) 1 y
				= \left( \frac{1}{y} \right) \left( \frac{1}{x} \right) x y
				= \left( \frac{1}{y} \right) \left( \frac{1}{x} \right) 0
				= 0
			$$

		This is a contradiction as axiom \ref{mul} (M4) says that $1 \neq 0$.
		Hence, if $x \neq 0$ and $y \neq 0$ then $xy \neq 0$.

		\item By \ref{dis} (D), we have $xy + (-x)y = (x+(-x))y$.
		By A5, we have $xy + (-x)y = 0y$.

		By (a), we have $xy + (-x)y = 0$.
		By \ref{addprop} (c), we have $(-x)y = -(xy)$. Hence, proved.

		Now, $x(-y) = (-y)x = -(yx) = -(xy)$ by first half of (c)
		and because multiplication is commutative \ref{mul} (M2).

		\item $(-x)(-y) = -(x(-y)) = -(-(xy)) = xy$ by (c) and \ref{addprop} (d).
	\end{enumerate}
\end{proof}


\begin{defn}
	\label{ordf}
	An {\it ordered field} is a field $F$ which is also an ordered set, such that
	\begin{enumerate}[(i)]
		\item If $x, y, z \in F$ and $y < z$ then $x + y < x + z$
		\item If $x, y \in F$, $x > 0$ and $y > 0$ then $xy > 0$
	\end{enumerate}
	If $x > 0$, we call it {\it positive}; if $x < 0$, $x$ is {\it negative}.
\end{defn}

\begin{prop}
	\label{ordfprop}
	The following statements are true in every ordered field.
	\begin{enumerate}[a)]
		\item If $x > 0$ then $ -x < 0$ and {\it vice versa}.
		\item If $x > 0$ and $y < z$ then $xy < yz$.
		\item If $x < 0$ and $y < z$ then $xy > xz$.
		\item If $x \neq 0$ then $x^2 > 0$. In particular, $1 > 0$
		\item If $0 < x < y$ then $0 < \frac{1}{y} < \frac{1}{x} $
	\end{enumerate}
\end{prop}

\begin{proof}
	\begin{enumerate}[a)]
		\item Since $0 < x$. By \ref{ordf} (i), $(-x) + 0 < (-x) + x = 0$. Thus, $(-x) < 0$.

		Now if $x < 0$ then $0 = (-x) + x < (-x) + 0$. Thus, $(-x) > 0$.
		Hence, proved.

		\item If $y < z$ then by \ref{ordf} (i) $(-y) + y < (-y) + z \Rightarrow z - y > 0$.

		Now by \ref{ordf} (ii), $x(z - y) > 0$ because $x > 0$ and $z - y > 0$.

		By distributive law, $xz - xy > 0$ implies $xz - xy + xy > 0 + xy$ by \ref{ordf} (i).

		Thus, $xz > xy$. Hence, proved.

		\item If $x < 0$ then $(-x) > 0$ by (a).
		Since $y < z$ so by (b), $(-x)y < (-x)z$.

		By \ref{disprop} (c), we have $-(xy) < -(xz)$.
		Now by \ref{ordf} (i), $-(xy) + xy + xz < -(xz) + xy + xz$.

		Thus, $xz < xy \Rightarrow xy > xz$. Hence, proved.

		\item If $x \neq 0$ then $x > 0$ and $x < 0$ are the only two possibilites.
		We'll prove each case.

		If $x > 0$ then by \ref{ordf} (ii), $x^2 > 0$.

		If $x < 0$, by (a), $(-x) > 0$ so $(-x)(-x) > 0$.
		By \ref{disprop} (d), $(-x)(-x) = xx = x^2$ so $x^2 > 0$.

		Hence, proved. Also $1^2 = 1 \cdot 1 = 1 > 0$.

		\item Assume $\frac{1}{x} < 0$. Given $x > 0$, by (b),
		 we have $x \cdot \left( \frac{1}{x} \right) < 0 \left( \frac{1}{x} \right)$.

		By \ref{disprop} (a) and \ref{mul} (M5), we have $1 < 0$.
		This is a contradiction as we proved in (d) that $1 > 0$.

		Thus, $\frac{1}{x} > 0$. Similarly, since $y > 0$ then $\frac{1}{y} > 0$.

		Now by \ref{ordf} (ii), we have $( \frac{1}{x} ) ( \frac{1}{y} )
			= ( \frac{1}{y} )( \frac{1}{x} ) > 0$.
		Since $x < y$ so by (b), we get $(\frac{1}{y}) (\frac{1}{x}) x < (\frac{1}{x}) (\frac{1}{y}) y $.
		Now by \ref{mul} M5, we get $( \frac{1}{y} ) < ( \frac{1}{x} )$.
	\end{enumerate}
\end{proof}

\begin{exmp}
	Let's take sets $A$ and $B$ such that $A = \{ r | r^2 < 2, r \in Q \}$
	and $B = \{ r | r^2 > 2, r \in Q \}$.

	We know that $Q$ is a ordered field. Now we will prove that $A$ has no upper-bound in $Q$.
	We know that for all $p \in A$ and $q \in B$, $p < q$. Thus, $B$ is the set of upper-bounds of $A$.
	And $A$ is the set of lower bounds of $B$.

	Let's take an element $p \in Q$ and $$ q = p - \frac{p^2 - 2}{p + 2} = \frac{2p + 2}{p + 2} $$
	Now $$ q^2 - 2 = \frac{4p^2 + 8p + 4}{p^2 + 4p + 4} - 2 = \frac{2p^2 - 4}{(p+2)^2} = \frac{2(p^2 - 2)}{(p+2)^2} $$

	If $p \in A$ then $p^2 < 2$ which implies $q > p$ and $q^2 - 2 < 0$ so $q \in A$.
	And if $p \in B$ then $p^2 > 2$ which implies $q > p$ and $q^2 - 2 > 0$ so $q \in B$.

	Now let's assume $\alpha \in B$ is the least-upper-bound of $A$.
	But as we've seen we can find $\beta < \alpha$ such that $\beta \in B$.
	This is a contradiction. Thus, there is no least-upper-bound of $A$.

	Similarly, there is no greatest-lower-bound of $B$

\end{exmp}

With this example, we have shown that $Q$ doesn't have a least-upper-bound property.

\section{The Real Field}

\begin{thm}
	There exists an ordered field $R$ which has the least-upper-bound property.

	Moreover, $R$ contains $Q$ as a subfield.
\end{thm}

\subsection{Proof}
\begin{enumerate}[{\bf Step 1.}]
\item
The members of the field $R$ will be certain subsets of $Q$ called {\it cuts}.
A cut is by definition, any $\alpha \subset Q$ with the following three properties.

\begin{enumerate}[(I)]
	\item $\alpha$ is not empty, and $\alpha \neq Q$
	\item If $p \in \alpha$, $q \in Q$ and $q < p$ then $q \in \alpha$
	\item If $p \in \alpha$, then $p < r$ for some $r \in \alpha$
\end{enumerate}

The letters $p, q, r, \dots$ always denote rational numbers
while $\alpha, \beta, \gamma, \dots$ will denote cuts.

Note that (III) simply says that there is no largest member.
(II) implies the following facts which will be used freely :

If $p \in \alpha$ and $q \notin \alpha$, then $q > p$.
[ Assume $q < p$ then by (II) implies that $q \in \alpha$. Contradiction. ]

If $r \notin \alpha$ and $r < s$ then $s \notin \alpha$.
[ Assume $s \in \alpha$ then (II) implies that $r \in \alpha$. Contradiction. ]

\item
Define $\alpha < \beta$ to mean: $\alpha$ is a {\it proper subset} of $\beta$.

Let us check that it meets the requirements of \ref{ords}.

It is clear that if $\alpha < \beta$ and $\beta < \gamma$ then $\alpha < \gamma$.
It is also clear that atmost one of the three conditions will hold :
$$ \alpha < \beta; \quad \alpha = \beta; \quad \beta < \alpha $$

Now we need to prove that atleast of these conditions hold.

Let's assume that first two of the above statements don't hold true for some $\alpha$ and $\beta$.
Then $\alpha$ is not a subset of $\beta$.
So there exists $p \in \alpha$ such that $p \notin \beta$.

Now, for all $q \in \beta$, we have $p > q$.
So by (II), we have $q \in \alpha$. Thus, $\beta < \alpha$.

We can assume the last two to be false and prove the first one exactly the same way.

Now if first and last statements are false then we need to prove that they are equal.
Let's assume they are not equal.

Then there exists $p \in \alpha$ and $q \in \beta$ such that $q \notin \alpha$ and $p \notin \beta$.
This means that $p \neq q$.

Since $p, q \in Q$, $p \neq q$ and $Q$ is an ordered set.
So either $p < q$ or $q < p$.

If $p < q$ then $p \in \beta$ by (II). If $q < p$ then $q \in \alpha$ by (II).
This is a contradiction. Hence, our assumption that $\alpha$ and $\beta$ are not equal must be wrong.

Thus, $R$ is an ordered set.

\item The ordered set has least upper bound property.

To prove this, let $A$ be a non-empty subset of $R$ and assume that $\beta \in R$ is an upper bound of $A$.
Define $\gamma$ to be the union of all elements of $A$.
Thus, $p \in \gamma$ if and only if $p \in \alpha$ for some $\alpha \in A$.
We shall prove that $\gamma \in R$ and $\gamma = \text{ sup } A$.

Since $A$ is not empty so there exists an $\alpha \in A$.
Since $\alpha \in R$ so $\alpha$ is not empty.
Thus, $\gamma$ is not empty as it is union of all such $\alpha$.

Since for each $\alpha \in A$, by definition of upper bound $\alpha \leq \beta \Rightarrow \alpha \subseteq \beta$ so $\gamma \subseteq \beta$.
Therefore, $\gamma \neq Q$.

If $p \in \gamma$, $q \in Q$ and $q < p$ then there exists some $\alpha \in A$ such that $p \in \alpha$.
Therefore, by (II), $q \in \alpha$ which implies that $q \in \gamma$ since $\alpha \subseteq \gamma$.

If $p \in \gamma$ then $p \in \alpha$ for some $\alpha \in A$. So $p < r$ for some $r \in \alpha$.
Since $r \in \alpha$ so $r \in \gamma$ as $\alpha \subseteq \gamma$.
Thus, if $p \in \gamma$ then $p < r$ for some $r \in \gamma$.

Hence, proved that $\gamma \in R$.

For each $\alpha \in A$, $\alpha \subseteq \gamma \Rightarrow \alpha \leq \gamma$. Thus, $\gamma$ is an upper bound of $A$.

Suppose $\delta < \gamma$. Then there is an $s \in \gamma$ such that $s \notin \delta$.
But $s \in \alpha$ for some $\alpha \in A$. So $\alpha$ is not a subset of $\delta$.
Thus, $\delta$ is not an upper bound of $A$.

Hence, proved that $\gamma = \text{ sup } A$.

\item If $\alpha \in R$ and $\beta \in R$ we define $\alpha + \beta$ to be the set of all sums $r + s$,
where $r \in \alpha$ and $s \in \beta$.

We defind $0^*$ to be the set of all negative rational number.
It is clear that $0^*$ is a cut.

We verify the axioms for addition at \ref{add} hold in $R$ with $0^*$ playing the role of $0$.

\begin{enumerate}[(\text{A}1)]
	\item If $\alpha \in R$ and $\beta \in R$ then $\alpha + \beta \in R$.
	\bigbreak \quad
	We need to show that $\alpha + \beta$ is a cut.
	Since $\alpha$ is not empty and $\beta$ is not empty so $\alpha + \beta$ is not empty as well.
	\bigbreak \quad
	If $p \notin \alpha$ and $q \notin \beta$ then $p > r$ for each $r \in \alpha$ and $q > s$ for each $s \in \alpha$.
	Thus, $p + q > r + s$ for each $r + s \in \alpha + \beta$. Thus, $p + q \notin \alpha + \beta$.
	Thus, $\alpha + \beta \neq Q$. Thus, it satisfies (I).
	\bigbreak \quad
	If $p = r + s \in \alpha + \beta$ with $r \in \alpha$ and $s \in \beta$, and $q < r + s$
	then let $r' = q - s < r$ so $q - s \in \alpha$ thus $q = (q - s) + s \in \alpha + \beta$.
	Thus, it satisfies (II).
	\bigbreak \quad
	If $p = r + s \in \alpha + \beta$ with $r \in \alpha$ and $s \in \beta$ then there exists
	$r' in \alpha$ such that $r' > r$ so $q = r' + s \in \alpha + \beta$ is such that $q > p$.
	Thus, it satisfies (III).
	\bigbreak \quad
	Hence, $\alpha + \beta \in R$.

	\item Since $\alpha + \beta = \{ r + s | r \in \alpha , s \in \beta \} = \{ s + r | s \in \beta, r \in \alpha \} = \beta + \alpha$.
	Because addition is commutative in rational numbers.

	Thus, addition is commutative.

	\item Same as above as $Q$ follows the associative law.

	\item If $r \in \alpha$ and $s \in 0^*$ then $s < 0$ which means $r + s < r$.
	By (II), $r + s \in \alpha$. Thus, $\alpha + 0^* \subseteq \alpha$.
	\bigbreak
	If $p \in \alpha$ then there is $r > p$ for some $r \in \alpha$.
	Thus, $p - r < 0$ so $p - r \in 0^*$. Now, $(p - r) + r = p \in \alpha + 0^*$.
	Thus, $\alpha \subseteq \alpha + 0^*$.
	\bigbreak
	So $\alpha = \alpha + 0^*$.

	\item Fix $\alpha \in R$. Let $\beta$ be a set of all $p$ with the following property :

	{\it There exists $r > 0$ such that $- p - r \notin \alpha$.}
	\bigbreak
	In other words, some rational number less than $- p$ fails to be in $\alpha$.

	{\it We need show that $\beta \in R$ and $\alpha + \beta = 0^*$ }
	\bigbreak \quad
	If $s \notin \alpha$ and $p = - s - 1$ then $- p - 1 = s \notin \alpha$ so $p \in \beta$. Thus, $\beta$ is not empty.
	If $q \in \alpha$ then $-q \notin \beta$ ( because $-(-q) - r = q - r < q$ so $q-r \in \alpha$ ). Thus, $\beta \neq Q$.
	Hence, $\beta$ satisfies (I).
	\bigbreak \quad
	If $p \in \beta$ and $q < p$ and $ - p - r \notin \alpha$ then $ -p < -q $ so $ - p - r < - q - r $.
	Since $ - p - r \notin \alpha$, by (II), $ - q - r \notin \alpha $. Thus, $q \in\beta$. Hence, $\beta$ satisfies (II).
	\bigbreak \quad
	If $p \in \beta$ and $ - p - r \notin \alpha$ and $q = p + r/2$ then $ - q - r/2 = - p - r \notin \alpha$ and $r > 0$ so $q > p$.
	Hence, $\beta$ satisfies (III).
	\bigbreak
	Thus, $\beta \in R$.
	\bigbreak \quad
	Now, $\alpha + \beta$ is the set of all sums $p + q$ where $p \in \alpha$ and $q \in \beta$.
	We need to show that $p + q$ is a negative rational number for all values of $p$ and $q$.
	Since $q \in \beta$ we have $ - q - r \notin \alpha$ for some $r > 0$.
	\bigbreak \quad
	Thus, $ - q - r > p \Rightarrow - r > p + q $.
	Since $r > 0$ so $- r < 0$. Thus, $p + q < 0$. So $p + q \in 0^*$ for all values of $p$ and $q$.
	Thus, $\alpha + \beta \subseteq 0^*$
	\bigbreak \quad
	Now pick $v \in 0^*$ and $w = -v/2$. Then $w > 0$, and there is an integer $n$ such that $nw \in \alpha$
	but $(n+1)w \notin \alpha$. (This is based on the archimedian proprty of rational numbers.)
	Put $p = -(n+2)w$. Now $p \in \beta$ because $- p - w \notin \alpha$ and $v = nw + p \in \alpha + \beta$.
	Thus, $0^* \subseteq \alpha + \beta$.
	\bigbreak \quad
	Hence, $\alpha + \beta = 0^*$. This $\beta$ will be deonted by $-\alpha$.

\end{enumerate}

\item Having proved the addition axioms defined in \ref{add} (A),
it follows the Proposition \ref{addprop} is valid in $R$.

Now we can prove one of the requirements in definition \ref{ordf}.

We need to prove that if $\beta < \gamma$ then $\alpha + \beta < \alpha + \gamma$.
If $p = r + s \in \alpha + \beta$ with $r \in \alpha$ and $s \in \beta$, then $s \in \gamma$ as well because $\beta \subset \gamma$.
Hence, $\alpha + \beta \subseteq \alpha + \gamma$.
If $\alpha + \beta = \alpha + \gamma$ then $\beta = \gamma$ by Proposition \ref{addprop}.
This is a contradiction. Hence, proved that $\alpha + \beta \subset \alpha + \gamma$.

Thus, if $\beta < \gamma$ then $\alpha + \beta < \alpha + \gamma$.

If also follows that if $\alpha > 0^*$ then $-\alpha < 0^*$.

\item Multiplication is a bit bothersome than addition, since product of negative rational numbers are positive.
So we define it only of $R^+$ for now.

If $\alpha \in R^+$ and $\beta \in R^+$ then $\alpha \beta$ is the set of all $p$ such that $p \leq rs$
for some $r \in \alpha, s \in \beta, r > 0, s > 0$.

We define $1^*$ to be the set of all $p < 1$.
Here we'll prove the multiplication axioms defined in \ref{mul} (M).

\begin{enumerate}[(M1)]
	\item To prove that if $\alpha \in R^+$ and $\beta \in R^+$ then $\alpha \beta \in R^+$.
	\bigbreak \quad
	If $\alpha > 0^*$ and $\beta > 0^*$ then there exists $p \in \alpha$ and $q \in \beta$ such that $p > 0$ and $q > 0$.
	By (III), since $0 \in \alpha$ and $0 \in \beta$.
	\bigbreak \quad
	Assume $0 \notin \alpha$ then if $x \in \alpha$ then $x < 0$ so $x \in 0^*$ so $\alpha \leq 0^*$ which is a contradiction.
	Thus, $0 \in \alpha$.
	\bigbreak \quad
	Since $pq \leq pq$ so $pq \in \alpha \beta$. Thus, $\alpha \beta$ is non empty.
	\bigbreak \quad
	If $p \notin \alpha$ and $q \notin \beta$ then $p > r$ for each $r \in \alpha$ and $q > s$ for each $s \in \beta$.
	Hence, $p > 0$ and $q > 0$ since $0 \in \alpha$ and $0 \in \beta$.
	Hence, $pq > rs$ for each $r \in \alpha$, $s \in \beta$ where $r > 0$ and $s > 0$. Thus, $pq \notin \alpha \beta$.
	Hence, $\alpha \beta \neq Q$.
	This satisfies (I).
	\bigbreak \quad
	If $p \in \alpha \beta$ and $q < p$ then there exists some $r \in \alpha$ and $s \in \beta$
	such that $p \leq rs$. Thus, $q < rs$ as well which implies $q \in \alpha \beta$. This satisfies (II).
	\bigbreak \quad
	If $p \in \alpha \beta$ then there exists some $r \in \alpha$ and $s \in \beta$ with $r > 0$ and $s > 0$
	such that $p \leq rs$. Since $r \in \alpha$ so there exists $r' \in \alpha$ such that $r' > r$.
	Since $s > 0$ and $r' > r > 0$ so $r's > rs \geq p$. Thus, $r's \in \alpha \beta$ and $r's > p$.
	This satisfies (III).
	\bigbreak \quad
	Since $0 \leq rs$ for any $r \in \alpha$, $s \in \beta$, $r > 0$ and $s > 0$ so $0 \in \alpha \beta$.
	Thus, $\alpha \beta > 0^*$ because if $x \in 0^*$ then $x < 0$ and $x \in \alpha \beta$ by (II).
	\bigbreak \quad
	Hence, $\alpha \beta \in R+$.

	\item To prove that multiplication is commutative.
	\bigbreak
	Let there is some $\alpha \in R^+, \beta \in R^+, r \in \alpha, s \in \beta, r > 0$ and $s > 0$.

	If $p \in \alpha \beta$ then $p \leq rs$ implies that $p \leq sr$ implies that $p \in \beta \alpha$.

	If $p \in \beta \alpha$ then $p \leq sr$ implies that $p \leq rs$ implies that $p \in \alpha \beta$.
	\bigbreak
	Thus, $\alpha \beta = \beta \alpha$.

	\item To prove that multiplication is associative.
	\bigbreak
	Let there is some $\alpha \in R^+, \beta \in R^+, \gamma \in R^+, r \in \alpha, s \in \beta, t \in \gamma, r > 0, s > 0$ and $t > 0$.

	If $p \in (\alpha \beta) \gamma$ then $p \leq (rs)t$ implies $p \leq r(st)$ implies $p \in \alpha (\beta \gamma)$.

	If $p \in \alpha (\beta \gamma)$ then $p \leq r(st)$ implies $p \leq (rs)t$ implies $p \in (\alpha \beta) \gamma$.
	\bigbreak
	Thus, $(\alpha \beta) \gamma = \alpha (\beta \gamma) = \alpha \beta \gamma$.

	\item To prove that $\alpha 1^* = \alpha$.
	\bigbreak \quad
	If $p \in \alpha 1^*$ then $p \leq rs$ for some $r \in \alpha$ and $s \in 1^*$ with $r > 0$ and $s > 0$.
	Since $s \in 1^*$ so $0 < s < 1$ so $0 < rs < r$.
	Since $p < rs$ so $p < r$ which imples that $p \in \alpha$ since $r \in \alpha$ by (II).
	Thus, $\alpha 1^* \subseteq \alpha$.
	\bigbreak \quad
	If $p \in \alpha$ and $\alpha > 0^*$ then by (III), there exists some $r \in \alpha$ such that $r > p$ and $r > 0$.
	[ If $p \geq 0$ then by (III), $r > p$ so $r > 0$.
	If $p < 0$ then by (III) $r > 0$ since $0 \in \alpha$ so $r > p$. ]
	\bigbreak \quad
	Now $p < r$ so $p (1/r) < 1$ by \ref{ordfprop} as $(1/r) > 0$.
	Since $p(1/r) < 1$ so $p(1/r) \in 1^*$.
	Now there exists a $q \in 1$ such that $q > p(1/r)$ and $q > 0$.
	Now $q \in 1^*$ and $r \in \alpha$ such that $q > p(1/r)$ and $r > 0$
	so $qr > p$. Thus, $p \in \alpha 1^*$.
	Thus, $\alpha \subseteq \alpha 1^*$.
	\bigbreak
	Hence, proved that $\alpha 1^* = \alpha$.

	\item To prove that there exists an inverse $(1/\alpha)$ such that $\alpha (1 / \alpha) = 1^*$.
	\bigbreak
	Fix $\alpha \in R^+$. Let $\beta^+$ be the set of all $p$ such that there exists $r > 1$ such that $(1/p)(1/r) \notin \alpha$.
	\bigbreak
	Now we need to show that $\beta = \mathbb{Q}^- \cup \beta^+ \in R^+$. Here $\mathbb{Q}^-$ is the set of all $p \leq 0$.
	\bigbreak \quad
	If $s \notin \alpha$ and $p = (1/s)(1/2)$ then $(1/p)(1/2) \notin \alpha$ so $p \in \beta^+$. Thus, $\beta$ is not empty.
	If $s \in \alpha$ then $(1/s) \notin \beta^+$
	[ Because $(1/(1/s))(1/r) = s(1/r) < s$ so $s(1/r) \in \alpha$ for $r > 1$ ].
	Thus, $\beta \neq Q$.
	This satisfies (I).
	\bigbreak \quad
	If $q \in \beta$ and $0 < s < q$ then there is an $r > 1$ such that $(1/q)(1/r) \notin \alpha$.
	Since $s < q$ then $(1/s) > (1/q)$ implies $(1/s)(1/r) > (1/q)(1/r)$. Thus, $(1/s)(1/r) \notin \alpha$.
	So $s \in \beta^+$. This, satisfies (II).
	\bigbreak \quad
	Let $p \in \beta$ such that $(1/p)(1/r) \notin \beta$.
	Put $s = (r+1)/2$ and $t = p(r/s)$. Since $r > 1$ so $1 < s < r$ and $(r/s) > 1$.
	Thus, $t > p$.

	Now $(1/t)(1/s) = (1/p)(1/r) \notin \alpha$. Thus, $t \in \beta^+$.
	This satisfies (III).

	Thus, $\beta \in R^+$.
	\bigbreak \quad
	Now if $r \in \alpha$ and $s \in \beta$ with $r > 0$ and $s > 0$,
	then $(1/s) \notin \alpha$ so $r < (1/s)$ implies $rs < 1$ so $rs \in 1^*$.
	Thus, $\alpha \beta \subseteq 1^*$.
	\bigbreak \quad
	To prove the opposite inclusion, pick $u \in 1^*$, $u > 0$.
	Put $v=[(1+u)/2]^2, w=2/(1+u)$. Then $u < v, w > 1$, and there is an integer $n$
	such that $w^n \in \alpha$ but $w^{n+1} \notin \alpha$.
	\bigbreak \quad
	Put $p = 1 / w^{n+2}$. Now $(1/p)(1/w) = w^{n+1} \notin \alpha$ so $p \in \beta$.
	And $v = 1/w^2 = w^n p \in \alpha \beta$. By (II), $u \in \alpha \beta$. Thus, $1^* \subseteq \alpha \beta$.
	\bigbreak
	Hence, $\alpha \beta = 1^*$. And $\beta$ is denoted by $1 / \alpha$


	\item[(D1)] To prove that $\alpha ( \beta + \gamma ) = \alpha \beta + \alpha \gamma $ for each $\alpha , \beta , \gamma \in R^+$.
	\bigbreak \quad
	Let $x \in \alpha(\beta + \gamma)$ then $x \leq r(s+t)$ for some $r \in \alpha$, $s \in \beta$ and $t \in \gamma$ such that $r > 0$ and $s + t > 0$.
	Now, if $x < r(s+t)$ then $x < rs + rt$ and $rs \in \alpha \beta$ and $st \in \alpha \gamma$.
	Thus, $rs + rt \in \alpha \beta + \alpha \gamma$. Now, by (II), $x \in \alpha \beta + \alpha \gamma$.
	Thus, $\alpha(\beta + \gamma) \subseteq \alpha \beta + \alpha \gamma$.
	\bigbreak \quad
	Let $x \in \alpha \beta + \alpha \gamma$ then $x = p + q$ such that $p \leq rs$ and $q \leq mn$
	for some $r \in \alpha, s \in \beta, m \in \alpha, n \in \gamma$ with $r, s, m, n > 0$.
	\bigbreak \quad
	If $r < m$ then $rs < ms \Rightarrow rs + mn < ms + mn \Rightarrow x < m(s + n)$.
	Now, $m \in \alpha, s \in \beta$ and $n \in \gamma$ so $m(s+n) \in \alpha(\beta + \gamma)$.
	So by (II), $x \in \alpha(\beta + \gamma)$.
	We can prove similarly the other two cases.
	Thus, $\alpha \beta + \alpha \gamma \subseteq \alpha ( \beta + \gamma )$.
	\bigbreak \quad
	Thus, $\alpha (\beta + \gamma) = \alpha \beta + \alpha \gamma$. Hence, proved.
\end{enumerate}

Notice that this follows that if $\alpha > 0^*$ and $\beta > 0^*$ then $\alpha \beta > 0^*$.

\pagebreak

\item We complete the definition of multiplication by definition $\alpha 0^* = 0^* \alpha = 0^*$.
and by setting

\begin{gather*}
	\alpha \beta =
	\begin{cases}
		[(-\alpha)(-\beta)] \quad \text{ if } \alpha < 0^*, \beta < 0^* \\
		-[(-\alpha)(\beta)] \quad \text{ if } \alpha < 0^*, \beta > 0^* \\
		-[(\alpha)(-\beta)] \quad \text{ if } \alpha > 0^*, \beta < 0^* \\
	\end{cases}
\end{gather*}

Having proved the axioms \ref{mul} (M) hold in $R^+$, it is now perfectly simple to prove them in $R$.
First four axioms are trivial. And the inverse of $\alpha$ is $-(1/(-\alpha))$ for $\alpha < 0^*$.

We can also prove the distributive property easily.

We have now completed the proof that $R$ is an ordered field with the least upper-bound property.

\item We associate each $r \in Q$ with $r^*$ which consists of all $p \in Q$
such that $p < r$. It is clear that each $r^*$ is a cut; that is, $r^* \in R$.
These cuts satisfy the following relations :

\begin{enumerate}[(a)]
	\item $r^* + s^* = (r+s)^*$
	\item $r^* s^* = (rs)^*$
	\item $r^* < s^*$ if any only if $r < s$
\end{enumerate}

\item We saw in last step that the replacement of the rational numbers $r$ by the conesponding ``rational cuts'' $r^* \in R$ preserves sums, products, and order.
This fact may be expressed by saying that the ordered field $Q$ is isomorphic to the ordered field $Q^*$ whose elements are the rational cuts.
Of course, $r^*$ is by no means the same as $r$, but the properties we are concerned with (arithmetic and order) are the same in the two fields.

It is this identification of $Q$ with $Q^*$ which allows us to regard $Q$ as a subfield of $R$.

\end{enumerate}


\begin{thm}
	The following statements hold :
	\begin{enumerate}[a)]
		\item If $x \in R$, $y \in R$, and $x > 0$ then there is a positive integer $n$ such that
		$ nx > y $
		\item If $x \in R$, $y \in R$, and $x < y$, then there exists a $p \in Q$ such that $x < p < y$.
	\end{enumerate}
	\begin{proof}
		\begin{enumerate}[a)]
			\item
			Let $A$ be the set of all numbers for the form $nx$ where $n$ is a positive integer.
			Assume that for all $nx \in A$, $nx \leq y$. Then $y$ is an upper-bound of $A$.
			Now by least-upper bound property, let $\alpha = \text{ sup } A$.
			Since $x > 0$, $\alpha - x < \alpha$ which means $\alpha - x$ is not an upper bound of $A$.
			So there exists $nx$ where $n$ is some positive integer such that $nx > \alpha - x$ implies that $(n+1)x > \alpha$.
			Now, $(n+1)x \in A$ but this is a contradiction since $\alpha$ is an upper-bound of $A$.

			Thus, our assumption must be wrong. Thus, there exists some $nx$ such that $nx > y$ where $n$ is some positive integer.

			\item
			Since $x < y$, we have $y - x > 0$ so by (a), $$ n(y - x) > 1 $$ for some positive integer $n$.
			Now, $1 > 0$ so $m_1 > nx$ and $m_2 > -nx$ so $$ -m_2 < nx < m_1 $$
			Thus, there is an integer such that $$ m - 1 \leq nx < m $$
			Combining these we get, $$ nx < m \leq 1 + nx < ny $$.
			Since $n > 0$ if follows that, $$ x < \frac{m}{n} < y $$
		\end{enumerate}
	\end{proof}
\end{thm}


\begin{thm}
	\label{pwrthm}
	For every real $x > 0$ and every integer $n > 0$ there is one and only one
	positive real number $y$ such that $y^n = x$.

	This number is written as $\sqrt[n]{x}$ or $x^{\frac{1}{n}}$
	\begin{proof}
		There is atmost one such $y$ is clear, since $0 < y_1 < y_2$ implies $y_1^n < y_2^n$.

		Let $E$ be set of all positive real numbers $t$ such that $t^n < x$.

		If $t = x / (1 + x)$ then $0 < t < 1$ and $t^n < t < x$. So $t \in E$ and $E$ is not empty.
		If $t > 1 + x$ then $t^n \geq t > x$, so that $t \notin E$.
		Thus, $E \neq R$ and $1 + x$ is an upper bound of $E$.

		Thus, by least-upper bound property of real numbers, there exists $y = \sup E$.

		To prove that $y^n = x$ we will show that each inequality $y^n < x$ and $y^n > x$ leads to a contradiction.

		The identity $b^n - a^n = (b - a)(b^{n-1} + b^{n-2}a + \cdots + a^{n-1} )$ yields the inequality
		$$ b^n - a^n < (b - a) n b^{n-1} $$
		when $0 < a < b$.

		Assume $y^n < x$. Choose $h$ so that $0 < h < 1$ and
		$$ h < \frac{x - y^n}{n(y+1)^{n-1}} $$
		Put $a = y$ and $b = y + h$ to get
		$$ (y + h)^n - y^n < hn(y+h)^{n-1} < hn(y+1)^{n-1} < x - y^n $$
		Thus, $(y + h)^n < x$, so $y + h \in E$.
		Since $y + h > y$, this contradicts the fact that $y$ is an upper bound of $E$.

		Now assume $y^n > x$. Choose $k$
		$$ k = \frac{y^n - x}{ny^{n-1}} $$
		Then $0 < k < y$ because $x > 0 \Rightarrow y^n - x < y^n \leq ny^n$ as $n > 0$ is an integer so $n \geq 1$.
		Now, if $t \geq y - k$ then
		$$ y^n - t^n \leq y^n - (y-k)^n < kny^{n-1} = y^n - x $$.
		Thus, $t^n > x$, and $t \neq E$. It follows that $y - k$ is an upper bound of $E$.

		But $y - k < y$, which contradicts the fact that $y$ is the least upper-bound of $E$.

		Hence, $y^n = x$, and the proof is complete.
	\end{proof}
\end{thm}

\begin{cor}
	If $a$ and $b$ are positive real numbers then $$(ab)^{1/n} = a^{1/n} b^{1/n}$$
	\begin{proof}
		Put $\alpha = a^{1/n}$ and $\beta = b^{1/n}$ then $$ ab = \alpha^n \beta^n = (\alpha \beta)^n $$
		since multiplication is commutative. The uniqueness assertion of Theorem \ref{pwrthm} shows that
		$$ (ab)^{1/n} = a^{1/n} b^{1/n} $$
	\end{proof}
\end{cor}

\pagebreak

\section{The Complex Field}

\begin{defn}
	A complex number is an ordered pair of real numbers.
	Ordered means that $(a, b)$ and $(b, a)$ are different if $a \neq b$.

	Let $x = (a, b)$ and $y = (c, d)$. We write $x = y$ if and only if $a = c$ and $b = d$.
	We defind $x + y = (a + c, b + d)$ and $xy = (ac - bd, ad + bc)$.
\end{defn}

\begin{thm}
	These definitions of addition and multiplication turn the set of all complex numbers into a field,
	with $(0, 0)$ and $(1, 0)$ in the role of $0$ and $1$.
	\begin{proof}
		The proof of axioms of addition are trivial. The additive inverse of $(a, b)$ is $(-a, -b)$.
		The proof of first four axioms of multiplication is also trivial.

		For (M5), the inverse of $x = (a, b)$ is $$ \frac{1}{x} = \left( \frac{a}{a^2 + b^2}, \frac{-b}{a^2 + b^2} \right) $$
		Now, $$ x \frac{1}{x} = (a, b) \left( \frac{a}{a^2 + b^2}, \frac{-b}{a^2 + b^2} \right)
					= \left( \frac{a^2 + b^2}{a^2 + b^2}, \frac{-ab + ab}{a^2 + b^2} \right) = (1, 0) $$

		The proof of distributive law is trivial as well.
	\end{proof}
\end{thm}

\begin{thm}
	For every real number $a$ and $b$, we have,
	$$ (a, 0) + (b, 0) = (a + b, 0) ; \quad (a, 0) (b, 0) = (ab, 0) $$
	The proof is trivial. Thus, the real field is a subfield of the complex field.
\end{thm}

\begin{thm}
	Define $i = (0, 1)$ and prove $i^2 = -1$.
	\begin{proof}
		$i^2 = (0, 1) (0, 1) = (0 - 1, 0 + 0) = (-1, 0) = -1$
	\end{proof}
\end{thm}

\begin{thm}
	If $a$ and $b$ are real numbers then $(a, b) = a + bi$
	\begin{proof}
		$a + bi = (a, 0) + (b, 0)(0, 1) = (a, 0) + (0, b) = (a, b)$
	\end{proof}
\end{thm}

\begin{defn}
	If $a$ and $b$ are real numbers and $z = a + bi$ then
	the complex number $\overline{z} = a - bi$ is called the {\it conjugate} of $z$.
	The numebrs $a$ and $b$ are the {\it real part} and the {\it imaginary part} of $z$, respectively.

	We write, $a = Re(z)$ and $b = Im(z)$.
\end{defn}

\begin{thm}
	If $z$ and $w$ are complex numbers then
	\begin{enumerate}[a)]
		\item $\overline{z+w} = \overline{z} + \overline{w}$
		\item $\overline{zw} = \overline{z} \overline{w}$
		\item $z + \overline{z} = 2Re(z)$ and $z - \overline{z} = 2 i Im(z)$
		\item $z\overline{z}$ is real and positive except when $z = 0$
	\end{enumerate}
\end{thm}

\begin{defn}
	If $z$ is a complex number then, its {\it absolute value} $|z|$ is the non-negative
	sqaure root of $z\overline{z}$ ; that is, $|z| = (z\overline{z})^{1/2}$.

	The existence and uniqueness of $|z|$ follows from Theorem \ref{pwrthm} and the fact that $z \overline{z}$ is a real number.
\end{defn}

\begin{thm}
	Let $z$ and $w$ be complex numbers. Then
	\begin{enumerate}[a)]
		\item $|z| > 0$ unless $z = 0$, $|0| = 0$
		\item $|\overline{z}| = |z|$
		\item $|zw| = |z||w|$
		\item $|Re (z)| \leq |z|$
		\item $|z + w| \leq |z| + |w|$
	\end{enumerate}
	\begin{proof}
		\begin{enumerate}[a)]
			\item Let $z = (a, b)$. If $z \neq 0$ then either $a \neq 0$ or $b \neq 0$ so $|z|^2 = a^2 + b^2 > 0$ by Proposition \ref{ordfprop} (d).
			So by definition, we take the positive square root to get, $|z| > 0$.
			If $z = 0$ then $|z| = \sqrt{0^2 + 0^2} = 0$

			\item If $z = (a, b)$ then $\overline{z} = (a, -b)$ so $|z|^2 = |\overline{z}|^2 = a^2 + b^2$.
			Now by uniqueness Theorem \ref{pwrthm}, we get $|z| = |\overline{z}|$.

			\item Let $z = (a, b)$ and $w = (c, d)$ so $zw = (ac - bd, ad + bc)$.
			Now,
			\begin{align*}
			|zw|^2 & = (ac - bd)^2 + (ad + bc)^2 \\
				& = a^2 c^2 + b^2 d^2 - 2abcd + a^2 d^2 + b^2 c^2 + 2abcd \\
				& = a^2(c^2 + d^2) + b^2(d^2 + c^2) \\
				& = (a^2 + b^2) (c^2 + d^2) = |z|^2 |w|^2 = (|z||w)|^2
			\end{align*}
			Now, by Theorem \ref{pwrthm}, $|zw| = |z||w|$.

			\item Let $z = (a, b)$. Note that $|a| = \sqrt{a^2} \leq \sqrt{a^2 + b^2}$.

			\item Note that $z \overline{w}$ is conjugate of $\overline{z} w$
			\begin{align*}
				|z + w|^2 & = (z + w)(\overline{z + w}) \\
						& = (z + w)(\overline{z} + \overline{w})
						& = z \overline{z} + z \overline{w} + w \overline{z} + w \overline{w} \\
						& = |z|^2 + 2 Re(z \overline{w}) + |w|^2 \\
						& \leq |z|^2 + 2 |z \overline{w}| + |w|^2 \\
						& = |z|^2 + 2|z||w| + |w|^2 \\
						& = (|z| + |w|)^2
			\end{align*}
			Hence, proved by taking square roots.
		\end{enumerate}
	\end{proof}
\end{thm}

We will end this section by a famous inequality called {\it the Schwarz inequality}

\begin{thm}
	If $a_1, \dots, a_n$ and $b_1, \dots, b_n$ are complex numbers, then
	$$
		\left| \sum_{j=1}^n a_j \overline{b_j} \right| ^ 2
		\leq
		\sum_{j=1}^n |a_j|^2 \sum_{j=1}^n |b_j|^2
	$$
	\begin{proof}
		Put $A = \sum |a_j|^2, B = \sum |b_j|^2, C = \sum a_j \overline{b_j}$.
		If $B = 0$ then $b_1 = \cdots = b_n = 0$ and the conclustion is trivial.
		Assume $B > 0$. By Theorem \ref{pwrthm}, we have
		\begin{align*}
			\sum |B a_j - C b_j|^2
				& = \sum (B a_j - C b_j)(B \overline{a_j} - \overline{C} \overline{b_j}) \\
				& = \sum ( B^2 |a_j|^2
					- B \overline{C} a_j \overline{b_j}
					- B C \overline{a_j} b_j
					+ |C|^2 |b_j|^2 ) \\
				& = B^2 A - B \overline{C} C - B C \overline{C} + |C|^2 |B|^2 \\
				& = B (BA - |C|^2)
		\end{align*}
		Now, $B > 0$ so $BA - |C|^2 \geq 0$ implies $|C|^2 \leq AB$.
		Hence, proved.
	\end{proof}
\end{thm}


\section{Euclidean Spaces}

\begin{defn}
	For each positive integer $k$, let $\mathbb{R}^k$ be the set of all ordered $k$-tuples
	$$ x = (x_1, x_2, \dots, x_k) $$
	where $x_1, x_2, \dots, x_k$ are called the coordinates of $x$.

The points of $\mathbb{R}^k$ are called vectors especially when $k > 0$.
If $y = (y_1, \dots, y_k)$ and $\alpha$ is a real number, put
$$ x + y = (x_1 + y_1, x_2 + y_2, \dots, x_k + y_k) $$
$$ \alpha x = (\alpha x_1, \alpha x_2, \dots, \alpha x_k) $$

so that $x + y \in \mathbb{R}^k$ and $\alpha x \in \mathbb{R}^k$.

This defines vector addition and multiplication by a real number.
These two propeties follow the commutative, associative and the distributive laws.
The proof is trivial. And this makes $\mathbb{R}^k$ {\it a vector space over the real field}.

We also define the inner product of $x$ and $y$ as
$$
x \cdot y = \sum_{i=1}^k x_i y_i
$$
and the norm as
$$
|x| = (x \cdot x)^{1/2} = \left( \sum_{i=1}^k x_i^2 \right)^2
$$

The structure now defined (the vector space $\mathbb{R}^k$ with inner product and norm)
is called the Euclidean k-space.
\end{defn}

\begin{thm}
	Suppose $x, y, z \in \mathbb{R}^k$, and $\alpha$ is real. Then
	\begin{enumerate}[a)]
		\item $|x| \geq 0$
		\item $|x| = 0$ {\it if and only if } $x = 0$
		\item $|\alpha x| = |\alpha| |x|$
		\item $|x \cdot y| \leq |x||y|$
		\item $|x+y| \leq |x| + |y|$
		\item $|x-z| \leq |x-z| + |y-z|$
	\end{enumerate}

	\begin{proof}
		(a), (b), (c) are trivial. (d) is a direct consequence of Schwarz inequality.
		From (d), we have,
		\begin{align*}
			|x+y|^2 & = (x+y) \cdot (x+y) \\
				& = |x|^2 + 2 x \cdot y + |y|^2 \\
				& \leq |x|^2 + 2|x||y| + |y|^2 \\
				& = (|x| + |y|)^2
		\end{align*}
		Now, replace $x$ and $y$ with $x-z$ and $y-z$ to get (f).
	\end{proof}

\end{thm}
