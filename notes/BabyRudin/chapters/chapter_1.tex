
%----------------------------------------------------------------------------------------
%	Chapter 1
%----------------------------------------------------------------------------------------

\chapter{The Real and Complex Number System}

\bigbreak
\section{Ordered Sets}

\begin{defn}
    Let $S$ be a set. An {\it order} on S is a relation, denoted by $<$, 
    with the following two properties : 
    \begin{enumerate}
        \item If $x \in S$ and $y \in S$ then one and only one of the statemments
            $$ x < y ; \quad x = y ; \quad y < x $$
            is true.
        \item If $x, y, z \in S$, if $x < y$ and $y < z$ then $x < z$.
    \end{enumerate}
    It is often convenient to write $y > x$ instead of $x < y$.
    The notation $x \leq y$ means either $x < y$ or  $x = y$. 
    In other words, $x \leq y$ is negation of $x > y$.
\end{defn}

\begin{defn}
    An {\it ordered set} is a set $S$ in which an order is defined.
\end{defn}
For example, the set of rational numbers $Q$ is a ordered set if $r < s$ 
is defined to mean that $s-r$ is a positive rational number.

\begin{defn}
    Suppose $S$ is an ordered set, and $E \subset S$. 
    If there exists a $\beta \in S$ such that $x \leq \beta$ for every $x \in E$, 
    we say that $E$ is {\it bounded above},
    and $\beta$ is an upper bound of $E$
\end{defn}
Lower bounds are defined in the same way (with $\geq$ in place of $\leq$).

\begin{defn}
    \label{supremum}
    Suppose $S$ is an ordered set, $E \subset S$, and $E$ is bounded above. 
    Suppose there extsts an $\alpha \in S$ with the following properties: 
    \begin{enumerate}
        \item $\alpha$ is an upper bound of E. 
        \item If $\gamma \in S$ and $\gamma < \alpha$ then $\gamma$ is not an upper bound of $E$.
    \end{enumerate}
    Then $\alpha$ is called the least upper bound of $E$ 
    [that there is at most one such $\alpha$ is clear from (2).] 
    or the supremum of $E$, and we write $$ \alpha = \text{ sup } E $$
    
    The greatest lower bound, or infimum, of a set $E$ which is bounded below is defined in the same manner: 
    The statement $$ \alpha = \text{ inf } E $$ means that $\alpha$ is a lower bound of $E$ 
    and that no $\beta \in S$ with $\beta > \alpha$ is not a lower bound of $E$.
\end{defn}

\begin{rem}
    If $\alpha \in S$ is the supremum of $E$ then $\alpha$ may or may not belong to $E$.
\end{rem}

\begin{defn}
    An ordered set $S$ is said to have the {\it least-upper-bound property} if the following is true:
    
    If $E \subset S$, $E$ is not empty and $E$ is bounded above, then $\text{ sup } E$ exists.
\end{defn}

\pagebreak

\begin{thm}
    Suppose $S$ is an ordered set with the least-upper-bound property, 
    $B \subset S$, $B$ is not empty and $B$ is bounded below.
    Let $L$ be the set of all lower bounds of $B$ then $$ \alpha = \text{ sup } L $$
    exists in $S$, and $\alpha = \text{ inf } B$.

    In particular, $\text{ inf } B$ exists in $S$.
\end{thm}

\begin{proof}
    Since $B$ is bounded below, $L$ is not empty. 
    \bigbreak 
    For all $x \in L$, $x$ is lower bound of $B$ so $x \leq y$ for all $y \in B$.
    So for all $y \in B$, $y \geq x$ for all $x \in L$.
    Thus, {\it each $y \in B$ is an upper bound of $L$}. 
    Thus, $L$ is bounded above.
    \bigbreak
    Since $L$ is bound above then by least-upper-bound property $\alpha = \text{ sup } L$ exists in $S$. 
    \bigbreak
    Assume that there exists an $x \in B$ such that $x < \alpha$.
    By Definition \ref{supremum}, $x$ is not an upper bound of $L$.
    This is a contradiction since each $x \in B$ is an upper bound of $L$ as shown earlier.
    \bigbreak
    Thus, for all $x \in B$, $x \geq \alpha$ which means that $\alpha$ is a lower bound of $B$. Thus, $\alpha \in L$.
    \bigbreak
    Since $y \leq \alpha$ for all $y \in L$. 
    This implies that if $\gamma > \alpha$ then $\gamma \notin L$.
    Thus, if $\gamma > \alpha$ then $\gamma$ is not a lower bound of $B$. 
    \bigbreak
    Now, we proved that $\alpha$ exists, it is the lower bound of $B$ and if $\gamma > \alpha$ 
    then $\gamma$ is not a lower bound of $B$.
     
    So, $\alpha$ is the {\it greatest-lower-bound} of $B$, that is, $\alpha = \text{ inf } B$.
\end{proof}