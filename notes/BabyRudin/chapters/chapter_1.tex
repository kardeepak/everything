
%----------------------------------------------------------------------------------------
%	Chapter 1
%----------------------------------------------------------------------------------------

\chapter{The Real and Complex Number System}

\bigbreak
\section{Ordered Sets}

\begin{defn}
    Let $S$ be a set. An {\it order} on S is a relation, denoted by $<$, 
    with the following two properties : 
    \begin{enumerate}
        \item If $x \in S$ and $y \in S$ then one and only one of the statements
            $$ x < y ; \quad x = y ; \quad y < x $$
            is true.
        \item If $x, y, z \in S$, if $x < y$ and $y < z$ then $x < z$.
    \end{enumerate}
    It is often convenient to write $y > x$ instead of $x < y$.
    The notation $x \leq y$ means either $x < y$ or  $x = y$. 
    In other words, $x \leq y$ is negation of $x > y$.
\end{defn}

\begin{defn}
    An {\it ordered set} is a set $S$ in which an order is defined.
\end{defn}
For example, the set of rational numbers $Q$ is a ordered set if $r < s$ 
is defined to mean that $s-r$ is a positive rational number.

\begin{defn}
    Suppose $S$ is an ordered set, and $E \subset S$. 
    If there exists a $\beta \in S$ such that $x \leq \beta$ for every $x \in E$, 
    we say that $E$ is {\it bounded above},
    and $\beta$ is an upper bound of $E$
\end{defn}
Lower bounds are defined in the same way (with $\geq$ in place of $\leq$).

\begin{defn}
    \label{supremum}
    Suppose $S$ is an ordered set, $E \subset S$, and $E$ is bounded above. 
    Suppose there extsts an $\alpha \in S$ with the following properties: 
    \begin{enumerate}
        \item $\alpha$ is an upper bound of E. 
        \item If $\gamma \in S$ and $\gamma < \alpha$ then $\gamma$ is not an upper bound of $E$.
    \end{enumerate}
    Then $\alpha$ is called the least upper bound of $E$ 
    [that there is at most one such $\alpha$ is clear from (2).] 
    or the supremum of $E$, and we write $$ \alpha = \text{ sup } E $$
    
    The greatest lower bound, or infimum, of a set $E$ which is bounded below is defined in the same manner: 
    The statement $$ \alpha = \text{ inf } E $$ means that $\alpha$ is a lower bound of $E$ 
    and that no $\beta \in S$ with $\beta > \alpha$ is not a lower bound of $E$.
\end{defn}

\begin{rem}
    If $\alpha \in S$ is the supremum of $E$ then $\alpha$ may or may not belong to $E$.
\end{rem}

\begin{defn}
    An ordered set $S$ is said to have the {\it least-upper-bound property} if the following is true:
    
    If $E \subset S$, $E$ is not empty and $E$ is bounded above, then $\text{ sup } E$ exists.
\end{defn}

\pagebreak

\begin{thm}
    Suppose $S$ is an ordered set with the least-upper-bound property, 
    $B \subset S$, $B$ is not empty and $B$ is bounded below.
    Let $L$ be the set of all lower bounds of $B$ then $$ \alpha = \text{ sup } L $$
    exists in $S$, and $\alpha = \text{ inf } B$.

    In particular, $\text{ inf } B$ exists in $S$.
\end{thm}

\begin{proof}
    Since $B$ is bounded below, $L$ is not empty. 
    \bigbreak 
    For all $x \in L$, $x$ is lower bound of $B$ so $x \leq y$ for all $y \in B$.
    So for all $y \in B$, $y \geq x$ for all $x \in L$.
    Thus, {\it each $y \in B$ is an upper bound of $L$}. 
    Thus, $L$ is bounded above.
    \bigbreak
    Since $L$ is bound above then by least-upper-bound property $\alpha = \text{ sup } L$ exists in $S$. 
    \bigbreak
    Assume that there exists an $x \in B$ such that $x < \alpha$.
    By Definition \ref{supremum}, $x$ is not an upper bound of $L$.
    This is a contradiction since each $x \in B$ is an upper bound of $L$ as shown earlier.
    \bigbreak
    Thus, for all $x \in B$, $x \geq \alpha$ which means that $\alpha$ is a lower bound of $B$. 
    Thus, $\alpha \in L$.
    \bigbreak
    Since $y \leq \alpha$ for all $y \in L$. 
    This implies that if $\gamma > \alpha$ then $\gamma \notin L$.
    Thus, if $\gamma > \alpha$ then $\gamma$ is not a lower bound of $B$. 
    \bigbreak
    Now, we proved that $\alpha$ exists, it is the lower bound of $B$ and if $\gamma > \alpha$ 
    then $\gamma$ is not a lower bound of $B$.
     
    So, $\alpha$ is the {\it greatest-lower-bound} of $B$, that is, $\alpha = \text{ inf } B$.
\end{proof}


\section{Fields}

\begin{defn}
    A {\it field} is a set $F$ with two operations called {\it addition} and {\it multiplication},
    which satisfy the following so-called ``field axioms'' (A), (M) and (D) : 
    \subsubsection{(A) Axiom of Additions}
    \label{add}
    \begin{enumerate}[(\text{A}1)]
        \item If $x \in F$ and $y \in F$ then $x + y \in F$.
        \item Addition is commutative : $x + y = y + x$ for all $x, y \in F$
        \item Addition is associative $x + (y + z) = (x + y) + z$ for all $x, y, z \in F$
        \item $F$ contains an identity element $0$ such that $0 + x = x$ for all $x \in F$
        \item For every $x \in F$, there exists a $(-x) \in F$ such that $x + (-x) = 0$
    \end{enumerate}

    \subsubsection{(M) Axiom of Multiplications}
    \label{mul}
    \begin{enumerate}[(M1)]
        \item If $x \in F$ and $y \in F$ then $xy \in F$ 
        \item Multiplication is commutative : $xy = yx$ for all $x, y \in F$
        \item Multiplication is associative : $x(yz) = (xy)z$ for all $x, y, z \in F$
        \item $F$ contains an identity element $1 \neq 0$ such that $1x = x$ for every $x \in F$
        \item For every $x \in F$ such that $x \neq 0$, there exists an element $(1/x) \in F$ 
        such that $x \cdot (1/x) = 1$
    \end{enumerate}

    \subsubsection{(D) Axiom of Distribution}
    \label{dis}
    $$ x (y + z) = xy + xz $$ holds for every $x, y,, z \in F$
\end{defn}

\begin{rem}
    \begin{enumerate}
        \item One usually writes $ x - y, \frac{x}{y}, x + y + z, xyz, x^2, x^3, 2x, 3x, \dots $ instead of
        $ x + (-y), x \cdot \left( \frac{1}{y} \right), (x+y)+z, (xy)z, xx, xxx, x+x, x+x+x, \dots $.
        \item The field axiums clearly hold in $Q$ if addition and multiplication have their customary meaning.
        Thus, $Q$ is a field.

    \end{enumerate}
    
\end{rem}

\begin{prop}
    \label{addprop}
    The axioms for addition imply the following statements : 
    \begin{enumerate}[a)]
        \item If $x + y = x + z$ then $y = z$
        \item If $x + y = x$ then $y = 0$
        \item If $x + y = 0$ then $y = (-x)$
        \item $-(-x) = x$
    \end{enumerate}
\end{prop}

\begin{proof}
    We'll use only the addition axioms of \ref{add}.
    \begin{enumerate}[a)]
        \item 
            By A4, $y = 0 + y$.         
            By A5, $y = (-x) + x + y$. 

            Given $x + y = x + z$ so $y = (-x) + x + z$. 

            By A5, $y = 0 + z$. By A4, $y = z$. Hence, proved. 
        
        \item Take $z = 0$ in (a) to prove this.
        \item Take $z = (-x)$ in (a) to prove this.
        \item Since, $(-x) + x = 0$, using (c), we get $x = -(-x)$. Hence, proved.
    \end{enumerate}
\end{proof}


\begin{prop}
    \label{mulprop}
    The axioms for multiplication imply the following statements : 
    \begin{enumerate}[a)]
        \item If $x \neq 0$ and $xy = xz$  then $y = z$
        \item If $x \neq 0$ and $xy = x$ then $y = 1$
        \item If $x \neq 0$ and $xy = 1$ then $y = (1/x)$
        \item If $x \neq 0$ then $\frac{1}{(1/x)}= x$
    \end{enumerate}
\end{prop}


\begin{proof}
    We'll use only the multiplication axioms of \ref{mul}.
    \begin{enumerate}[a)]
        \item 
            By M4, $y = 1y$.         
            By M5, $y = (1/x) x y$ since $x \neq 0$. 

            Given $xy = xz$ so $y = (1/x) x z$.

            By M5, $y = 1z$. By M4, $y = z$. Hence, proved. 
        
        \item Take $z = 1$ in (a) to prove this.
        \item Take $z = (1/x)$ in (a) to prove this.
        \item Now, $(1/x) \frac{1}{(1/x)} = 1$ if $x \neq 0$, using (c), 
        we get $x = \frac{1}{(1/x)}$. Hence, proved.
    \end{enumerate}
\end{proof}


\begin{prop}
    \label{disprop}
    The field axioms imply the following statements, for any $x, y, z \in F$.
    \begin{enumerate}[a)]
        \item $0x = 0$
        \item If $x \neq 0$ and $y \neq 0$ then $xy \neq 0$
        \item $(-x)y = -(xy) = x(-y)$
        \item $(-x)(-y) = xy$
    \end{enumerate}
\end{prop}

\begin{proof}
    \begin{enumerate}[a)]
        \item By \ref{dis} (D), $0x + 0x = (0 + 0)x$.
        
        By A4, $0x + 0x = 0x$. By \ref{addprop} (b), we get $0x = 0$. Hence, proved.

        \item Assume that there exist $x, y$ such that $x \neq 0$ and $y \neq 0$ but $xy = 0$
        Sincr $x \neq  0$ and $y \neq 0$, then by \ref{mul} (M5), 
        we know that $\left( \frac{1}{x} \right) x = 1$ 
        and $\left( \frac{1}{y} \right) y = 1$.

        Since $x \neq 0$ and $y \neq 0$. Then using (a), we get :
            $$ 1 = \left( \frac{1}{y} \right) y = \left( \frac{1}{y} \right) 1 y
                = \left( \frac{1}{y} \right) \left( \frac{1}{x} \right) x y 
                = \left( \frac{1}{y} \right) \left( \frac{1}{x} \right) 0 
                = 0
            $$

        This is a contradiction as axiom \ref{mul} (M4) says that $1 \neq 0$.
        Hence, if $x \neq 0$ and $y \neq 0$ then $xy \neq 0$.

        \item By \ref{dis} (D), we have $xy + (-x)y = (x+(-x))y$.
        By A5, we have $xy + (-x)y = 0y$.

        By (a), we have $xy + (-x)y = 0$.
        By \ref{addprop} (c), we have $(-x)y = -(xy)$. Hence, proved.

        Now, $x(-y) = (-y)x = -(yx) = -(xy)$ by first half of (c) 
        and because multiplication is commutative \ref{mul} (M2).

        \item $(-x)(-y) = -(x(-y)) = -(-(xy)) = xy$ by (c) and \ref{addprop} (d).
    \end{enumerate}
\end{proof}


\begin{defn}
    \label{ordf}
    An {\it ordered field} is a field $F$ which is also an ordered set, such that
    \begin{enumerate}[(i)]
        \item If $x, y, z \in F$ and $y < z$ then $x + y < x + z$
        \item If $x, y \in F$, $x > 0$ and $y > 0$ then $xy > 0$
    \end{enumerate}
    If $x > 0$, we call it {\it positive}; if $x < 0$, $x$ is {\it negative}.
\end{defn}

\begin{prop}
    The following statements are true in every ordered field.
    \begin{enumerate}[a)]
        \item If $x > 0$ then $ -x < 0$ and {\it vice versa}.
        \item If $x > 0$ and $y < z$ then $xy < yz$.
        \item If $x < 0$ and $y < z$ then $xy > xz$.
        \item If $x \neq 0$ then $x^2 > 0$. In particular, $1 > 0$
        \item If $0 < x < y$ then $0 < \frac{1}{y} < \frac{1}{x} $
    \end{enumerate}
\end{prop}

\begin{proof}
    \begin{enumerate}[a)]
        \item Since $0 < x$. By \ref{ordf} (i), $(-x) + 0 < (-x) + x = 0$. Thus, $(-x) < 0$.

        Now if $x < 0$ then $0 = (-x) + x < (-x) + 0$. Thus, $(-x) > 0$.
        Hence, proved.

        \item If $y < z$ then by \ref{ordf} (i) $(-y) + y < (-y) + z \Rightarrow z - y > 0$.
        
        Now by \ref{ordf} (ii), $x(z - y) > 0$ because $x > 0$ and $z - y > 0$.
        
        By distributive law, $xz - xy > 0$ implies $xz - xy + xy > 0 + xy$ by \ref{ordf} (i).
        
        Thus, $xz  > xy$. Hence, proved.

        \item If $x < 0$ then $(-x) > 0$ by (a). 
        Since $y < z$ so by (b), $(-x)y < (-x)z$. 

        By \ref{disprop} (c), we have $-(xy) < -(xz)$.
        Now by \ref{ordf} (i), $-(xy) + xy + xz < -(xz) + xy + xz$.
        
        Thus, $xz < xy \Rightarrow xy > xz$. Hence, proved.

        \item If $x \neq 0$ then $x > 0$ and $x < 0$ are the only two possibilites.
        We'll prove each case.
        
        If $x > 0$ then by \ref{ordf} (ii), $x^2 > 0$.

        If $x < 0$, by (a), $(-x) > 0$ so $(-x)(-x) > 0$. 
        By \ref{disprop} (d), $(-x)(-x) = xx = x^2$ so $x^2 > 0$.
        
        Hence, proved. Also $1^2 = 1 \cdot 1 = 1 > 0$.
        
        \item Assume $\frac{1}{x} < 0$. Given $x > 0$, by (b), 
         we have $x \cdot \left( \frac{1}{x} \right) < 0 \left( \frac{1}{x} \right)$.
    
        By \ref{disprop} (a) and \ref{mul} (M5), we have $1 < 0$. 
        This is a contradiction as we proved in (c) that $1 > 0$.

        Thus, $\frac{1}{x} > 0$. Similarly, since $y > 0$ then $\frac{1}{y} > 0$.

        Now by \ref{ordf} (ii), we have $( \frac{1}{x} ) ( \frac{1}{y} ) 
            = ( \frac{1}{y} )( \frac{1}{x} ) > 0$.
        Since $x < y$ so by (b), we get $(\frac{1}{y}) (\frac{1}{x}) x < (\frac{1}{x}) (\frac{1}{y}) y $.
        Now by \ref{mul} M5, we get $( \frac{1}{y} ) < ( \frac{1}{x} )$.
    \end{enumerate}
\end{proof}