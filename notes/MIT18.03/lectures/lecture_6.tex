
%----------------------------------------------------------------------------------------
%	Lecture 6
%----------------------------------------------------------------------------------------

\chapter{Complex Numbers and Roots of Unity} 

\bigbreak

\section{Polar Representation}

For any number $a + bi$, we can represent in polar coordinates as $r \cos \theta + i r \sin \theta$.

We can rewrite as $r(\cos \theta + i \sin \theta)$ then Euler decided to write it as $r e^{i \theta}$.

We write it as an exponential function because it satisfies the two properties of exponentials.
That is,
\begin{enumerate}
	\item $e^x \cdot e^y = e^{x+y}$
	\item $e^{at}$ is the solution to \ilds{\diff{y}{t} = ay} with $y(0) = 1$.
	\item The Taylor Series should be converging and consistent.
\end{enumerate}

Let's take the first one.

\begin{align*}
e^{i\theta_1} \cdot e^{i \theta_2} 
	& = (\cos \theta_1 + i \sin \theta_1) \cdot (\cos \theta_2 + i \sin \theta_2) \\
	& = (\cos \theta_1 \cos \theta_2 - \sin \theta_1 \sin \theta_2) \\
	& + i (\sin \theta_1 \cos \theta_2 + \cos \theta_1 \sin \theta_2) \\
	& = \cos (\theta_1 + \theta_2) + i \sin (\theta_1 + \theta_2) \\
	& = e^{i(\theta_1 + \theta_2)}
\end{align*}

Thus, this function satisfies the first property of exponential functions.

Now, we can differentiate the function $e^{it}$ because the input to the function is real.
So $e^{it}$ is a complexed value function of a real variable. 
You can think of it as two separate function, one being the real part and one being the imaginary part.
And you can differentiate them separately.

$$
\diff{e^{i t}}{t} = \diff{}{t} \cos t + i \diff{}{t} \sin t
	= - \sin t + i \cos t
	= i ( i \sin t + \cos t)
	= i e^{it}
$$

Also, $e^{i 0} = \cos 0 + i \sin 0 = 1 + 0 i = 1 $.

Thus, it satisfies the second property of the exponential function as well.
The general exponential law says that $e^{a+bi} = e^a \cdot e^{ib}$.

So any number in the complex plane can be written as $re^{i \theta}$
where $r$ is called the modulus of the complex number 
and $\theta$ is called the argument of the complex number.

Advantage of polar forms is that it is good for multiplications. 
You can just multiply the modulus and add the arguments of the complex numbers.

Thus, you also have a geomtric insight into multiplication of complex number.
That is, multiplying by a number $r e^{i \theta}$ means scaling the number by $r$ 
and rotating it counter-clockwise by an angle $\theta$.


{\bf Example : } Integrate 
$$ \int e^{-x} \cos x dx $$
Here, we can write $\cos x$ as $Re(e^{ix})$. 
Thus, $e^{-x} \cos x$ is real part of $e^{-x + ix}$.
So if we integrate $e^{(-1+i)x}$ then we can take its real part and get our original integral.
Now, 
$$
\int e^{(-1+i)x} dx = \frac{e^{(-1+i)x}}{-1+i}
$$

Multiplying by $(-1-i)$ on numerator and denomintator we get, 
$$
\int e^{(-1+i)x} dx = \frac{ e^{(-1+i)x} (-1-i) }{(-1+i)(-1-i)}
= \frac{ e^{-x}e^{ix}(-1-i) }{ 2 }
= \frac{1}{2} e^{-x} (\cos x + i \sin x)(-1-i)
$$

Taking the real part of this, we get,
$$ 
\int e^{-x} \cos x dx 
	= \frac{1}{2} e^{-x} ( - \cos x + \sin x) 
$$

In general, any integal of a function of the form $e^{ax} \cos (bx)$ or $e^{ax} \sin (bx)$ 
can be solved by moving to the complex domain.


\section{Complex Roots of Unity}

We want calculate the $n^{th}$ root of $1$, that is, $\sqrt[n]{1}$ or solve for $x$ in $x^n = 1$.
In the complex plane, there are always exactly $n$ solutions to this.

Let's take the unit circle and $n = 5$ then we can place 5 equally spaced points on the unit circle.
And the angle for each of them will be $\theta = \frac{2 \pi k}{5}$ for $k = 0, 1, 2, 3, 4$.
Now it is clear that those are the 5 roots of $x^5 = 1$ because the complex number associated with these
are $e^{\frac{2 \pi k}{5}}$. 

Now if we take it to the fifth power we'll get $e^{2 \pi k}$ for $k = 0, 1, 2, 3, 4$.
Now angle is an integer multiple of $2 \pi$ so we'll reach the point $1$. 

Thus, the $n^{th}$ roots of unity are $e^{\frac{2 \pi k}{n}}$ for $k = 0, 1, 2, ... n-1$.


