
%----------------------------------------------------------------------------------------
%	Lecture 8
%----------------------------------------------------------------------------------------

\chapter{Applications to Temperature, Mixing, RC-circuit, Decay, and Growth Models} 

\bigbreak

\section{First Order Linear Equations with Sinusoidal Inputs}

Last time we converted $y' + ky = k \cos \omega t$
to $\tilde{y}' + k\tilde{y} = k e^{i \omega t}$
where $\tilde{y} = \frac{1}{\sqrt{1 + \left( \frac{\omega}{k} \right)^2}} e^{i(\omega t - \phi)}$
and $\phi = \arctan \frac{\omega}{k}$.


Last time we did this in polar coordinates, now we do it in cartesian coordinates.
\begin{align*}
    \tilde{y} & = \frac{1}{1 + i \frac{\omega}{k}} e^{i \omega t} \\
    \tilde{y} & = \frac{1 - i \frac{\omega}{k}}{1 + \frac{\omega^2}{k^2}} (\cos \omega t + i \sin \omega t) \\
\end{align*}

Now we take the real part to get, 

$$ y(t) = \frac{1}{1 + \frac{\omega^2}{k^2}} \left( \cos \omega t + \frac{\omega}{k} \sin \omega t \right) $$

We know that $a \cos \theta + b \sin \theta = c \cos \theta - \phi$ where $\tan \phi = \frac{b}{a}$ and $c = \sqrt{a^2 + b^2}$.

Now we can convert it into the other form by setting $\tan \phi = \frac{\omega}{k}$.
We get, 

$$ y(t) = \frac{1}{\sqrt{1 + \frac{\omega^2}{k^2}}} \cos(\omega t - \phi) $$


\subsection{Proof of the Cosine Formula}

We want to prove that $a\cos \theta + b\sin \theta = c \cos (\theta - \phi)$.
Now we have two vectors $\left< a, b \right>$ and $\left< \cos \theta, \sin \theta \right>$.
The left hand side is the dot product.
The angle these two vectors make with the $X$-axis is $\phi$ and $\theta$, respectively, where $\tan \phi = \frac{a}{b}$.

Thus, the angle between these two vectors is $\theta - \phi$. So the right hand side gives us the dot product by using the cosine formula of dot product.
and $c = |\left< a, b \right>| = \sqrt{a^2 + b^2}$.

You can also prove this using the comlex product of $(a-bi)\cdot(\cos \theta + i \sin \theta)$ and taking its real part.
This can be converted to $\sqrt{a^2+b^2} e^{-i\phi} e^{i\theta} = \sqrt{a^2 + b^2} e^{i(\theta - \phi)}$.
Taking the real part, we get, $\sqrt{a^2 + b^2} \cos (\theta - \phi)$.

\pagebreak

\section{Basic Linear ODE}

There are three cases that of this equation.

\begin{enumerate}
    \item $y' + ky = kq_e(t)$ where $k > 0$
    \item $y' + ky = q(t)$ where $k > 0$
    \item $y' + k(t)y + q(y)$
\end{enumerate}

There is another case here which we did not see and that is where $k < 0$.

Let's take an example. We have a tank through which water is flowing.
The incoming water has a concentration of salt $c_e$ at the rate of $r$.
Let $x(t)$ denote the amount of salt in the tank.
There is also an outflow through which the water flows out at a rate of $r$.

So our rate of change of salt in the take is in flow rate minus the out flow rate.
$$ \diff{x}{t} = rc_e - r\frac{x}{v} \Rightarrow \diff{x}{t} + r\frac{x}{v} = rc_e $$

Here, $v$ is the volume of the tank. 
If we convert it into concentration then we'll have $x = cv$ so our equation is :
$$ v \diff{c}{t} + rc = rc_e \Rightarrow \diff{c}{t} + \frac{r}{v} c = \frac{r}{v} c_e $$ 

Here, $\frac{r}{v}$ is equivalet to $k$ in the temperature equation. 
So the solutions are the same. Here, $\frac{r}{v}$ is the fractional rate of flow.

Examples of second equation:

In electrical circuits with a resistance of $R$ and a capacitor of $C$ the equation of the charge flow is $\diff{q}{t} = \frac{q}{RC} = \frac{\epsilon(t)}{R}$ where $\epsilon(t)$ is the voltage across the battery.

In radioactive decay, where $A$ decays to $B$ at a decay rate of $k_1$ and $B$ is decaying at the rate of $k_2$.
So, $B'(t) = k_1A(t) - k_2B(t)$ becomes $B'(t) + k_2B(t) = k_1A(t)$.

{\bf Note : } If $k < 0$ then none of the interpretations apply.
You can solve the equation in the same way but it does not correspond to steady state
because the second term in the solution does not tend to zero as time passes. Instead it goes to infinity.

$k$ is typically negative in Biology and Economics.
Simplest equation for population growth is $P' = aP$ that is $P' - aP = 0$.