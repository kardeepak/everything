

%----------------------------------------------------------------------------------------
%	Lecture 4
%----------------------------------------------------------------------------------------

\chapter{Change of Variables} 

\bigbreak

\section{Introduction}

Until now we only know how to solve two kinds of differential equations :
Separation of Variables and First Order Linear Equation.

Other Equations can be solved by changing variables to convert the equations into a simple equation
that can be solved by Separation of Variables or First Order Linear Equation.

\section{Substitutions}

\subsection{Scaling}

$$ x_1 = \frac{x}{a} ; \quad y_1 = \frac{y}{b} $$

Scaling can be used to change units or make the variables dimensionless.
Another reason is to reduce or simplify constants.

{\bf Example : } Let's take the following equation : 
$$ \diff{T}{t} = k(M^4 - T^4) $$
Here, $M$ is the external temperature and $T$ is the internal temperature.

Let's take a variable $T_1 = T / M$. 
Now we can substitute $T = M T_1$ to get 
\begin{gather*}
    M \diff{T_1}{t} = k M^4(1 - T_1^4) \\
    \diff{T_1}{t} = k M^3 (1 - T_1^4) = k_1 (1 - T_1^4) 
\end{gather*} 
Here, $k_1 = k M^3$ is the new constant.

\subsection{Types of Substitution}

There are two types of substitution : Direct and Inverse.

In Direct Substitution, the new variables are a funciton of the old variables.

In Inverse Substitution, the old variables are a funciton of the new variables and old variables.

\subsection{Bernoulli Equation}

Let's take the equation $$ y' = p(x) y + q(x) y^n $$ where $n$ can be any number.
This is called the Bernouilli Equation.
This must be exactly like this, that is, there must be no other term.

Here, we divided by $y^n$ to get $$ y' y^{-n} = p(x)y^{1-n} + q(x) $$

Let's take a new variable as $ v = y^{1-n} $ and  $ v' = y' y^{-n} $.
By substituting, we get, $$ v' = p(x) v + q(x) $$

This is a First Order Linear Equation which can be solved by using an Integrating Factor.

\subsection{Homogenous ODEs}

The equation of the following form are called Homogenous Equations.
$$ y' = F \left( \frac{y}{x} \right) $$

For example, In equation, $ xy' = \sqrt{x^2 + y^2} $ we can divided by $x$ to get \ilds{y' = \sqrt{1 + \left( \frac{y}{x} \right)^2 }}.

Another way to recognize is to see that this equation is invariant under the zoom operation.
The zoom operation is the substitution, $x = ax_1$ and $y = ay_1$.

Here, the LHS is \ilds{ \diff{(ay_1)}{ax_1} = \diff{y_1}{x_1} }. 

And the RHS is  \ilds{ F \left( \frac{y}{x} \right) = F \left( \frac{ay_1}{ax_1} \right) = F \left( \frac{y_1}{x_1} \right) }

Here, we make the substitution $y = zx$ so $y' = z'x + z$, so the equation becomes,
$$ z'x + z = F(z) $$ 
Now we can separate the variables to solve this equation.

{\bf Example : } Let's say there is a boat which wants to run away from a light house that is constantly shining light on it and hide somwhere.
Going difrectly away from the light house doesn't make sense as the light house can always shine the light in the same direction to see the boat.
So it decides to go in the direction that is $45^{\circ}$ anti-clockwise to the direction of the light.

Now we want to find out the curve along which the boat travels.

Here, we know that the angle between the slope of the curve and the direction of the light is $45^{\circ}$.
Let's take the light house to be the origin.
So we can write a differential equation.

$$ \diff{y}{x} = \tan \left( \alpha + \frac{\pi}{4} \right) = \frac{\tan(\alpha) + 1}{1 - \tan(\alpha)} $$

Now we know that $tan(\alpha) = \frac{y}{x}$ because the light is always pointing towards the boat.
So, 
$$
\diff{y}{x} = \frac{\frac{y}{x} + 1}{1 - \frac{y}{x}}
$$

Now we can make the substitution, $y = zx$ and $y' = z'x + z$ so 

\begin{align*}
z'x + z & = \frac{z+1}{1-z} \\
x z' & = \frac{z+1}{1-z} - z \\
x z' & = \frac{z+1-z+z^2}{1-z} = \frac{1+z^2}{1-z} \\
\int \frac{1-z}{1+z^2} dz & = \int \frac{1}{x} dx = \ln(x) + c \\
\arctan(z) - \frac{1}{2}\ln(1+z^2) & = \ln(x) + c \\ 
\arctan(z) & = \ln( x \sqrt{1+z^2} ) + c
\end{align*}

Now we can substitute, $z = \frac{y}{x}$ to get,

$$
\arctan \left( \frac{y}{x} \right)
= \ln \left( x \sqrt{ 1 + \left( \frac{y}{x} \right)^2 } \right) + c
= \ln(\sqrt{x^2 + y^2}) + c
$$

Converting this to the polar coordinates, we get, 

$$ \theta = \ln(r) + c \Rightarrow  r = Ae^{\theta} $$


