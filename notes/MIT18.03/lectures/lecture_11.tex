
%----------------------------------------------------------------------------------------
%	Lecture 11
%----------------------------------------------------------------------------------------

\chapter{Theory of General Second-order Linear Homogeneous ODE's: Superposition, Uniqueness, Wronskians}

\bigbreak

\section{Second Order Linear Homogeneous ODE's}

The solution method was to find two solution $y_1$ and $y_2$
such that they are independent.
That is, $y_1 \neq c y_2$ for some constant $c$.
And $y_2 \neq c' y_1$ for some constant $c'$.

We have to write both to exclude the case where $y_2$ might be zero. 
So $y_1$ is not a constant multiple of $y_2$ but $y_2 = 0 y_1$.
That is why we need to say both the equations are required.

Then all the general solutions of the ODE are of the form $c_1 y_1 + c_2 y_2$.
We have two questions today.

\subsection{Why are all equations of the form $c_1 y_1 + c_2 y_2$ solutions to the ODE?}

This is answered by the Superposition Principle.
It says  that if $y_1$ and $y_2$ are the solution to a linear differential equation
then $c_1 y_1 + c_2 y_2$ is also a solution to that equation.

{\bf Proof : } We'll write the differential equation by using the differentiation operator.

$$ y'' + py' + qy = D^2y + pDy + qy = (D^2 + pD + q)y = 0 $$

Now we can say that $L = D^2 + pD + q$ which is a linear operator.
So our equation is $Ly = 0$.
The important thing is that $L$ is a linear operator.
So that means that $L(u_1  + u_2)  = Lu_1 + Lu_2$ and $L(cu_1) = cL(u_1)$  where $c$ is a constant.

Now $L$ is linear because the differentiaton is linear.
Now our ODE is $Ly = 0$ has solutions $y_1$ and $y_2$ so $Ly_1 = 0$ and $Ly_2 = 0$.
Thus, $L(c_1 y_1 + c_2 y_2) = L(c_1 y_1) + L(c_2 y_2) = c_1 L y_1 + c_2 L y_2 = 0$.

Here, the only property that is being used is that $L$ is a linear operator. 

\subsection{Why all the solutions to the ODE are in the form $c_1 y_1 + c_2 y_2$ ? }

{\bf Theorem : } The set of solutions $\{ c_1 y_1 + c_2 y_2 | c_1, c_2 \in \mathbb{R} \}$ is enough to satisfy any intial values.

{\bf Proof : } Let's say the intial condition be $y(x_0) = a$ and $y'(x_0) = b$.
Now $y' = c_1 y_1' + c_2 y_2'$ so our intial condition becomes.
\begin{gather*}
    c_1 y_1(x_0) + c_2 y_2(x_0) = a \\
    c_1 y_1'(x_0) + c_2 y_2'(x_0) = b 
\end{gather*}
From the above equations $c_1, c_2$ are the unknowns.
So this is a system of equations in two variables.

A system of two equation is solvable only if the matrix of coefficients is non-zero.
That determinant is called the Wronskian of the two functions $y_1$ and $y_2$.
And the Wronskian is a function of $x$.
$$
W(y_1, y_2) = 
\begin{vmatrix*}
    y_1 & y_2 \\
    y_1' & y_2' 
\end{vmatrix*}
    = y_1 y_2' - y_1' y_2
$$

{\bf Theorem : } If $y_1$ and $y_2$ are solution of the ODE then
$W(y_1, y_2)$ is a function which is either always zero or never zero.

{\bf Proof : } The trivial cases are easily proved.
If $y_1 = 0$ or $y_2 = 0$ then $W(y_1, y_2) = 0$ everywhere.

If $y_1 = cy_2$ then $W(y_1, y_2) = y_1 y_2' - y_1' y_2 = c y_2 y_2' - c y_2' y_2 = 0$.
Thus, the Wronskian is zero everywhere if $y_1$ and $y_2$ are not linearly independent.
This is the same if $y_2 = cy_1$.

If $y_1$ and $y_2$ are independent and non-zero, then $W(y_1, y_2)$ is non-zero everywhere.
We can write the Wronskian as 
$$ 
W(y_1, y_2) 
    = y_1 y_2' - y_1' y_2 
    = y_1^2 \diff{}{x} \left( \frac{y_2}{y_1} \right) 
$$

This form also shows that if $y_1$ and $y_2$ are non-zero and linearly dependent then the Wronskian is always zero.

\begin{align*}
    W & = y_1 y_2' - y_1' y_2 \\
    W' & = y_1' y_2' + y_1 y_2'' - y_1' y_2' - y_1'' y_2 \\
    W' & = y_1 y_2'' - y_1'' y_2
    W' + pW & = y_1 y_2'' - y_1'' y_2 + p y_1 y_2' - p y_1' y_2 \\
    W' + pW & = y_1 (y_2'' + p y_2') - y_2( y_1'' + p y_1' )
\end{align*}

Now since $y_1$ and $y_2$ are solution to the ODE $y'' + py' + qy = 0$.
So, 

$$
W' + pW = y_1( - q y_2 ) - y_2( - q y_1 ) = 0
$$ 

Thus, $W(y_1, y_2) = c e^{- \int p dx}$ if $y_1$ and $y_2$ are solution to $y'' + py  + q = 0$.
Thus, proved that Wronskian is non-zero everywhere.
This is called {\bf Abel's Theorem}.

Thus, if $y_1$ and $y_2$ are linearly independent then all initial value conditions are solvable.


\section{Finding Normalized Solution}

We want to find the normalized solutions $Y_1$ and $Y_2$ which satisfy the following initial conditions.
Initial Conditions : $Y_1(0) = 1$ and $Y_1'(0) = 0$ and $Y_2(0) = 0$ and $Y_2'(0) = 1$.

{\bf Example : } Let's say the equation is $y'' + y = 0$

So $y_1 = \cos x$ and $y_2 = \sin x$.
Here, $y_1(0) = 1$ and $y_1'(0) = 0$ and $y_2(0) = 0$ and $y_2'(0) = 1$.
So these are the normalized solution.

{\bf Example : } Let's say the equation is $y'' - y = 0$

So $y_1 = e^x$ and $y_2 = e^{-x}$.
Here, $y_1(0) = 1$ and $y_1'(0) = 1$ and $y_2(0) = 1$ and $y_2'(0) = -1$.
To find the normal solutions, let's take $Y_1 = ae^x + be^{-x}$ and $Y_2 = ce^x + de^{-x}$.

Now, $Y_1(0) = a + b = 1 $ and $Y_1'(0) = a - b = 0$.
Solving these, we'll get, $a = b = \frac{1}{2}$.

Also, $Y_2(0) = c + d = 0$ and $Y_2'(0) = c - d = 1$.
Solving these, we'll get, $c = \frac{1}{2}$ and $d = -\frac{1}{2}$.

Thus, \ilds{ Y_1 = \frac{e^x + e^{-x}}{2} } and \ilds{ Y_2 = \frac{e^x - e^{-x}}{2} }.
These are called the hyperbolic cosine and sin, respectively.

That is, \ilds{ Y_1 = \cosh x = \frac{e^x + e^{-x}}{2} } and \ilds{ Y_2 = \sinh x = \frac{e^x - e^{-x}}{2} }.


\subsection{Uses of Normalized Solution}

If $Y_1$ and $Y_2$ are normalized solution to the ODE then the solution 
to the initial value problem $y(0) = a$ and $y'(0) = b$ is $a Y_1 + b Y_2$.


\section{Existence and Uniqueness Theorem}

\begin{mdframed}
\begin{center}
{\bf  Existence and Uniqueness Theorem for Second Order Linear Homogeneous Equation : }
For the equation $y'' + py' + qy = 0$, where $p$ and $q$ are continuous for all $x$, 
there is one and only one solution such that $y(0) = A$ and $y'(0) = B$ for any $A, B$.
\end{center}
\end{mdframed}


Thus, all the solutions to the ODE are in the set $\{ c_1 Y_1 + c_2 Y_2 \}$.
To show this, we'll show that any solution to the ODE is in the above set.

{\bf Proof : } Given a solution $u(x)$ to the ODE we can find $u(0) = u_0$ and $u'(0) = u'_0$.
Then $u_0 Y_1 + u'_0 Y_2$ has the same intial conditions as $u(x)$
and it also satisfies the ODE. But, by the Uniqueness and Existence Theorem, 
there can be one and only one such solution, which means that both of these solutions must be the same.
