
%----------------------------------------------------------------------------------------
%	Lecture 12
%----------------------------------------------------------------------------------------

\chapter{Second Order Linear Non-homogenous ODEs}

\bigbreak

\section{Standard Form}

$$ y'' + p(x) y' + q(x) y = f(x) $$

In the most applications, $x$ is time.
$f(x)$ is called many things like input, signal, driving term or forcing term.

The solution is called the response to the input.
We cannot solve this without knowning the homogenous solutions.

So, $y'' + p(x)y' + q(x)y = 0$ is called associated homogenous equation. 
It is also called the reduced equation.
Its solution is $y_c = c_1 y_1 + c_2 y_2$ 
where $y_1$ and $y_2$ are linearly independent solutions to the reduced equation.
It is also called the Complimentary Solutions.


{\bf Example 1. } 
$mx'' + bx' + kx = f(t)$ is the spring-mass-dashpot system where $f(t)$ is the external force on the mass.
This is the forced system where $f(t) \neq 0$.

{\bf Example 2. } 
Another example is an electric circuit with a resistance of $R$,
a capcitance of $C$ and an inductance of $L$. So the equation is 
$$ Li' + Ri + \frac{q}{C} = \epsilon(t) $$
Here, $\epsilon{t}$ is the voltage accross the battery.
You can differentiate this to get everything in terms of $i$ as $i = q'$.
So, our equation becomes 
$$ Li'' + Ri' + \frac{i}{C} = \diff{}{t} \epsilon(t) $$

\subsection{Solutions to Second Order Non-Homogenous ODEs}

We'll denote the differential equation as a linear operator $L$.

\begin{mdframed}
\begin{center}
    {\bf Theorem. } Given a differential equation of the form $Ly = f(x)$, 
    then the solution is of the form $y_p + y_c$ 
    where $y_c$ is the solution to the associated homogenous equation
    and $y_p$ is a particular solution to $Ly = f(x)$.
    Here, the particular solution is any one solution to the equation $Ly = f(x)$.
\end{center}
\end{mdframed}

{\bf Proof. }
We have to prove two things.

\begin{enumerate}
    \item All $y = y_p + c_1 y_1 + c_2 y_2$ are the solution to the differential equation.
    This can be proved by just plugging it in the equation. 
    The solution to the homogenous equation will vanish and only $y_p$ will be left which is by definition a solution.
    $$ Ly = L(y_p + c_1 y_1 + c_2 y_2) = L(y_p) + c_1 L(y_1) + c_2 L(y_2) = f(x) + 0 + 0 = f(x)$$ 
    \item Any solution to the equation $Ly = f(x)$ is of the form $y_p + c_1 y_1 + c_2 y_2$.
\end{enumerate}

{\bf Proof : } Let $u(x)$ be the solution to $Ly = f(x)$.
Now $y_p$ is also a solution to this equation. 
So, $L(u) = f(x)$ and  $L(y_p) = f(x)$ implies that $L(u) - L(y_p) = 0$.
Since $L$ is a linear operator so $L(u - y_p) = 0$.
Thus, we can conclude that $u - y_p$ is the solution to the associated homogenous equation $Ly = 0$.
That is, $u - y_p = c_1 y_1 + c_2 y_2$ implies that any solution $u(x)$  is of the form $y_p + c_1 y_1 + c_2 y_2$.


\subsection{Relation to First Order Linear ODEs}

Let's say we have the equation $y' + ky = q(t)$ where $k$ is constant.
Here, the solution was
$$ y = e^{-kt} \int q(t) e^{kt} dt + ce^{-kt} $$

Here, we can see that the second term is the solution to the associated homogenous equation $y' + ky = 0$
and the first term is a particular solution to the equation $y' + ky = q(t)$.

For constant coefficients and $k > 0$, the particular solution was called the steady state equation 
and the complimentary solution was called the transient solution.

Now let's take the second order differential equations with constant coefficients.
$$ y'' + Ay' + By = f(t) $$

The solutions are of the form $y_p + c_1 y_1 + c_2 y_2$.
Here the complimentary solution contain information about the initial values.
So for it to be divided into steady state solution and transient solution, 
we want $y_1$ and $y_2$ to die off as time goes to infinity.
If this happens, then the particular solution $y_p$ is called the steady state solution.
And the ODE is said to be stable.

\subsubsection{Conditions for Steady State Solutions}

We'll make a case by case analysis based on roots of the characteritics equation.

\begin{case}
Roots are real and different : $r_1$ and $r_2$.

Here, $y_1 = e^{r_1 t}$ and $y_2 = e^{r_2 t}$ will go to zero if $r_1 < 0$ and $r_2 < 0$.
\end{case}

\begin{case}
Roots are real and same : $r_1 = r_2$.

Here, the solution is $y_c = e^{r_1 t}(c_1 + c_2 t)$ will go to zero if $r_1 < 0$.
\end{case}

\begin{case}
Roots are complex conjugates : $a \pm bi$.

Here, the solution is $y_c = e^{at}(c_1 \cos bt + c_2 \sin bt)$ will go to zero if $a < 0$. 
\end{case}

So the ODE is stable if the real part of all the roots of the characteritics equation is negative.