
%----------------------------------------------------------------------------------------
%	Lecture 3
%----------------------------------------------------------------------------------------

\chapter{First Order Linear Equation } 

\bigbreak

\section{Introduction}

$$ a(x)y' + b(x)y = c(x) $$

This is the first version of the First Order Linear Equation.
This is called linear because $y'$ and $y$ are linear in this equation.
If $c(x) = 0$ then it is called homogenous equation.

The standard linear form of this equation is :
$$ y' + p(x) y = q(x) $$

This equation can always be solved. 
And this also comes up in some important models.

\subsection{Models}
\begin{enumerate}
    \item Temperature / Diffusion Models
    \item Mixing Models
    \item Radioactive Decays
    \item Rocket Equation
    \item Banking Models
\end{enumerate}

\subsection{Condution Model}

Imagine a tank of liquid water. And there is a suspend object in the water.
Let's say its walls are partly insulated.

Let's say the temperature of the object is $T$ and the temperature of the water is $T_e$.
Newton's law of cooling says,
$$ \diff{T}{t} = K(T_e - T) $$

Here, $K$ is called the conductivity of the medium through which heat is transfered. 
And $K$ is a positive number.
From this equation, we can see that $T_e > T$ then the rate of change of temperature is positive.
If $T_e < T$ then the rate of change of temperature is negative.

\subsection{Concentration Model}

Here, the temperature is replaced by the concentration of some chemical, let's say salt.
The walls are said to be semi-permeable membrane through which salt can travel.
So the external concentration is $c_e$ and the internal concentration is $c$.
The Diffusion Equation says, 
$$ \diff{c}{t} = k(c_e - c) $$

Here, $k > 0$ for the same reason as before.

\subsection{Solving Conduction / Concentration Model}

Writing the tem=perature equation in the standard linear form we'll get, 
$$ \diff{T}{t} + kT = kT_e $$
Here, $p(x) = k$ and $q(x) = kT_e $.
We can look at it as a general linear equation by allowing $k$ and $T_e$ to vary with time.

Let's take the standard equation : $ y' + p(x) y = q(x) $

We will solve this equation by finding an Integrating Factor : $u(x)$.
We will multiply this equation by $u(x)$ and we want to choose $u(x)$ such that the left hand side is the differentiation of $(uy)$.

Multiplying by $u$ gives us : $uy' + puy = qu$

Now $(uy)' = uy' + u'y$ so we can find $u$ by solving $u' = pu$.
Now we can use separation of variables to solve it to get $ln(u) = \int p dx \Rightarrow u = e^{\int p dx}$.

Here, we need only one possible $u$. 
Here, $u = e^{\int p dx}$.
So now by substituting, we get,

$$
    uy' + puy = qu \Rightarrow
    (uy)' = qu \Rightarrow
    uy = \int qu dx \Rightarrow
    y = \frac{\int qu dx}{u}
$$

\subsubsection{Steps to Solve}

\begin{enumerate}
    \item Write the equation in the standard form $y' + p(x)y = q(x)$
    \item Find the Integrating Factor : $e^{\int p(x) dx}$
    \item Simplify the Integrating Factor and multiply on the both sides of the equation.
    \item Integrate
\end{enumerate}

{\bf Example : } Solve $xy' - y = x^3$.

Writing in standard form, we get, 
$$ y' - \frac{1}{x} y = x^2 $$

Here, $p(x) = - \frac{1}{x}$ and $q(x) = x^2$.

Integrating Factor is 
$$ \int p(x) dx = \int - \frac{1}{x} dx = - \ln(x) \Rightarrow u = e^{\int p(x) dx} = e^{-\ln(x)} = \frac{1}{x} $$ 

Multiplying by the Integrating Factor on both sides of the equations we get,
\begin{align*}
    \frac{1}{x}y' - \frac{1}{x^2}y = x \\
    \left( \frac{1}{x} y \right)' = x \\
    \frac{1}{x} y = \frac{x^2}{2} + c \\
    y = \frac{x^3}{2} + cx 
\end{align*}


{\bf Exmaple : } Solve $ (1 + \cos x) y' - (\sin x) y = 2x $

Standard Form : $$ y' - \frac{\sin x}{1 + \cos x} y = \frac{2x}{1 + \cos x} $$

Here, \ilds{ p(x) = - \frac{\sin x}{1 + \cos x} } and \ilds{ q(x) = \frac{2x}{1 + \cos x} }

Integral of $p(x)$ is $$ \int \frac{ - \sin x }{1 + \cos x} dx = \ln( 1 + \cos x ) $$

So the Integrating Factor is $ e^{\int p(x) dx} = 1 + \cos x $

Multiplying by the Integrating Factor, we get 

\begin{align*}
    (1+ \cos x)y' - ( \sin x )y & = 2x \\
    ((1 + \cos x) y)' & = 2x \\
    (1 + \cos x) y & = x^2 + c \\
    y & = \frac{x^2 + c}{1 + \cos x}
\end{align*}

\subsection{Linear Equation with Constant Coefficeints}

Let's solve the equation $ y' + py = q(x) $ where $p$ is constant.
Here, our Integrating Factor will be $ e^{\int p dx} = e^{px} $.
Multiplying both sides by Integrating Factor, we get 

\begin{gather*}
e^{px}y' + e^{px} p y = q(x) e^{px} \\
(e^{px} y)' = q(x) e^{px} \\
e^{px} y = \int q(x) e^{px} dx + c \\
y = e^{-px} \int q(x) e^{px} dx + c e^{-px} 
\end{gather*}

Now for the heat equation we have $p = k$ and  $q(t) = kT_e(t)$ so 
$$
T(t) = e^{-kt} \int_0^t k T_e(t) e^{kt} dt + c e^{-kt} 
$$

Here we take a definite integral because we want an unambiguous solution.
Now to find $c$ we can solve $T(0) = T_0$ and we get $c = T_0$.
Thus, the final equation becomes,

$$
T(t) = \underbrace{ e^{-kt} \int_0^t k T_e(t) e^{kt} dt }_{\text{Steady State Function}} + \underbrace{T_0 e^{-kt}}_{\text{Transient Function}}
$$

Since $k > 0$ so as $t \to \infty$ the last terms tends to zero.
Therefore, the last term is said to be transient. And the integral term is called the Steady State Function.
Because in the long run the transient term vanishes.

