
%----------------------------------------------------------------------------------------
%	Lecture 5
%----------------------------------------------------------------------------------------

\chapter{Second Derivative Test}

\bigbreak
\section{An example}

Let $w = ax^2 + bxy + cy^2$. It has a critical point at the origin.
Let's complete the square. For this, we will assume that $a \neq 0$.
\begin{align*}
    w   & = a \left( x^2 + \frac{b}{a}xy \right)  + cy^2 \\
        & = a \left( x + \frac{b}{2a}y \right)^2  + \left( c - \frac{b^2}{4a} \right) y^2 \\
        & = \frac{1}{4a} \left[ (2ax + by)^2 + (4ac - b^2)y^2 \right]
\end{align*}

Basically, we've written this as the sum or difference of two squares. There are 3 cases:

\begin{enumerate}
    \item If $4ac - b^2 < 0$ then we have a difference of two squares, which means there will be a saddle point.
    \item If $4ac - b^2 = 0$ then we have a single square term, which is the same as $w = x^2$ so we have a parabola extended across 3D space.
    This function doesn't depend at all on $y$. Basically, it is a valley whose bottom is completely flat.
    It is called {\bf degenerate critical point} because there is a direction in which nothing happens and there are infinitely many critical points in that direction.
    \item If $4ac - b^2 > 0$ then we have a sum of two squares which means we'll have a maxima if $a > 0$ or minima if $a < 0$.
\end{enumerate}

{\bf Degenerate critical points } in our special function represent a valley but for a general function it can be anything and the behaviour will depend on higher order derivatives.

\section{General Test}

In genral look at the second derivative, \ilds{\pxx{f}{x} = f_{xx}}.
We also have \ilds{\pxy{f}{x}{y} = f_{xy}} and \ilds{\pxy{f}{y}{x} = f_{yx}} are the same, that is, $f_{xy} = f_{yx}$.

Say that you have a critical point $(x_0, y_0)$ then let $A = f_{xx}, B = f_{xy}$ and $C = f_{yy}$.
Now we look at the quantity $AC - B^2$.

\begin{enumerate}
    \item If $AC - B^2 > 0$ and $A > 0$ then it is a local minima.
    \item If $AC - B^2 > 0$ and $A < 0$ then it is a local maxima.
    \item If $AC - B^2 < 0$ then it is a saddle point.
    \item If $AC - B^2 < 0$ then the test is inconclusive.
\end{enumerate}

\subsection{Relating the example with the general test}

In our example, we had $w = ax^2 + bxy + cy^2$. Now $w_x = 2ax + by$ and $w_{y} = bx + 2cy$. \\
And, $f_{xx} = 2a$, $f_{yy} = 2c$ and $f_{xy} = b$. This tells us that $A = 2a$, $B = b$ and $C = 2c$. \\
So, $AC - B^2 = 4ac - b^2$.


\section{Quadratic Approximation}

Accroding to quadratic approximation tells us that, 
$$
\Delta f \approx f_x \Delta x + f_y \Delta y + \frac{1}{2} f_{xx} (\Delta x)^2 + f_{xy} \Delta x \Delta y + \frac{1}{2} f_{yy} (\Delta y)^2 
$$
If we replace the function with this approximation then the critical points remain the same and the type of function remains the same.
So then you can deduce the second derivative test using the same method that we used in the first example.

{\bf Example : } \ilds{ f = x + y + \frac{1}{xy} }

We get, \ilds{ f_x = 1 - \frac{1}{x^2 y} } and  \ilds{ f_y = 1 - \frac{1}{x y^2} }.
Solving these we get, $x = 1$ and $y = 1$. This is the only critical point.

Next, \ilds{ f_{xx} = \frac{2}{x^3 y} } so $A = 2$. 
And, \ilds{ f_{xy} = \frac{1}{x^2 y^2} } so $B = 1$.
And, \ilds{ f_{yy} = \frac{2}{x y^3} } so $C = 2$. \\
So, $AC - B^2 = 3 > 0$ and $A = 2 > 0$ so we have a local minima.

We still have to check the boundaries and points of discontinuity. 
So $x \to \infty$ or $y \to \infty$ means $f \to \infty$.
So $x \to 0$ or $y \to 0$ means $f \to \pm \infty$. Thus, we have only one minima.
