
%----------------------------------------------------------------------------------------
%	Lecture 23
%----------------------------------------------------------------------------------------

\chapter{Vector Fields in Space, Surface Integrals and Flux} 

\bigbreak

\section{Vector Fields}

A vector field in space is ${\bf F} = \left< P, Q, R \right>$ where $P$, $Q$, $R$ are each function of $x, y, z$.

Gradien field of a function $f(x, y, z)$ is $\nabla f = \left< f_x, f_y, f_z \right>$.

\section{Flux and Surface Integrals}

In two dimenions, \ilds{Flux = \oint_C {\bf F} \cdot \hat{n} ds } is a line integral.
In three dimenions, flux will be measured through a surface so it will be a surface integral.

Let's say we have a vector field ${\bf F}$ and a surface $S$ in space.
We want to find out the normal component of the vector field.
In 3D we have to choose the orientation of the surface, that is choosing between the two normal vectors.

For closed surface, you should choose the normal vector pointing outwards. 
But there is not set conventions for this. Thus, 

$$ Flux = \iint_S {\bf F} \cdot \hat{n} dS $$

Here we're using $dS$ for the area element. 
Sometimes you may also see $d{\bf S}$ which is the same as $\hat{n} dS$.
This is because computing vector $d{\bf S}$ is easier than computing $\hat{n}$ and $dS$ separately.

{\bf Example 1.} Find flux of  ${\bf F} = \left< x, y, z \right>$ through the sphere of radius $a$ at the origin.
The vector field is pointing radially outwards. 
The normal vector will also be pointing radially outwards if we choose the outwards normal.
So $\hat{n} = \left< x, y, z \right> / a$ as the length of the vector is equal to the radius at the surface of the sphere.
$$
Flux = \iint_S {\bf F} \cdot \hat{n} dS = \iint_S \frac{x^2+y^2+z^2}{a} dS 
$$
But on the surface of the sphere : $x^2 + y^2 + z^2 = a^2$. Thus,
$$
Flux = \iint_S a dS = 4 \pi a^3
$$

Here, note that we are only considering things on the surface of the integral. 
We are not considering anything inside or outside.


{\bf Example 2.} Let the ${\bf F} = z \hat{k}$ and the surface be the sphere of radius $a$ centered at origin.

Now $\hat{n} = \left<x, y, z \right> / a$, so ${\bf F} \cdot \hat{n} = z^2 / a$. So, 
$$
Flux = \iint_S \frac{z^2}{a} dS
$$

Now we can use the spherical coordinates to parametrize this surface.
Last time we showed that the area element on a sphere is $r^2 \sin \phi d\phi d\theta$.
And $z = r \cos \phi$. But here $r = a$, so
$$
Flux = \iint_S \frac{z^2}{a} dS = \iint_S \frac{a^2 \cos^2 \phi}{a} a^2 \sin \phi d\phi d\theta
$$

Now, the limits of $\phi$ will be $0$ to $\pi$ and limits of $\theta$ will be $0$ to $2 \pi$.
\begin{align*}
Flux & = \int_0^{2\pi} \int_0^{\pi} a^3 \cos^2 \phi \sin \phi d\phi d\theta \\
    & = a^3 \int_0^{2\pi} \left[ \frac{-\cos^3 \phi}{3} \right]_0^{\pi} d\theta \\
    & = a^3 \int_0^{2\pi} \frac{-1}{3} ((-1)^3 - 1^3) d\theta \\
    & = a^3 \int_0^{2\pi} \frac{2}{3} d\theta \\
Flux & = \frac{2a^3}{3} 2 \pi = \frac{4 \pi a^3}{3} 
\end{align*}


{\bf Conclusion : } Either use geometry or parametrize the surface.
