
%----------------------------------------------------------------------------------------
%	Lecture 6
%----------------------------------------------------------------------------------------

\chapter{Differentials and Chain Rule}

\bigbreak
\section{Total Differential}

In single variable calculus, we learned that if $y = f(x)$ then $dy = f'(x)dx$. This is called implicit differentiation.
This is called total differential because it contains all the values that the function depends on.
Let $f(x, y, z)$ be a function.

\begin{align*}
    df  & = f_x dx + f_y dy + f_z dz \\
        & = \px{f}{x} dx + \px{f}{y} dy + \px{f}{z} dz
\end{align*}

{\bf Important:} $df$ is not $\Delta f$. $\Delta f$ is a number.
A differential can only be represented in the terms of other differentials.
It encodes how changes $x, y, z$ affect $f$. It can be used to approximate $f$.

We can divide everything that all the variables depend on. 
Let's say $x, y, z$ depend on some parameter $t$. 
Then the differential will tells use the rate of change of $f$ with respect to $t$ as you plug in the values of $x, y, z$.
$$
\diff{f}{t} = f_x \diff{x}{t} + f_y \diff{y}{t} + f_z \diff{z}{t}
$$

This is known as {\bf the chain rule}. It is one instance of the chain rule.

{\bf Proof : } By our approximation formula, we have
\begin{align*}
\Delta f & \approx f_x \Delta x + f_y \Delta y + f_z \Delta z \\
\frac{\Delta f}{\Delta t} & \approx  \frac{ f_x \Delta x + f_y \Delta y + f_z \Delta z }{ \Delta t } \\
\text{Taking the limit} \quad \Delta t \to 0 : \diff{f}{t} & = f_x \diff{x}{t} + f_y \diff{y}{t} + f_z \diff{z}{t}
\end{align*}


\section{Chain Rule}

Let $w = f(x, y)$ and $x = x(u, v)$ and $y = y(u, v)$.
Now we want to find \ilds{\px{w}{u}} and \ilds{\px{w}{v}} in terms of $w_x, w_y, x_u, x_v, y_u$ and $y_v$.
We can write :
\begin{align*}
dw & = f_x dx + f_y dy \\
dx & = x_u du + x_v dv \\
dy & = y_u du + y_v dv \\
\text{So, } dw & = f_x x_u du + f_x x_v dv + f_y y_u du + f_y y_v dv \\
dw & = (f_x x_u + f_y y_u) du + (f_x x_v + f_y y_v) dv
\end{align*}

If we plugin the functions $x(u, v)$ and $y(u, v)$ in $w = f(x, y)$.
Then we'll get $w = g(u, v)$. So, 
$$ dw = \px{w}{u} du + \px{w}{v} dv $$
By comparision, we get
$$
\px{w}{u} = (f_x x_u + f_y y_u) ; \quad 
\px{w}{v} = (f_x x_v + f_y y_v)
$$

We can extend this to more that two variables as well.
