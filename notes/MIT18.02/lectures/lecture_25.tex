
%----------------------------------------------------------------------------------------
%	Lecture 25
%----------------------------------------------------------------------------------------

\chapter{Divergence Theorem} 

\bigbreak

\begin{mdframed}
\begin{center}
If a closed surface $S$ encloses a space $D$ with its normal vector defined to be pointing outwards,
and a vector field ${\bf F} = \left< P, Q, R \right>$ is defined and differentiable everywhere in $D$ then
$$
\oiint {\bf F} \cdot \hat{n} dS = \iiint_D div({\bf F}) dx dy
$$
where $div({\bf F}) = P_x + Q_y + R_z$ is called divergence of ${\bf F}$.
\end{center}
\end{mdframed}

It is also known as "Gauss-Green's Theorem". It is a 3D analog of Green's Theorem.


\subsection*{Del Notation}

$\nabla$ is called `Del'. It is defined as \ilds{ \nabla = \left< \px{}{x}, \px{}{y}, \px{}{z} \right> }.
\begin{enumerate}
    \item Gradient : \ilds{ \nabla f = \left< \px{f}{x}, \px{f}{y}, \px{f}{z} \right> }
    \item Divergence : \ilds{ \nabla \cdot {\bf F} = \left< \px{}{x}, \px{}{y}, \px{}{z} \right> \cdot \left< P, Q, R \right> = \px{P}{x} + \px{Q}{y} + \px{R}{z} }
    \item Curl : \ilds{ 
        \nabla \times {\bf F} 
        =
        \begin{vmatrix*}
            \hat{i} & \hat{j} & \hat{k} \\
            \px{}{x} & \px{}{y} & \px{}{z} \\
            P & Q & R
        \end{vmatrix*}
        = \ijk  { \left( \px{R}{y} - \px{Q}{z} \right) }
                { \left( \px{P}{z} - \px{R}{x} \right) }
                { \left( \px{Q}{x} - \px{P}{y} \right) }
    }
\end{enumerate}

\section{Physical Interpretation}

$div({\bf F})$ is the source rate, that is, amount of flux generated per unit time.

In an incompressible flow of liquid, that is, the denisty is constant. 
Let's say we are given its velocity vector field as ${\bf F}$ then the divergence of ${\bf F}$
amount of fluid generated per unit time at a point.

\section{Proof of Divergence Theorem}

\subsection{Dividing the flux into its components}

We can divided the surface integral in the following way :
$$
\oiint_S \left< P, Q, R \right> \cdot \hat{n} dS
    = \oiint_S \left< P, 0, 0 \right> \cdot \hat{n} dS
    + \oiint_S \left< 0, Q, 0 \right> \cdot \hat{n} dS
    + \oiint_S \left< 0, 0, R \right> \cdot \hat{n} dS
$$

We will prove that : 
$$ \oiint_S \left< P, 0, 0 \right> \cdot \hat{n} dS = \iiint_D P_x dV $$
$$ \oiint_S \left< 0, Q, 0 \right> \cdot \hat{n} dS = \iiint_D Q_y dV $$
$$ \oiint_S \left< 0, 0, R \right> \cdot \hat{n} dS = \iiint_D R_z dV $$

\subsection{Simplifying The Triple Integral of Divergence}

Let's take the third equation and start with the right hand side.
Let's take a surface $S$ that encloses a vertically simple interval $D$, that is,
the enclosed region has a bottom and top sides given by the equations $z_{bottom} = z_1(x, y)$ and $z_{top} = z_2(x, y)$ at any point $(x, y)$.
Let's say the region projected by the region on the $XY$-plane is $U$. 
So, our RHS becaomes,
\begin{align*}
    \iiint_D R_z dV & = \iint_U \int_{z_1(x, y)}^{z_2(x, y)} R_z dz dA \\
        & = \iint_U (R(x, y, z_2(x, y)) - R(x, y, z_1(x, y)) dx dy \\
\end{align*}

\subsection{Simplifying the Flux Integral}

Now let's compute the flux on the LHS.
Since we are in a vertically simple region, so our surface consists of a top, a bottom and sides.
So we can divide our surface into these.

$$
\iint_{S} \left< 0, 0, R \right> \cdot \hat{n} dS 
    = \iint_{S_{top}} \left< 0, 0, R \right> \cdot \hat{n} dS 
    + \iint_{S_{bottom}} \left< 0, 0, R \right> \cdot \hat{n} dS 
    + \iint_{S_{sides}} \left< 0, 0, R \right> \cdot \hat{n} dS 
$$

Let's start with the top. Here we have a formula a formula for $\hat{n}dS$ for a function in the form of $z = z_2(x, y)$.
Thus, \ilds{ \hat{n}dS = \left<-\px{z_2}{x}, -\px{z_2}{y}, 1\right> dx dy }.

Here we chose the positive branch because we want the normal vector to be pointing up.

Since we are on the surface $z = z_2(x, y)$ so we can substitute that for $z$.
And the region over which we have to integrate is the projection of $S_{top}$ on the $XY$-plane which is same as $U$.
Thus, our flux is,
$$
\iint_{S_{top}} \left< 0, 0, R \right> \cdot \hat{n} dS 
    = \iint_{S_{top}} R(x, y, z) dx dy
    = \iint_U R(x, y, z_2(x, y)) dx dy
$$


Now for the bottom we can use the same formula for \ilds{ \hat{n}dS = \left< \px{z_1}{x}, \px{z_1}{y}, -1 \right> dx dy }.
But we now take the negative branch as the normal vector must be pointing outwards which is towards the bottom.
We can use the substitution $z = z_1(x, y)$ and convert it into a double integral over the region $U$ in the $XY$-plane in the same way we did before.
So,
$$
\iint_{S_{bottom}} \left< 0, 0, R \right> \cdot \hat{n} dS 
    = \iint_{S_{top}} - R(x, y, z) dx dy
    = - \iint_U R(x, y, z_1(x, y)) dx dy
$$

Now for the sides, the normal vector will be pointing outwards. 
Since the sides are vertical so their normal vector will be parallel to the $XY$-plane.
So it will be of the form $\left< a, b, 0 \right>$. 
Thus, our dot product will be zero and so will the integral.

Adding all these integrals we get the total flux as, 
$$
\oiint_S \left< 0, 0, R \right> \cdot \hat{n} dS = \iint_U R(x, y, z_2(x, y)) - R(x, y, z_1(x, y)) dx dy 
$$

\subsection{Dividing Any Region Into Vertically Simple Region}

For any region you can divide it into vertically simple region.
Notice that the flux along the sides will always be zero in this case. 
So adding vertical sufraces does not make a difference on the flux.

\subsection{Proving the same for $x$ and $y$}

You can prove the same for $x$ and $y$ by taking a simple region which is simple for $x$ and $y$ respectively.
And the vector fields $\left< P, 0, 0 \right>$ and $\left< 0, Q, 0 \right>$.

\subsection{Conclusion}

After proving all three equations we can add these to get the final result.
$$ \oiint_S \left< P, Q, R \right> \cdot \hat{n} dS = \iiint_D P_x + Q_y + R_z dV $$

{\bf Note : } Earlier we proved Green's Theorem by dividing the region into infinitesimal rectangles and approximating flux.
But we can prove it using vertically simple regions as well.

\pagebreak

\section{Applications : Diffusion Equation / Heat Equation}

A diffusion equation governs the diffusion of particles in an unmoving air.
The equation is 
$$ \px{u}{t} = k \nabla^2 u = k \nabla \cdot (\nabla u) = k \left( \pxx{u}{x} + \pxx{u}{y} + \pxx{u}{z}  \right) $$

Here $u$ is the concentration at any point. The operator $\nabla^2$ is called the Laplacian.
This equation is also known as the Heat equation when you replace $u$ by temperature $T$.

To understand this equation we need a vector field.
Here our vector field ${\bf F}$ is the flow of particles.

Physics tells us the the flow will be from higher concentration of particles to lower concentration of particles.
That is, the flow is in the direction where the concentration decreases the fastest. 
And we know that flow decreases fastest in the opposite direction of the gradient so ${\bf F} = - k \nabla u$ where $k$ is the proportionality constant.

Now we need to understand the second part which is once we know how the flow is going how does it affect the concentration of particles. 
That is, we need to relate ${\bf F}$ and \ilds{\px{u}{t}} and that part is Divergence Theorem.

Let's take a small region in space called $D$ and having a surface $S$.
Let's take the flux of ${\bf F}$ outof $D$.
$$
Flux = \oiint_S {\bf F} \cdot \hat{n} dS
$$

The flux is the total amount of particles through $S$.
Thus it is the rate of change of the total particles $D$ over time.
Except we count flux as positive if it is going out, that is total particles in $D$ are decreasing.
So the flux is negative of the rate of change of total particles in $D$.

$$
Flux = - \diff{}{t} \left( \iiint_D u dV \right)
$$

Now using the Divergence Theorem tells us that :

$$
Flux = \oiint_S {\bf F} \cdot \hat{n} dS = \iiint_D div({\bf F}) dV = \iiint_D - k \nabla \cdot (\nabla u) dV 
$$

Thus, we get
$$
    \diff{}{t} \iiint_D u dV = \iiint_D k \nabla^2 u dV
$$

On the left hand side, we are taking the derivative of the total smoke in each box $dV$. 
We can also take the sum of the derivatives in each little boxes $dV$.
$$
\iiint_D \px{u}{t} dV = \iint_D k \nabla^2 u dV
\Rightarrow
\iiint_D \left( \px{u}{t} - k \nabla^2 u \right) dV = 0
$$

Now this is true for any region $D$. That can only be true for every region if the function inside is zero everywhere.
Thus,
$$ 
\px{u}{t} = k \nabla^2 u
$$
