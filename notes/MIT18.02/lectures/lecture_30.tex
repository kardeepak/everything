
%----------------------------------------------------------------------------------------
%	Lecture 30
%----------------------------------------------------------------------------------------

\chapter{Maxwell's Equation}

\bigbreak

\section{Curl}

Earlier we saw that the curl of a velocity fields gives us twice the angular velocity of the flow.
In physics, we have torque which is responsible for the rotational motion of objects.

In Translation Motion : Force / Mass = Acceleration = Rate of Change of Velocity

In Rotational Motion : Torque / Moment of Inertia = Angular Acceleration  = Rate of Change of Angular Velocity

Just as the curl relates velocity to angular velocity, it also relates acceleration to angular acceleration
and force fields to torque fields.

So if the curl of a force field is zero then the force field will not cause any rotational motion.
That is, if force ${\bf F}$ is derived from a potential then its curl is zero 
so it does not generate any rotational motion.

\section{Electromagnetism} 

Let's say we have an electice field ${\bf E}$ and a magnetic field ${\bf B}$.
Then they are related to each other. And their relationship is given by Maxwell's equations.

The force by the electric field is ${\bf F} = q {\bf E}$ where $q$ is the charge of the particle.
The force by the magnetic field is given by ${\bf F} = q {\bf v} \times {\bf B}$ where $q$ is the charge of the particle and ${\bf v}$ is the velocity of the particle.
 
Maxwell's equations give you the divergence and curl of the electric and magnetic fields.

\subsection{Gauss - Couloumb Law} 

$$ \nabla \cdot {\bf E} = \frac{\rho}{\epsilon_0} $$

This law says that the divergence of the electic field is proportional to the charge density $\rho$.
Here, $\epsilon_0$ is a constant of nature.

This is not useful in itself.
But if we want to find out the flux of the electic field through some surface, 
then we can apply the divergence theorem to get the following.

$$ 
\oiint_S {\bf E} \cdot \hat{n} dS 
    = \iiint_D \nabla \cdot {\bf E} dV 
    = \iiint_D \frac{\rho}{\epsilon_0} dV 
    = \frac{\text{Total Charge Inside The Region}}{\epsilon_0}
$$


\subsection{Faraday's Law}

$$ \nabla \times {\bf E} = - \px{{\bf B}}{t} $$

This law tells us that the curl of the electric field is equal to the negative rate of chagne of the magnetic field.
So the electric field is not conservative if you have a variable magnetic field. 

To make sense of it we shall use the Stokes' Theorem.
To compute the work done by the electric field along a closed curve.

$$
\oint_C {\bf E} \cdot d{\bf r}
    = \iint_S (\nabla \times {\bf E}) dS 
    = - \iint_S \px{{\bf B}}{t} dA
$$

This equation tells you that if you have a variable magnetic field then it creates a electric field out of nowhere.

\subsection{Equations For Magnetic Field}

$$ \nabla \cdot {\bf B} = 0 $$

$$ \nabla \times {\bf B} = \mu_0 {\bf J} + \epsilon_0 \mu_0 \px{{\bf E}}{t} $$