
%----------------------------------------------------------------------------------------
%	Lecture 13
%----------------------------------------------------------------------------------------

\chapter{Applications of Double Integral}

\bigbreak

\section{Finding the area of a region}

Area of the region can be thought of as the sum of all the areas of the smaller rectangles.
$$ Area(R) = \iint_R 1 dA $$

\section{Finding the mass of a flat object given its density}

$$ Mass = \iint_R \rho(x, y) dA $$

Here $\rho$ is the density per unit area of the object. 
The density may be a constant or a function of position.

\section{Average Value of a quantity in a region}

In mathematical sense the average is the sum of the value of the function everywhere divided by the area of the region.

$$ Avg(f) = \frac{1}{Area(R)}  \left( \iint_R f(x, y) dA \right) $$

\section{Weighted Average of a quantitly in a region}

In the average defined in the last section, every point is equally likely.
But if we want to give more value to some sub-region then we will use weighted average. 
We'll defined a density function $\rho(x, y)$ who's value is the weight of each point.

$$  \text{Weighted Average}(f) = \frac{1}{Mass(R)} \left( \iint_R \rho(x, y) f(x, y) dA \right) $$

\section{Finding Center of Mass of an object}

The center of mass is just the weighted average of position vector where the weight is the density of the object.
So,
$$ x_{CM} = \frac{1}{Mass} \left( \iint x \rho(x, y) dA \right) \quad \text{and} \quad y_{CM} = \frac{1}{Mass} \left( \iint y \rho(x, y) dA \right) $$

\section{Finding Moment of Inertia}

Mass is how hard it is to impart a translation motion.
Moment of Inertia is defined about an axis and measures how hard it is to impart a rotational motion.
For a point mass, it is defined as $(Mass) * (Radius) ^ 2$ where the radius is the perpendicular distance from the axis.
Thus,

$$ \text{Moment Of Inertia} = \iint r^2 \rho(x, y) dA $$ 

Moment of intertia is denoted by $I_A$ where $A$ is the axis of rotation. 
The rotation kinetic enery is $\frac{1}{2} I_A \omega^2 $.

{\bf Example :} Given a disk or radius $a$ with its center at origin with density 1. Find its moment of intertia about the z-axis.

$$ 
I_A = \iint r^2 dA = \int_0^{2\pi} \int_0^{a} r^2 r dr d\theta 
    = 2\pi \frac{a^4}{4} = \frac{\pi a^4}{2}
$$