
%----------------------------------------------------------------------------------------
%	Lecture 15
%----------------------------------------------------------------------------------------

\chapter{Vector Fields and Line Integrals}

\bigbreak

\section{Vector Fields}

A vector field are just vector ${\bf F} = \ij{m}{n}$ but their components $m(x, y)$ and $n(x, y)$ are functions of position.
It is like at every point there is a vector ${\bf F}$ that depends on $x$ and $y$.

A good example of a vector field is the speed of wind at each point on the plane.
Another example are force fields such as gravitation fields and electric fields.

\section{Work and Line Integrals}

For a force field, the work done by the force is the product of the force and the distance travelled by the object in the direction of the force.
But for a complicated path we want to integrate over the path of the object.

So, we have $W = (force)\cdot(distance) = {\bf F} \cdot (\Delta {\bf r})$.

Along some trajectory $C$, work adds up to \ilds{W = \int_C {\bf F} \cdot d{\bf r} = \lim_{\Delta {\bf r}_i \to 0} \sum {\bf F} \cdot (\Delta {\bf r}_i) }

Let's say we take a time interval of $\Delta t$ over which we calculate $\Delta r$. So our integral becomes, 

$$ W = \int_C {\bf F} \cdot \diff{{\bf r}}{t} dt $$

Here, \ilds{ \diff{{\bf r}}{t} } is the velocity vector along the curve.

\bigbreak

{\bf Example : } Let's say ${\bf F} = \ij{-y}{x}$. And our curve is $x = t$ and $y = t^2$ in $0 <= t <= 1$.

Now, ${\bf F} = \ij{-y}{x} = \ij{-t^2}{t}$.

And, 
$$\diff{{\bf r}}{t} = \ij{\diff{x}{t}}{\diff{y}{t}} = \ij{1}{2t} $$.

Thus, $${\bf F} \cdot \diff{{\bf r}}{t} = -t^2 + 2t^2 = t^2$$

So, our integral is $$ \int_0^1 t^2 dt = \frac{1}{3} $$ 


{\bf Note : } Our trajectory can be anything, so the line integral will depends on the trajectory.

Another way to evalute such integral is to write ${\bf F} = \ij{m}{n}$ and $d{\bf r} = \ij{dx}{dy}$
So our integral becomes $$ \int_C mdx + ndy $$.

Now we can't just integrate with respect to $dx$ as we will get a function of $y$.
Notice that along the curve $x$ and $y$ are related so we need to write $x$ and $y$ as function of another variable $t$.
So \ilds{ dx = \diff{x}{t} dt }  and \ilds{dy = \diff{y}{t} dt}.
Thus, 
$$ \int_C (m\diff{x}{t} + n\diff{y}{t}) dt $$

We can also choose to represent $y$ as a function of $x$ so \ilds{dy = \diff{y}{x} dx}.
Thus, $$ \int_C (m + n \diff{y}{x}) dx $$.

{\bf Note: } The line integral depends on the curve but not on the parametrization. 


\subsection{Geometric Approach}

We can say that the vector $d{\bf r}$ has a magnitude of $ds$ and is in the direction of the tangent to the curve $\hat{T}$.
So we can write the line integral as $$ \int_C {\bf F} \cdot \hat{T} ds $$
This is easier to reason about if the vector field and the curve are related in a geometric sense.

{\bf Example : } Let the vector field be ${\bf F} = \ij{x}{y}$ and the curve is a circle with radius $a$ centered at the origin going in the anticlockwise direction.

We know that the tangent to the circle at any point will be perpendicular to the radius vector of that point.
Since the vector field at that point is equal to the radius vector, their do product will be ${\bf F} \cdot \hat{T} = 0$.

So the integral \ilds{\int_C ({\bf F} \cdot \hat{T}) ds  = 0}

{\bf Example : } Let the vector field be ${\bf F} = \ij{-y}{x}$ and the curve is a circle with radius $a$ centered at the origin going in the anticlockwise direction.

Now if we draw the vector field, it will be perpendicular to the radius vector at every point so the dot product of ${\bf F} \cdot \hat{T} = a$.
The dot product is positive because the vector field points in the anticlockwise direction which is the same as the curve.
So our line integral will be integration over a constant $a$ along a length equal to the circumference of the circle.
Thus, the final answer will be $2 \pi a^2$.