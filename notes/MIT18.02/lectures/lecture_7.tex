
%----------------------------------------------------------------------------------------
%	Lecture 7
%----------------------------------------------------------------------------------------

\chapter{Gradient and Directional Derivative}

\bigbreak
\section{Gradient Vector}

We define the Gradient vector as $\nabla w = \left< w_x, w_y, w_z \right>$.
This vector depends on $x, y, z$ so it varies from point to point. 
This represents a vector at each point $(x, y, z)$. This is what  we'll later call a vector field.

{\bf Theorem : } $\nabla w$ is perpendicular to the level surface $w(x, y, z) = constant$.

{\bf Example : } Let $w = a_1 x + a_2 y + a_3 z$. 
Its leve surface is $a_1 x + a_2 y + a_3 z = c$. 
And the normal vector to this plane is $\left< a_1, a_2, a_3 \right>$.
Now, $\nabla w  = \left< a_1, a_2, a_3 \right>$, hence proved.

This is the only case we need to prove because you can replace the level surface with its linear approximation, that is, its tangent plane.
And the tangent plane is perpendicular to the gradient vector.

{\bf Example : } Let $ w = x^2 + y^2 $.
$\nabla w = \left< 2x, 2y \right> = 2 \cdot \left< x, y \right> $. 
Now the level surface of this function are circles and the gradient vector is in the direction of the radius vector.
And we know that the radius vector is perpendicular to the circle at a point. Thus, proved.


{\bf Proof : } Take the curve ${\bf r} = {\bf r}(t)$ that stays on the level surface of $w = c$.
The velocity vector of the curve will be tangent to the level surface.  
Because the curve lies on the level surface and \ilds{ \diff{ {\bf r} }{t} } is tangent to the curve so it is tangent to the surface as well.

By chain rule, \ilds{ \diff{w}{t} = (\nabla w) \cdot \diff{ {\bf r} }{t} }. 
But the curve lies on a level surface of $w$ so \ilds{ \diff{w}{t} = 0 \implies \nabla w \cdot \diff{ {\bf r} }{t} = 0}.
Thus, $\nabla w$ is perpendicular to the velocity vector. 
Since this is true for every curve on the level surface we can say the $\nabla w$ is perpendicular to the level surface.

Hence, proved.


\section{Directional Derivative}

We know that $w_x$ is the derivative of $w$ when we move in the $x$-direction. 
But what about directions other than the coordinate axis.
Let's say we move in the direction of a unit vector $\hat{u}$.

Let the curve along the $\hat{u}$ be the curve ${\bf r}(s)$ and  \ilds{\diff{ {\bf r} }{s} = \hat{u} }.
Now we want to find out \ilds{ \diff{w}{s} }. 
If $\hat{u} = \left< a, b \right>$, then $x(s) = x_0 + as$ and $y(s) = y_0 + bs$.
We can plug this into $w$ and find out the derivative. Or we can use the chain rule.

{\bf Definition : } \ilds{\diff{w}{s}\Big|_{\hat{u}}} is called the directional derivative of $w$ in the direction of $\hat{u}$.
Geometrically, it means that we'll be slicing the curve by a plane that is in the direction of the vector $\hat{u}$ and we'll take the slope of the graph.
Finally, the chain rule implies that \ilds{ \diff{w}{s} = (\nabla w) \cdot \diff{ {\bf r} }{s} = (\nabla w) \cdot \hat{u} }.

With this, we can show that $ (\nabla w) \cdot \hat{i} = w_x $. The same can be shown for $y$ and $z$.

{\bf Corollary : } We have,  \ilds{ \diff{w}{s}\Big|_{\hat{u}} = (\nabla w) \cdot \hat{u} = |\nabla w| |\hat{u}| \cos \theta }. 
Since $\hat{u}$ is a unit vector so $|\hat{u}| = 1$. Thus, the value of the directional derivative only depends on $\theta$.
We can see that the directional derivative is maximum when $\theta = 0$ so the rate of change of the function is maximum in the direction of $\nabla w$.

Similarly, the minimum value of the directional derivative is when $\theta = \pi$. 
That is the direction opposite to the gradient vector.

The function does not change when $\theta = \pi/2$. That is when the unit vector is perpendicular to the gradient vector.
Another way to look at it is that the function stays constant when we move tangent to the level curve of the function. 
Thus, the gradient is perpendicular to any vector that is tangent to the level curve of the function. 
So the gradient vector is perpendicular to the level curve.

On a contour plot, you can find the gradient vector's direction by finding directional perpendicular to the level curve and pointing towards higher values of the function.



