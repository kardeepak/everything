
%----------------------------------------------------------------------------------------
%	Lecture 29
%----------------------------------------------------------------------------------------

\chapter{Summary} 

\bigbreak

\section{Theorems}

{\bf Note : } In all the following theorems, 
the vector field ${\bf F}$ must be defined and differentiable everywhere withing the enclosed region / surface.

\subsection{Green's Theorem}

Given a closed curve $C$ in the $XY$-plane which encloses a region $R$.
and a vector field ${\bf F}$ in the plane. 

$$ \oint_C {\bf F} \cdot d{\bf r} = \iint_R curl({\bf F}) dA $$

The curve $C$ must be going in the counter-clockwise direction.
And $d{\bf r}$ is tangent to the curve.

\subsection{Green's Theorem For Flux}

Given a closed curve $C$ in the $XY$-plane which encloses a region $R$.
and a vector field ${\bf F}$.

$$ \oint_C {\bf F} \cdot \hat{n} ds = \iint_R div({\bf F}) dA $$

Here, $C$ must be going in the counter-clockwise direction.
And $\hat{n}$ must be the normal vector pointing outwards the enclosed region.

\subsection{Divergence Theorem}

Given a closed surface $S$ in space which encloses a domain $D$ in space
and a vector field ${\bf F}$.

$$ \oiint_S {\bf F} \cdot \hat{n} dS = \iiint_D (\nabla \cdot {\bf F}) dV $$

Here, $\hat{n}$ is the normal vector to the surface pointing outwards from the enclosed domain.

\subsection{Stokes' Theorem}

Given a closed curve $C$ in space and any surface $S$ whose boundary is the curve $C$
and a vector field ${\bf F}$.

$$ \oint_C {\bf F} \cdot d{\bf r} = \iint_S (\nabla \times {\bf F}) \cdot \hat{n} dS $$

Here, you must orient the surface and the curve in compatible ways.
