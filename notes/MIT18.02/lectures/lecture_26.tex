
%----------------------------------------------------------------------------------------
%	Lecture 26
%----------------------------------------------------------------------------------------

\chapter{Line Integrals in Space, Curl, Exactness and Potentials} 

\bigbreak

\section{Line Integral in Space}

Let's say we have a vector field ${\bf F} = \ijk{P}{Q}{R}$ and a curve $C$ in space.
So our integral is,
$$ \int_C (\ijk{P}{Q}{R}) \cdot d{\bf r} = \int_C Pdx + Qdy + Rdz $$

We solve this by parametrizing the curve $C$ and converting the above integral into a single variable integral.

\subsection{Test for Gradient Field in Space}

We want to find if for a vector field ${\bf F} = \ijk{P}{Q}{R}$ there exists a function $f(x, y, z)$
such that $P = f_x$, $Q = f_y$ and $R = f_z$.

If yes, then $P_y = f_{xy} = f_{yx} = Q_x$.
This is one condition. We can do the same for $x-z$ and $y-z$.
So, $P_z = f_{xz} = f_{zx} = R_x$ and $Q_z = f_{yz} = f_{zy} = R_y$.

The above criterion work if ${\bf F}$ is defined in a simply connected region.

The above test also works for differentials. That is, if the above condition holds then
$Pdx + Qdy + Rdz$ is an exact differential $df$ for a function $f(x, y, z)$ and vice versa.

{\bf Example : }  For which values of $a$ and $b$ the following differential is an exact differential.
$$ axy dx + (x^2 + z^3) dy + (byz^2 - 4z^3)dz $$

We have $P = axy$, $Q = x^2 + z^3$ and $R = byz^2 - 4z^3$.
So, 
\begin{gather*}
P_y = Q_x \Rightarrow ax = 2x \Rightarrow a = 2 \\
P_z = R_x \Rightarrow 0 = 0 \\
Q_z = R_y \Rightarrow 3z^2 = bz^2 \Rightarrow b = 3
\end{gather*}

Thus, we get $a = 2$ and $b = 3$.
So our differential is exact for a function $f(x, y, z) = x^2 y + yz^3 - z^4$
$$ df = 2xy dx + (x^2 + z^3) dy + (3yz^2 - 4z^3) dz $$


\section{Finding the Potential Function}

\subsection{Using Line Integrals}

For a conservative/gradient vector field ${\bf F} = \left< P, Q, R \right>$, 
we defined the potential as the line integral going from the origin to that point.
$$ f(x_1, y_1, z_1) = \int_C {\bf F} \cdot d{\bf r} + c $$

Here $c$ is the integration constant.
Now since line integrals on conservative fields are path independent so we can take any curve.
Let's take the curve $(0, 0, 0) \to (x_1, 0, 0) \to (x_1, y_1, 0) \to (x_1, y_1, z_1)$.
So our line integral becomes, 

$$ f(x_1, y_1, z_1) = \int_0^{x_1} P(x, 0, 0) dx + \int_0^{y_1} Q(x_1, y, 0) dy + \int_0^{z_1} R(x_1, y_1, z) dz + c$$

Here each of the three lines have two variables fixed.

\subsection{Using Antiderivatives}

Given $f_x$, $f_y$ and $f_z$, we can find $f(x, y, z)$ by integrating one of these.
Let's say $f_x = 2xy$ then by integrating we get $f(x, y) = x^2 y + g(y, z)$.

Now we can differentiate to get $f_y = x^2 + g_y = x^2 + z^3$.
So we get $g_y = z^3$ so by integrating, we get $g(y, z) = yz^3 + h(z)$.
Now by differatiating again with respect o $z$ we get,
$f_z = g_z = 3yz^2 - 4z^3$.
But $g_z = 3yz^2 + h_z$ so $h_z = -4z^3$

Thus, we get $h(z) = -z^4$, $g(y, z) = yz^3 - z^4$ and $f(x, y, z) = x^2 y + yz^3 - z^4$.

\section{Curl in Space}

If a vector field ${\bf F} = \left< P, Q, R \right>$ then
$curl({\bf F}) = \ijk{(R_y - Q_z)}{(P_z - R_x)}{(Q_x - P_y)}$
If ${\bf F}$ defined in a simply connected region then ${\bf F}$ is conservative iff $curl({\bf F}) = {\bf 0}$.
You can think of the curl as $\nabla \times {\bf F}$ that is the cross product between $\nabla$ and ${\bf F}$,
where \ilds{ \nabla = \left< \px{}{x}, \px{}{y}, \px{}{z} \right> }.

Now, 
$$
\nabla \times {\bf F} =
\begin{vmatrix*}
  \hat{i} & \hat{j} & \hat{k} \\
  \px{}{x} & \px{}{y} & \px{}{k} \\
  P & Q & R
\end{vmatrix*}
$$

You can verify the above formula by expanding the determinant.

\subsection{Geometric Interpretation of Curl}

If the vector field represents the velocity of particles, 
then curl measures the rotational component of the motion.
Precisely, the direction of curl represents the axis of rotation and the magnitude is twice the angular velocity.

